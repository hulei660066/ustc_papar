\documentclass[master,oneside,euler,openright,macfonts]{ustcthesis}
% 默认twoside 双面打印
% 将master修改为bachelor, doctor or master
% 要使用adobe字体,添加adobefonts选项
% 要使用Mac系统的字体,添加macfonts选项
% 使用euler数学字体,如不愿使用,去掉euler
% 使用外文写作,请添加notchinese

% 设置图形文件的搜索路径
\graphicspath{{figures/}}

%仅用于本示例文档中显示特殊字符串
\usepackage{xltxtra}
\usepackage{listings}
\usepackage{color}
\usepackage{hyperref}

\definecolor{dkgreen}{rgb}{0,0.6,0}
\definecolor{gray}{rgb}{0.5,0.5,0.5}
\definecolor{mauve}{rgb}{0.58,0,0.82}

\lstset{frame=tb,
  language=Java,
  aboveskip=3mm,
  belowskip=3mm,
  showstringspaces=false,
  columns=flexible,
  basicstyle={\small\ttfamily},
  numbers=none,
  numberstyle=\tiny\color{gray},
  keywordstyle=\color{blue},
  commentstyle=\color{dkgreen},
  stringstyle=\color{mauve},
  breaklines=true,
  breakatwhitespace=true,
  tabsize=3
}

\colorlet{punct}{red!60!black}
\definecolor{background}{HTML}{EEEEEE}
\definecolor{delim}{RGB}{20,105,176}
\colorlet{numb}{magenta!60!black}

\lstdefinelanguage{json}{
    basicstyle=\normalfont\ttfamily,
    numbers=left,
    numberstyle=\scriptsize,
    stepnumber=1,
    numbersep=8pt,
    showstringspaces=false,
    breaklines=true,
    frame=lines,
    backgroundcolor=\color{background},
    literate=
     *{0}{{{\color{numb}0}}}{1}
      {1}{{{\color{numb}1}}}{1}
      {2}{{{\color{numb}2}}}{1}
      {3}{{{\color{numb}3}}}{1}
      {4}{{{\color{numb}4}}}{1}
      {5}{{{\color{numb}5}}}{1}
      {6}{{{\color{numb}6}}}{1}
      {7}{{{\color{numb}7}}}{1}
      {8}{{{\color{numb}8}}}{1}
      {9}{{{\color{numb}9}}}{1}
      {:}{{{\color{punct}{:}}}}{1}
      {,}{{{\color{punct}{,}}}}{1}
      {\{}{{{\color{delim}{\{}}}}{1}
      {\}}{{{\color{delim}{\}}}}}{1}
      {[}{{{\color{delim}{[}}}}{1}
      {]}{{{\color{delim}{]}}}}{1},
}

%%%%%%%%%%%%%%%%%%%%%%%%%%%%%%
%% 封面部分
%%%%%%%%%%%%%%%%%%%%%%%%%%%%%%

  % 中文封面内容
  \title{基于手机主题推荐系统的用户画像模型}%一般情况下扉页和封皮、书脊共用一个标题文本,可以不用定义\spinetitle(仅硕博有用), \covertitle(本硕博均有用)和\encovertitle(仅本科有用)。特殊情况见下。
  %\spinetitle{\small{中国科学技术大学本硕博毕业论文模板示例文档\raisebox{-3pt}{(Beta)}}}
  %特殊情况1:本例中\title命令里含有换行控制字符,这会导致制作书脊的时候出现错误,例如如果你注释掉\spinetitle{...}这一行就会报错。这时需要定义一个不含换行等命令的\spinetitle,这并不表示\spinetitle里不能有任何命令——只能使用有限的命令。
  %特殊情况2:本例中标题过长,所以需要缩小书脊标题的字号。
  %特殊情况3:本例中中英文混排,由于tex竖排的原理限制,中英文基线不重合,所以需要人工调整英文的基线。具体调整量根据不同字体有所不同。
  %\covertitle{中国科学技术大学本硕博毕业\\论文模板示例文档(Beta)}
  %\covertitle{中文题目第一行\\中文题目第二行}
  %不要在此调整封皮字体大小! Do not set Cover Page font size here!
  %特殊情况4:本例中\title中含有多个换行,导致标题超过了两行。根据制本厂规定,封皮标题不能超过两行。因此需要定义封皮使用的标题\covertitle. 如果你注释掉这一行,就会发现封皮不符合规定。
 % \encovertitle{USTC Thesis Template for Bachelor, Master and Doctor User's Guide(Beta)}
  %\encovertitle{English Title Line 1\\English Title Line 2\\English Title Line 3}
  %不要在此调整封皮字体大小! Do not set Cover Page font size here!
  %特殊情况5:仅本科生有用。本科封皮中有英文标题,不超过三行。与上类似。

  \author{胡磊}
  \depart{软件学院}%系别,硕博请用系代号,本科请用全称如
  %\depart{数理化和信息工程系}
  \major{信息安全专业}%专业,硕博请用全称,本科不需要
  \advisor{胡小宇\ }
%  \coadvisor{XXX\ 胡小宇}%第二导师,没有请注释掉
  \studentid{SA13226110}%For bachelor only
  \submitdate{二〇一六年九月}

  % 英文封面内容
  \entitle{The User Profile Based on Phone Theme Recommendation System}
  \enauthor{Lei Hu}
  \enmajor{Information Security}
  \enadvisor{Prof. Wuyang Zhou}
  \encoadvisor{Dr. Sihai Zhang}%另外一个导师
  \ensubmitdate{September 1, 2016}
  
%%%%%%%%%%%%%%%%%%%%%%%%%%%%%%%%%%%%%%%%%%%%%%%%%%%%%%%%%%%%%%%%%%%%%

\begin{document}

  % 封面
  \maketitle

%特别注意,以下述顺序为准,在对应部分添加文档部件,切勿颠倒顺序:
%本科论文的文档部件顺序是:
%    frontmatter:致谢、目录、中文摘要、英文摘要、
%    mainmatter: 正文章节
%    backmatter: 参考文献或资料注释、附录
%硕博论文的文档部件顺序是:
%    frontmatter:中文摘要、英文摘要、目录、符号说明
%    mainmatter: 正文章节
%    backmatter: 参考文献、附录、致谢、发表论文
%%%%%%%%%%%%%%%%%%%%%%%%%%%%%%
%% 前言部分
%%%%%%%%%%%%%%%%%%%%%%%%%%%%%%
\frontmatter
\makeatletter

	%%%%%%%%%%%%%%%%%
	%硕博论文修改这里
	%%%%%%%%%%%%%%%%%
	% 摘要
	 \begin{cnabstract}
信息爆炸使得用户很难有效的从海量的数据中快速获取自己需要的信息,推荐系统凭借精准定位和"千人千面"的个性化服务受到互联网企业的青睐和研究者的重视。本论文讨论了如何构建一个基于手机主题推荐系统的用户画像模块和用户兴趣探索模块。

传统的个性化推荐系统面临着诸多挑战,其中最根本的问题是如何根据企业的商业目标和业务特点来优化推荐系统,具体到手机主题行业,推荐系统需要解决社交化、长尾性、冷启动、动态推荐等一系列综合问题。由此,笔者提出并实现了一种适用于手机主题个性化推荐系统的用户画像模型,本文的主要工作和贡献有:
\begin{itemize}
	\item 实现了推荐系统的用户画像模块。利用信息检索(Information Retrieval)技术从用户注册信息获取到用户的人口属性、职业、地理位置、性别等信息并标签化,不同标签的来源,标签的本身,以及标签与用户之间的共现关系决定着这个标签的初始权重,然后根据用户行为构建相应的AB测试产出标签的实际权重,权重越高则认为该标签对用户影响越大。AB测试显示推荐系统利用用户画像标签进行推荐能显著提升诸如点击转换率等重要指标。
	\item 实现了推荐系统的用户兴趣探索模块。用户兴趣探索通过特征提取技术和用户满意度量化算法,定期更新用户兴趣标签和标签对应的权重。首先,利用用户兴趣特征向量和商品特征向量计算出用户-商品的相关分数。然后,利用用户行为(购买、评分、点赞、划屏频率等)量化用户满意度。一次成功的用户兴趣标签探索,首先应该有很低的相关分数和很高的满意度,其次兴趣标签应该是一个小众兴趣标签。用户兴趣探索能够实时更新用户的兴趣标签,帮助推荐系统持续满足用户的不断变化的需求。
	\item 利用时间因子衰减模型融合用户的长期兴趣和短期兴趣:用户画像针对的是用户的静态信息,代表了用户的长期兴趣,用户兴趣探索针对的是用户的动态信息,代表了用户的短期兴趣,衰减模型法的本质是利用自然遗忘规律对用户的兴趣进行衰减。
\end{itemize}

\keywords{推荐系统\enskip 长尾效应\enskip 动态兴趣\enskip 用户画像建模\enskip 用户兴趣探索\enskip}
\end{cnabstract}

\begin{enabstract}
Information explosion in the new age let it's hard for users to get valuable infomation from the vast amounts of data, so the recommended system begin to go to the middle of the stage because it's precise forecast and Personalized service. So we here to discuss how to modeling users profile model and users interested exploration model for a android phone theme application recommended system.

There are so many weekness of the traditional recommended system, the most import one is how to sell more products, specific for android phone application, the recommended system need to solve Socializing problem, cool start problem, dynamic recommend based on timeline and so on. So the author proposed and implemented users profile model and users interested exploration model which include: 
\begin{itemize}
	\item Realized the use profile model of recommended system, we use information retrieval technology to get use basic information like occupation, location, gender from user registration information, different tag has different weight depending on the way they got, the path of they transfer and the relation between use and tags, the more weight of tag the high of credibility the tag has. AB test show that recommended system can improve click conversion rate rapidly.
	\item Realized the users interested exploration model of recommended system, which using feature extraction technology and user satisfaction scoring algorithm, we maintain a dynamic interesting tags vector space for all user. first, we can get user-item-scores by product users interesting vector metric and items feature metric. Then get the users satisfaction based on users history actions like buying, rateing, clicking and so on. one successful exploration means it has low user-item-relation-scores and high user satisfaction, and the tag also is minority. Experiments show that with the users interested exploration model, the recommended system has more long-tail effect.
	\item Sucessfully put user long term interesting and short term interesting into one model using linear decay algorithm, users profile model contains static infomation of users, users interested exploration model contains dynamic infomation of users interesting, this papar come up with the strategy to balance the static infomation and the dynamic infomation.
\end{itemize}
\enkeywords{recommend system, long-tail, dynamic, user profile, user interest explore}
\end{enabstract}
%此文件中含有中英文摘要
	% 目录
	\tableofcontents
	%默认表格、插图、算法索引名称分别为“表格索引”、“插图索引”和“算法索引”
	%如果需要自行修改lot,lof,loa的名称,请定义
	%\ustclotname{...}
	%\ustclofname{...}
	%\ustcloaname{...}

	% 表格索引
	\ustclot
	% 插图索引
	\ustclof
	%算法索引 
	%如果需要使用算法环境并列出算法索引,请加入补充宏包。
	%\ustcloa
	
	%符号说明,需要加入补充包
	 %\begin{denotation}

\item[] 
\end{denotation}
%不是必需的,如果不想列出请注释掉
\makeatother

%%%%%%%%%%%%%%%%%%%%%%%%%%%%%%
%% 正文部分
%%%%%%%%%%%%%%%%%%%%%%%%%%%%%%
\mainmatter

   
\chapter{绪论}
\label{chap:introduction}
\section{研究背景与意义}
	自互联网诞生以来,用户寻找信息的方法经历了几个阶段。早期的用户主要靠直接记住感兴趣网站的网址来寻找内容,直接促使Yahoo!提出了分类目录系统,将网站分门别类方便用户查询。但随着信息越来越多,分类目录也只能记录少量的网站,于是产生了搜索引擎。以Google为代表的搜索引擎可以让用户通过关键词找到自己需要的信息,但是,搜索引擎需要用户主动的提供显式关键词来寻找信息,因此它不能解决用户的更多的潜在需求,当用户无法精准描述自己的需求时,搜索引擎就无能为力了,于是又催生出推荐系统\citep{recmd-system}。以亚马逊电商官网为代表的推荐系统是一种帮助用户快速发现有用信息的工具,和搜索引擎不同的是推荐系统不需要提供明确的需求,而是通过分析用户的历史行为来给用户画像建模\citep{demo-data}从而主动给用户推荐出能够满足他们兴趣和需求的信息。因此,从某种意义上说推荐系统和搜索引擎是两个互补的工具。搜索引擎满足用户显式的需求,而推荐系统能够在用户没有明确目的的时候帮助他们发现潜在的需要。随着物联网和用户终端设备的发展,人们逐渐从信息的匮乏时代走进了信息的过载时代。无论是作为信息消费者的普通用户,还是作为信息生产者的提供商面临着数据爆炸时代的挑战。作为用户,如何从充斥着大量噪声的大数据中找到自己感兴趣的信息是一件非常耗时费力的事情,笔者曾有过这样的一种购物体验:在淘宝商城购买一台笔记本电脑,花费了一上午的时间才浏览、比较完所有的 thinkpad 品牌商家店面,如\autoref{fig:hl_taobao}。
	\begin{figure}
		\centering
		\includegraphics[width=0.9\textwidth]{hl_taobao}
		\figcaption{淘宝购物搜索图}
		\label{fig:hl_taobao}
	\end{figure}

	而近年来淘宝的交易额增长规模巨大,2005年淘宝交易额为80亿,2010年为4000亿,而到2015年淘宝双十一单日交易额就为912亿元,可见未来几年内笔者的这种关键字搜索+逐条浏览的购物方式已经不再具有可行性。而作为提供商,如何让自己生产的信息不埋没在大数据洪流中而受到潜在用户的充分关注,这也是其所要解决的一个课题,很多企业已经或者正在开发适合本公司的推荐系统(Recommender System)来解决这一矛盾。

	推荐系统广泛应用于电子商务领域,通过分析用户的数据,帮助用户找到喜欢和感兴趣的商品,然后推荐给他们。推荐系统的最大优点在于它能收集用户的兴趣信息并根据用户的不同偏好,主动的为用户做出个性化推荐,而且此推荐信息是动态更新的,也就是说随着时间的推移,用户的兴趣在逐渐改变,推荐系统的推荐结果也会随之改变。因此,推荐系统大大的提高了网站的用户体验,方便了用户对资源信息的查询。推荐系统的主要任务就是联系用户和信息,一方面协助用户发现自己潜在感兴趣的信息从而提升用户的满意度,另一方面让信息针对性的展现在只对它有兴趣的用户面前从而提升商品的转化率,于是实现了消费者和生产者的双赢。

	\subsection{推荐系统的定义}
	推荐系统的研究和很多早期的研究相关,比如认知科学(cognitive science)\citep{cognitive-science},信息检索(information retrieval)和预测理论\citep{Forecast-principle}。随着互联网的兴起,研究人员开始研究如何利用用户对物品行为数据来预测用户的兴趣并给用户做推荐\citep{cf-sn}。推荐系统开始成为一个比较独立的研究问题。到2006年为止推荐系统的研究主要集中在基于邻域的协同过滤算法,目前工业界应用最广泛、最知名的算法应该就是亚马逊开发并使用的协同过滤算法\citep{Amazon-cf}。推荐系统推荐给用户的商品首先不能与用户购买过的商品重复,其次也不能与用户刚浏览过的商品太相关。推荐系统的形式化定义如:设C是所有用户的集合,S是所有可以推荐给用户的主题的集合。实际上,C和S集合的规模通常很大,如上百万的顾客以及上万款手机主题。设函数u()可以计算主题s对用户c的推荐度R,即$u=C\times S \rightarrow R$,R是一定范围内的全序的非负实数,推荐要研究的问题就是找到推荐度R最大的那些主题S*,如\autoref{equ:fromal}。
	\begin{equation}
	\forall c \in C,S^{*}=arg  max_{s \in S} u(c,s)
	\label{equ:fromal}
	\end{equation}

	\subsection{推荐系统的产生与发展}
	随着科学技术与信息传播的迅猛发展,人类社会进入了一个全新的大数据时代,互联网和物联网无处不在的影响着人类生活的方方面面,并颠覆性改变了人们的生活方式,互联网用户既代表了网络信息的消费者,也代表了网络内容的生产者。尤其是随着Web 2.0时代的到来,社交化网络媒体的异军突起,互联网中的信息量呈指数级增长,而由于用户的辨别能力有限,使得其在庞大且复杂的互联网信息中找寻有用信息的成本巨大,这就是所谓的信息过载问题\citep{info-overload, info-overload:1}。搜索引擎和推荐系统的出现为用户解决信息过载提供了非常重要的技术手段。搜索引擎是被动的,用户在搜索互联网中的信息时需要在搜索引擎中输入关键词,搜索引擎根据输入在系统后台进行信息匹配,将与用户查询相关的信息展示给用户。但是当用户无法精确描述自己需求时,搜索引擎就无能为力了。推荐系统是主动的,用户不需要提供明确的需求,而是通过分析用户的历史行为来对用户进行分析,从而主动给用户推荐可能满足他们兴趣和需求的信息。因此搜索引擎和推荐系统是两个互补的技术手段。

	推荐系统概念是1995年在美国人工智能协会\citep{recmd-history}上由CMU大学的教授Robert Armstrong首先提出并推出了推荐系统的原型系统——Web Watcher。随后推荐系统的研究工作开始慢慢壮大。第一个正式商用的推荐系统是1996年Yahoo网站推出的个性化入口MyYahoo。21新世纪推荐系统的研究与应用随着电子商务的快速发展而风起云涌,各大电子商务网站都开发、部署了推荐系统,Amazon公司称其网站中35\%的营业额来自于自身的推荐系统。2006年美国的DVD租赁公司Netflix\citep{recmd-netflix}在网上公开设立了一个推荐算法竞赛并公开了真实网站中的一部分数据,包含用户对电影的评分。Netflix竞赛有效地推动了学术界和产业界对推荐算法的兴趣,很多有效的算法在此阶段被提了出来。

	自从1992年施乐的科学家为了解决信息负载的问题,第一次提出协同过滤算法,个性化推荐已经经过了二十几年的发展。1998年,林登和他的同事申请了item-to-item协同过滤技术的专利,经过多年的实践,亚马逊宣称销售的推荐占比可以占到整个销售GMV(Gross Merchandise Volume,即年度成交总额)的30\%以上。随后Netflix举办的推荐算法优化竞赛,吸引了数万个团队参与角逐,期间有上百种的算法进行融合尝试,加快了推荐系统的发展,其中SVD(Sigular Value Decomposition,即奇异值分解,一种正交矩阵分解法)和Gavin Potter跨界的引入心理学的方法进行建模,在诸多算法中脱颖而出。其中,矩阵分解的核心是将一个非常稀疏的用户评分矩阵R分解为两个矩阵:User特性的矩阵P和Item特性的矩阵Q,用P和Q相乘的结果R'来拟合原来的评分矩阵R,使得矩阵R'在R的非零元素那些位置上的值尽量接近R中的元素,通过定义R和R'之间的距离,把矩阵分解转化成梯度下降等求解的局部最优解问题。与此同时,Pandora、LinkedIn、Hulu、Last.fm等一些网站在个性化推荐领域都展开了不同程度的尝试,使得推荐系统在垂直领域有了不少突破性进展,但是在全品类的电商、综合的广告营销上,进展还是缓慢,仍然有很多的工作需要探索。特别是在全品类的电商中,单个模型在母婴品类的效果还比较好,但在其他品类就可能很差,很多时候需要根据品类、推荐栏位、场景等不同,设计不同的模型。同时由于用户、SKU不停地增加,需要定期对数据进行重新分析,对模型进行更新,但是定期对模型进行更新,无法保证推荐的实时性,一段时间后,由于模型训练也要相当时间,可利用传统的批处理的Hadoop的方法是无法再缩短更新频率,最终推荐效果会因为实时性问题达到一个瓶颈。推荐算法主要有基于人口统计学的推荐、基于内容的推荐、基于协同过滤的推荐等,而协同过滤算法又有基于邻域的方法(又称基于记忆的方法)、隐语义模型、基于图的随机游走算法等。基于内容的推荐解决了商品的冷启动问题,但是解决不了用户的冷启动问题,并且存在过拟合问题(往往在训练集上有比较好的表现,但在实际预测中效果大打折扣),对领域知识要求也比较高,通用性和移植性比较差,换一个产品形态,往往需要重新构建一套,对于多媒体文件信息特征提取难度又比较大,往往只能通过人工标准信息。基于邻域的协同过滤算法,虽然也有冷启动问题和数据稀疏性等问题,但是没有领域知识要求,算法通用性好,增加推荐的新颖性,并且对行为丰富的商品,推荐准确度较高。基于模型的协同过滤算法在一定程度上解决了基于邻域的推荐算法面临的一些问题,在RMSE(Root Mean Squared Error,即均方根误差)等推荐评价指标上更优,但是通常算法复杂,计算开销大,所以目前基于邻域的协同过滤算法仍然是最为流行的推荐算法。

	自推荐系统诞生后学术界对其关注的兴趣度也越来越大。从1999年开始美国计算机学会每年召开电子商务研讨会以来,发表的与推荐系统相关的论文数以千计。ACM信息检索专业组在2001年开始把推荐系统作为该会议的一个独立研究主题。同年召开的人工智能联合大会也将推荐系统作为一个单独的主题。目前为止数据库、数据挖掘、人工智能、机器学习方面的重要国际会议(如KDD、AAAI、ICML等)都有大量与推荐系统相关的研究成果发表。同时第一个以推荐系统命名的国际会议ACM Recommender Systems Conference 于2007年首次举办。在近几年的数据挖掘及知识发现国际会议举办的竞赛中,连续两年的竞赛主题都是推荐系统。2011年的KDD CUP 竞赛中,两个竞赛题目分别为音乐评分预测和识别音乐是否被用户评分(\href{http://www.kdd.org/kdd2011/kddcup.shtml}{www.kddcup2011.org})。2012年的KDD CUP 竞赛中,两个竞赛题目分别为腾讯微博中的好友推荐和计算广告中的点击率预测。(\href{www.kddcup2012.org}{www.kddcup2012.org})

	\subsection{推荐系统的作用}
	推荐系统改变了没有活力的网站与其用户通信的方式。无需提供一种静态体验,让用户搜索并可能购买产品,推荐系统加强了交互,以提供内容更丰富的体验。推荐系统根据用户过去的购买和搜索历史,以及其他用户的行为,自主地为各个用户识别推荐内容。个性化推荐的最大的优点在于它能收集用户特征资料并根据用户特征,如兴趣偏好,为用户主动作出个性化的推荐。而且,系统给出的推荐是可以实时更新的,即当系统中的商品库或用户特征库发生改变时,给出的推荐序列会自动改变。这就大大提高了电子商务活动的简便性和有效性,同时也提高了企业的服务水平。总体说来,一个成功的个性化推荐系统的作用主要表现在以下几个方面:
	\begin{enumerate}[(1)]
	\item 将电子商务网站的浏览者转变为购买者:电子商务系统的访问者在浏览过程中经常并没有购买欲望,个性化推荐系统能够向用户推荐他们感兴趣的商品,从而促成购买过程。
	\item 提高电子商务网站的交叉销售能力:个性化推荐系统在用户购买过程中向用户提供其他有价值的商品推荐,用户能够从系统提供的推荐列表中购买自己确实需要但在购买过程中没有想到的商品,从而有效提高电子商务系统的交叉销售。
	\item 提高客户对电子商务网站的忠诚度:与传统的商务模式相比,电子商务系统使得用户拥有越来越多的选择,用户更换商家极其方便,只需要点击一两次鼠标就可以在不同的电子商务系统之间跳转。个性化推荐系统分析用户的购买习惯,根据用户需求向用户提供有价值的商品推荐。如果推荐系统的推荐质量很高,那么用户会对该推荐系统产生依赖。因此,个性化推荐系统不仅能够为用户提供个性化的推荐服务,而且能与用户建立长期稳定的关系,从而有效保留客户,提高客户的忠诚度,防止客户流失。
	\end{enumerate}

	\subsection{推荐系统与电子商务}
	近几年随着电子商务蓬勃发展,推荐系统在互联网中的优势地位也越来越明显。在国外比较著名的电子商务网站有Amazon和eBay,其中Amazon平台中采用的推荐算法是非常成功的。在国内比较典型的电子商务平台网站有淘宝网、网页云音乐、爱奇艺PPS等。在这些电子商务平台中,网站提供的商品数量不计其数,网站中的用户规模也非常巨大。据不完全统计天猫商城中的商品数量已经超过了5000万。在商品数量如此庞大的电商网站中,如果用户仅仅根据自己的购买意图输入关键字查询只会得到很多用户很难区分的相似结果,也不便用户做出选择。因此推荐系统作为能够根据用户兴趣\citep{user-interest}为用户推荐商品的主要途径,从而为用户在购物的选择中提供建议的需求非常明显。目前比较成功的电子商务网站中,都不同程度地利用推荐系统在用户购物的同时为用户推荐一些商品,从而提高商品的销售额。另一方面,随着以智能手机为代表的物联网推动了移动互联网的发展。在用户在连入移动互联网的过程中,其所处的地理位置信息可以非常准确地被获取,并由此出现了大量的基于用户位置信息的网站。国外比较著名的有Uber和Coupons。国内著名的有滴滴出行和美团网。例如,在美团网这种基于位置服务的网站中,用户可以根据自己的当前位置搜索餐馆、酒店、影院、旅游景点等信息服务。同时,可以对当前位置下的各类信息进行点评,为自己在现实世界中的体验打分,分享自己的经验与感受。当用户使用这类基于位置的网站服务时,同样会遭遇信息过载问题。推荐系统可以根据用户的位置信息为用户推荐当前位置下用户感兴趣的内容,为用户提供符合其真正需要的内容,提升用户对网站的满意度。

	随着社交网络的深入人心,用户在互联网中的行为不再局限于获取信息,更多的是与网络上的其他用户进行互动。国外著名的社交网络有Facebook、Twitter等,国内的社交网络有微信、米聊等。在社交网站中用户不再是单个的个体,而是与网络中的很多人具有了错综复杂的社交关系链。社交网络中最重要的资源就是用户与用户之间的这种联系。社交网络中用户间的关系是多维度的,建立社交关系的因素可能是在现实世界中是亲人、同学、同事、朋友关系,也可能只是网络中的虚拟朋友,比如都是有着共同爱好的会员成员。在社交网络中用户与用户之间的联系紧密度反映了用户之间的信任关系,用户不在是一个个体存在,其在社交网络中的行为或多或少地会受到其他用户关系的影响。因此推荐系统在这类社交网站中的研究与应用应该考虑用户社交的影响。

	现如今推荐系统在很多领域得到了广泛的应用,如出租车推荐、商品推荐、美餐推荐、电影推荐和音乐推荐,几乎囊括了人类的吃住行穿四大领域,团购网站美团网早已经利用推荐系统提供面向不同业务的个性化服务:1,猜你喜欢:美团最重要的推荐产品,目标是让用户打开美团App的时候,可以最快找到用户想要的团购服务;2,首页频道推荐:若干频道是固定的,若干频道是根据用户的个人偏好推荐出来的;3,今日推荐个性化推送:美团的个性化推送的产品,目的是在用户打开美团App前,就把用户最感兴趣的服务推送给用户,促使用户点击及下单,从而提高用户的活跃度;4,品类列表的个性化排序:美团首页的那些品类频道区。

\section{大数据时代下的推荐系统}
	虽然推荐系统己经被成功运用在很多大型系统、网站,但是在当前大数据的时代下,推荐系统的面临的场景越来越复杂,推荐系统不仅需要解决传统的数据稀疏、冷启动和动态兴趣问题,还面临由大数据引发的更多、更复杂的实际问题,例如数以亿计的用户数目和海量用户同时访问推荐系统所造成的性能压力,使传统的基于单节点架构的推荐系统不再适用。同时Web服务器处理系统请求在大数据集下变得越来越多,Web服务器响应速度缓慢制约了当前推荐系统为大数据集提供推荐。基于实时模式的推荐在大数据集下也面临着严峻考验,用户难以忍受超过秒级的推荐结果返回时间。传统推荐系统的单一数据库存储技术在大数据集下变得不再适用,急需一种对外提供统一接口、对内采用多种混合模式存储的存储架构来满足大数据集下各种数据文件的存储。并且传统推荐系统在推荐算法上采取的是单机节点的计算方式也不能满足海量用户行为数据的计算需求。大数据本身具有的复杂性、不确定性也给推荐系统带来诸多新的挑战,传统推荐系统的时间效率、空间效率和推荐准确度都遇到严重的瓶颈。

	\subsection{推荐系统的关键技术}
	分布式文件系统。传统的推荐系统技术主要处理小文件存储和少量数据计算,大多是面向服务器的架构,中心服务器需要收集用户的浏览记录、购买记录、评分记录等大量的交互信息来为单个用户定制个性化推荐。当数据规模过大,数据无法全部载入服务器内存时,就算采用外存置换算法和多线程技术,依然会出现I/O上的性能瓶颈,致使任务执行效率过低,产生推荐结果的时间过长。对于面向海量用户和海量数据的推荐系统,基于集中式的中心服务器的推荐系统在时间和空间复杂性上无法满足大数据背景下推荐系统快速变化的需求。大数据推荐系统采用基于集群技术的分布式文件系统管理数据。建立一种高并发、可扩展、能处理海量数据的大数据推荐系统架构是非常关键的,它能为大数据集的处理提供强有力的支持。Hadoop 的分布式文件系统架构是其中的典型。与传统的文件系统不同,数据文件并非存储在本地单一节点上,而是通过网络存储在多台节点上。并且文件的位置索引管理一般都由一台或几台中心节点负责。客户端从集群中读写数据时,首先通过中心节点获取文件的位置,然后与集群中的节点通信,客户端通过网络从节点读取数据到本地或把数据从本地写入节点。在这个过程中由HDFS来管理数据冗余存储、大文件的切分、中间网络通信、数据出错恢复等,客户端根据HDFS 提供的接口进行调用即可,非常方便。
	
	分布式计算框架。集群上实现分布式计算的框架很多,Spark作为推荐算法并行化的依托平台,既是一种分布式的计算框架,也是一种新型的分布式计算编程模型,是一种常见的开源计算框架。其基于内存的MapReduce算法的核心思想是分而治之,把对大规模数据集的操作,分发给一个主节点管理下的各个分节点共同完成,然后通过整合各个节点的中间结果,得到最终结果。计算框架负责处理并行编程中分布式存储、工作调度、负载均衡、容错均衡、容错处理以及网络通信等复杂问题,把处理过程高度抽象为两个函数: map和reduce。map负责把任务分解成多个任务,reduce负责把分解后多任务处理的结果汇总起来。
	
	推荐算法并行化。大型企业所需的推荐算法要处理的数据量非常庞大,从TB级别到PB级甚至更高,腾讯Peacock主题模型分析系统需要进行高达十亿文档、百万词汇、百万主题的主题模型训练,仅一个百万词汇乘以百万主题的矩阵,其数据存储量已达3TB。面对如此庞大的数据,若采用传统串行推荐算法,时间开销太大。当数据量较小时,时间复杂度高的串行算法能有效运作,但数据量极速增加后,这些串行推荐算法的计算性能过低,无法应用于实际的推荐系统中。因此,面向大数据集的推荐系统从设计上就应考虑到算法的分布式并行化技术,使得推荐算法能够在海量的、分布式、异构数据环境下得以高效实现。

	\subsection{推荐系统算法简介}
	现有的推荐算法类型很多,但是各有各的局限,因此推荐系统经常采用组合推荐算法,即融合了协同过滤推荐、聚类算法和其他算法的组合推荐算法。
	\begin{enumerate}[(1)]
	\item 协同过滤算法。

	利用用户的历史喜好信息计算用户之间的距离,然后利用目标用户的最近邻居用户对评价的加权评价值来预测目标用户对特定手机主题的喜好程度,系统从而根据这一喜好程度来对目标用户进行推荐。协同过滤是基于这样的假设:为一用户找到他真正感兴趣的内容的好方法是首先找到与此用户有相似兴趣的其他用户,然后将他们感兴趣的内容推荐给此用户。协同过滤正是把这一思想运用到手机推荐系统中来,基于其他用户对某一类手机主题的评价来向目标用户进行推荐。基于协同过滤的推荐系统可以说是从用户的角度来进行相应推荐的,而且是自动的,即用户获得的推荐是系统从购买模式或浏览行为等隐式获得的,不需要用户努力地找到适合自己兴趣的推荐信息,如填写一些调查表格等。

	协同过滤的根本原理是,人们可以从和自己有相同品味、习性的人群那里获得高质量的推荐。协同过滤算法主要研究如何聚类具有相似兴趣特征的人群并基于此做出推荐,因为算法本身是基于用户社交群体,因此往往会涉及到大规模的用户行为数据的计算。协同过滤的应用领域也很广:电子商务,金融信贷,搜素引擎,互联网企业,网络社区等需要对用户提供个性化体验的服务商。因为中国现有的人口国情,协同过滤算法往往需要面对亿万级用户和海量的用户-主题交互数据。作为输入数据,一个用户是以一个N维度的向量来表示,N代表所有的主题数量。向量内容可以为正也可为负,分别表示了用户喜欢、讨厌该主题的程度。对于热门主题,给其打分的用户会很多,其分数应该乘以一个因子u得到有效的分数,u代表所有给其打分的用户个数的倒数,大多数用户向量是稀疏的。在协同过滤算法中关键性的一步就是要选择测量的距离,描述集合相似度算法有欧氏距离、闵可夫斯基距离、汉明距离等,其中最常用的有余弦距离公式(cosine similiarity),公式描述如,
	其中$similarity_{uv}$代表用户u与v之间的兴趣相似度,N(u)表示用户u曾经喜欢过的物品集合,N(v)表示用户v曾经喜欢过的物品集合。
	\begin{equation}
	similarity_{uv} = \frac{|N(u)\cdot N(v)|}{||N(u)||\cdot||N(v)||}
	\label{cosine-similiarity}
	\end{equation}
	
	然后利用相似度算法把用户分类成独立的集合,每个用户有且只属于其中的一个集合,对于每个集合,取这个集合最受欢迎的top N 个主题,作为推荐内容推荐给集合的所有用户。大多数情况下协同过滤算法面都临着一个问题:最坏情况下需要遍历所有的用户和所有的主题,算法计算复杂度为O(MN),M是用户数N是主题数,解决方法可以借助一种简单的降维思想加以解决:通过去掉那些非常冷门的主题对N做降维,通过去掉那些非常不活跃的用户对M做降维,计算维度下降的代价是降低了推荐系统的准确性。
	\item 聚类算法

	聚类分析是对于统计数据分析的一门技术,和分类算法一个主要的区别就是聚类不需要人工参与打标签,基于聚类的协同过滤方法,也可以在一定程度上解决传统协同过滤算法用户评分矩阵稀疏和冷启动问题,在降低用户评分矩阵稀疏性的同时提高目标用户最近邻居的查询速度。聚类是把相似的对象通过静态分类的方法分成不同的组别或者更多的子集,这样让在同一个子集中的成员对象都有相似的一些属性,聚类结果不仅可以揭示数据间的内在联系与区别,还可以为进一步的数据分析与知识发现提供重要依据。在结构性聚类中关键性的一步就是要选择测量的距离。一个简单的测量就是使用曼哈顿距离,它相当于每个变量的绝对差值之和。该名字的由来起源于在纽约市区测量街道之间的距离就是由人步行的步数来确定的。聚类模块可以是对用户兴趣属性相似度做聚类,也可以对用户社交属性相似度做聚类,或者俩种兼有。

	在现实社会中人们的兴趣和选择往往受到身边亲朋好友的影响。在互联网中随着诸如国内的腾讯,国外的Twitter等社会网络网站的兴起,如何利用用户的社会属性做推荐是近几年推荐领域比较热门的研究问题。基于社会网络的推荐算法被称为社会化推荐。近几年在工业界已经有了很多社会化推荐系统。最简单的社会化过滤算法是基于邻域的算法。给定用户u,令F(u)为用户u的好友集合,N(u)为用户u喜欢的物品集合。那么用户u对物品i的喜好程度定义为用户u的好友中喜欢物品i的好友个数,如公\autoref{Social-Rec}。
	\begin{equation}
		P_{vi} = \sum_{v\in F(v),i\in N(u)}^{} 1
		\label{Social-Rec}
	\end{equation}

	聚类算法在许多领域受到广泛应用,包括机器学习,数据挖掘,模式识别,图像分析以及生物信息,最常用的k-means算法\citep{recmd-kmeans}表示以空间中k个点为中心进行聚类,对最靠近他们的对象归类。
	\item 基于内容的推荐算法。

	基于内容的推荐是信息过滤技术的延续与发展,它是建立在对手机主题的标签信息上作出推荐的,而不需要依据用户对手机主题的评价意见,需要用机器学习的方法从关于内容的特征描述的事例中得到用户的兴趣资料。手机主题是通过相关的特征的属性来定义,系统基于用户评价对象的特征,学习用户的兴趣,考察用户资料与待预测手机主题的相匹配程度。用户的资料模型取决于所用学习方法,采用了综合决策树、神经网络和基于向量的组合方法。 基于内容的用户资料是需要有用户的历史数据,用户资料模型可能随着用户的偏好改变而发生变化。基于内容推荐方法的优点是:不需要其它用户的数据,没有冷开始问题和稀疏问题。能为具有特殊兴趣爱好的用户进行推荐。能推荐新的或不是很流行的手机主题,没有产品问题。通过列出推荐手机主题的内容特征,可以解释为什么推荐那些手机主题。

	本节利用spark mllib中ALS算法解释基于内容的推荐。首先,给出一个(用户,主题,评分)三元组的数据集,ALS会建立一个user$\ast$product的m$\ast$n的矩阵,其中,m为用户的数量,n为商品的数量。这个矩阵的每一行代表一个用户 ($u_1$,$u_2$,…,$u_9$)、每一列代表一个产品 ($v_1$,$v_2$,…,$v_9$)。用户的打分在0到10之间。但是在这个数据集中,并不是每个用户都对每个产品进行过评分,所以这个矩阵往往是稀疏的,所以需要做预处理将其填满,然后开始训练:
    假设m$\ast$n的评分矩阵R,可以被近似分解成$U*V^{T}$ ,U为m$\ast$d的用户特征向量矩阵,V为n$\ast$d的产品特征向量矩阵,d为用户和商品的特征值的数量。

    \begin{center} 
	$\begin{Bmatrix}
	  & u_1 & u_2\\ 
	p_1 & 8 & 7\\ 
	p_2 & 44 & 39
	\end{Bmatrix} =
	\begin{Bmatrix}
	  & f_1 & f_2\\ 
	p_1 & 0 & 1\\ 
	p_2 & 2 & 3
	\end{Bmatrix} \ast 
	\begin{Bmatrix}
	  & u_1 & u_2\\ 
	p_1 & 10 & 9\\ 
	p_2 & 8 & 7
	\end{Bmatrix}$
	\end{center}

	对于电影类型的手机主题,可以从d个角度进行评价,如主角,铃声,背景,特效4个角度来评价,那么d就等于4。矩阵V由n个product$\ast$d个特征值组成。对于矩阵U,假设对于任意的用户A,该用户对一款手机主题的综合评分和主题的特征值存在一定的线性关系,综合评分=(a1*d1+a2*d2+a3*d3+a4*d4) ,其中$a_{i}$为用户A的特征值,$d_{i}$为之前所说的主题的特征值。ALS算法认为m*n的评分矩阵R可以被近似分解成$U*V^{T}$,得到目标函数:
    \begin{equation}
    L(U,V)=\sum_{i,j}(R_{ij}-U_{i}^{T}V_{j})^{2}
    \label{F-Measure}
    \end{equation}

    其中a表示评分数据集中用户i对产品j的真实评分,另外一部分表示用户i的特征向量和产品j的特征向量,加上正则化参数$\lambda (||U_i||^2+||V_j||^2)$以防止过度拟合,固定V对U求导得到公式:
    \begin{equation}
    U_{t}=R_{t}V_{ut}(V_{ut}^TV_{ut}+\lambda n_{ut}I)^{-1}, i \in [1,m]
    \label{equa-least2}
    \end{equation}

    其中$R_{t}$表示用户i评过的手机主题的评分向量,$V_{ut}$表示用户i评过的手机主题的特征向量组成的特征矩阵。$n_{ut}$表示用户i评过的手机主题数量。同理,固定U,可以得到求解$V_{j}$的公式:
    \begin{equation}
    V_{j}=R_{j}^{T}U_{mj}(U_{mj}^{T}U_{mj}+\lambda n_{mj}I)^{-1}
    \label{equa-least3}
    \end{equation}

    $R_{j}$表示评过手机主题j的用户向量,$U_{mj}$表示评过手机主题j的用户特征向量组成的矩阵,$m_{mj}$表示评过电影j的用户的数量。

    首先用一个小于1的随机数初始化V,根据\autoref{equa-least2}求U,此时就可以得到初始的UV矩阵了,根据计算得到的U和\autoref{equa-least3}重新计算并覆盖V,反复进行以上两步的计算,直到目标函数和小于一个预设的值,或者迭代次数满足要求则停止。
	\item 组合推荐。

	由于各种推荐方法都有优缺点,手机主题推荐采用了组合推荐方式。研究和应用最多的是基于内容的推荐和协同过滤推荐的组合。最简单的做法就是分别用基于内容的方法和协同过滤推荐方法去产生一个推荐预测结果,然后用某方法组合其结果。组合推荐一个最重要原则就是通过组合后要能避免或弥补各自推荐技术的弱点。在组合方式上使用了几种组合思路:加权(Weight):加权多种推荐技术结果。变换(Switch):根据问题背景和实际情况或要求决定变换采用不同的推荐技术。混合(Mixed):同时采用多种推荐技术给出多种推荐结果为用户提供参考。特征组合(Feature combination):组合来自不同推荐数据源的特征被另一种推荐算法所采用。层叠(Cascade):先用一种推荐技术产生一种粗糙的推荐结果,第二种推荐技术在此推荐结果的基础上进一步作出更精确的推荐。特征扩充(Feature augmentation):一种技术产生附加的特征信息嵌入到另一种推荐技术的特征输入中。
	\end{enumerate}

	\subsection{推荐系统面临的问题}
	\begin{enumerate}[(1)]
	\item 特征提取问题。

	推荐系统的推荐对象种类丰富,例如新闻、博客等文本类对象,视频、图片、音乐等多媒体对象以及可以用文本描述的一些实体对象等。如何对这些推荐对象进行特征提取一直是学术界和工业界的热门研究课题。对于文本类对象,可以借助信息检索领域己经成熟的文本特征提取技术来提取特征。对于多媒体对象,由于需要结合多媒体内容分析领域的相关技术来提取特征,而多媒体内容分析技术目前在学术界和工业界还有待完善,因此多媒体对象的特征提取是推荐系统目前面临的一大难题。此外推荐对象特征的区分度对推荐系统的性能有非常重要的影响。目前还缺乏特别有效的提高特征区分度的方法。
	\item 数据稀疏问题。

	现有的大多数推荐算法都是基于用户—物品协同过滤矩阵数据,数据的稀疏性问题主要是指用户—物品评分矩阵的稀疏性,即用户与物品的交互行为太少。一个大型网站可能拥有上亿数量级的用户和物品,用户评分数据总量在面对增长更快的“用户—物品评价矩阵”时,仍然表现出稀疏性,推荐系统研究中的经典数据集MovieLens的稀疏度仅4.5\%, Netflix百万大赛中提供的音乐数据集的稀疏度是1.2\%。这些都是已经处理过的数据集,实际上真实数据集的稀疏度都远远低于1\%。例如, Bibsonomy的稀疏度是0.35\%,Delicious的稀疏度是0.046\%,淘宝网数据的稀疏度甚至仅在0.01\%左右。根据经验,数据集中用户行为数据越多,推荐算法的精准度越高,性能也越好。若数据集非常稀疏,只包含极少量的用户行为数据,推荐算法的准确度会大打折扣,极容易导致推荐算法的过拟合,影响算法的性能。
	\item 冷启动问题。

	冷启动问题是推荐系统所面临的最大问题之一。冷启动问题总的来说可以分为3类:系统冷启动问题、新用户问题和新物品问题。系统冷启动问题指的是由于数据过于稀疏,“用户—物品评分矩阵”的密度太低,导致推荐系统得到的推荐结果准确性极低。新物品问题是由于新的物品缺少用户对该物品的评分,这类物品很难通过推荐系统被推荐给用户,用户难以对这些物品评分,从而形成恶性循环,导致一些新物品始终无法有效推荐。新物品问题对不同的推荐系统影响程度不同:对于用户可以通过多种方式查找物品的网站,新物品问题并没有太大影响,如电影推荐系统等,因为用户可以有多种途径找到电影观看并评分;而对于一些推荐是主要获取物品途径的网站,新物品问题会对推荐系统造成严重影响。通常解决这个问题的途径是激励或者雇佣少量用户对每一个新物品进行评分。新用户问题是目前对现实推荐系统挑战最大的冷启动问题:当一个新的用户使用推荐系统时,他没有对任何项目进行评分,因此系统无法对其进行个性化推荐;即使当新用户开始对少量项目进行评分时,由于评分太少,系统依然无法给出精确的推荐,这甚至会导致用户因为推荐体验不佳而停止使用推荐系统。当前解决新用户问题主要是通过结合基于内容和基于用户特征的方法,掌握用户的统计特征和兴趣特征,在用户只有少量评分甚至没有评分时做出比较准确的推荐。

	\item 马太效应。

	马太效应(Mattnew Effect)是指强者愈强、弱者愈弱的现象,在互联网中引申为热门的产品受到更多的关注,冷门内容则愈发的会被遗忘的现象。很不幸的是推荐系统的出现加剧了互联网商品的马太效应,因为很多商品只有很少的评分,因此很难在推荐系统中应用,导致推荐结果大部分为热门商品。与马太效应相对于的是长尾理论,由美国人克里斯·安德森提出。长尾理论认为,由于成本和效率的因素,当商品储存流通展示的场地和渠道足够宽广,商品生产成本急剧下降以至于个人都可以进行生产,并且商品的销售成本急剧降低时,几乎任何以前看似需求极低的产品,只要有卖,都会有人买。这些需求和销量不高的产品所占据的共同市场份额,可以和主流产品的市场份额相比,甚至更大。
	\end{enumerate}

	\subsection{推荐系统开源项目介绍}
	工欲善其事,必先利器,关于大数据,有很多令人兴奋的事情,但如何分析、利用它也带来了很多困惑。好在开源观念盛行的今天,有一些在大数据领域领先的免费开源技术可供利用。
	\begin{itemize}
		\item Apache Hadoop:Hadoop是一个由Apache基金会所开发的分布式系统基础架构,是一种用于分布式存储和处理商用硬件上大型数据集的开源框架,可让各企业迅速从海量结构化和非结构化数据中获得洞察力。Hadoop的框架最核心的设计就是HDFS和MapReduce。HDFS为海量的数据提供了存储,则MapReduce为海量的数据提供了计算。HDFS有高容错性的特点,并且设计用来部署在低廉的硬件上;而且它提供高吞吐量来访问应用程序的数据,适合那些有着超大数据的应用程序。MapReduce 本身就是用于并行处理大数据集的软件框架,其根源是函数性编程中的 map 和 reduce 函数。它由两个可能包含有许多实例的操作组成。Map 函数接受一组数据并将其转换为一个键/值对列表,输入域中的每个元素对应一个键/值对。
		\item Apache Hive:Hive是建立在 Hadoop 上的数据仓库基础构架。它提供了一系列的工具,可以用来进行数据提取转化加载,这是一种可以存储、查询和分析存储在Hadoop中的大规模数据的机制。Hive定义了简单的类SQL查询语言,称为HQL,它允许熟悉SQL的用户查询数据。同时,这个语言也允许熟悉 MapReduce 开发者的开发自定义的 mapper 和 reducer 来处理内建的 mapper 和 reducer 无法完成的复杂的分析工作,十分适合数据仓库的统计分析。
		\item Apache Spark:Spark是加州大学伯克利分校所开源的类Hadoop的通用并行框架,Spark拥有 Hadoop 所具有的优点;但不同于 Hadoop 的是Job中间输出结果可以保存在内存中,从而不再需要读写HDFS,因此Spark能更好地适用于数据挖掘与机器学习等需要迭代的MapReduce的算法。
		\item Apache Kafka:Kafka 是一种高吞吐量的分布式发布订阅消息系统,它可以处理消费者规模的网站中的所有用户行为流数据。这种用户行为流数据是在现代网络上的许多社会功能的一个关键因素。这些数据通常是由于吞吐量的要求而通过处理日志和日志聚合来解决。对于像Hadoop的一样的日志数据和离线分析系统,但又要求实时处理的限制,Kafka一个可行的解决方案。其目的是通过Hadoop的并行加载机制来统一线上和离线的消息处理,也是为了通过集群机来提供实时的消费。
	\end{itemize}

	\subsection{推荐系统的应用案例}
	近几年随着社会化网络的发展,推荐系统在工业界广泛应用并且取得了显著进步。比较著名的推荐系统应用有:淘宝网的电子商务推荐系统、Youtube的视频推荐系统\citep{recmd-youtube}、网易云音乐推荐系统以及Facebook好友推荐系统。个性化推荐系统具有良好的发展和应用前景。目前,几乎所有的大型电子商务系统,如Amazon、eBay等,都不同程度的使用了各种形式的推荐系统。各种提供个性化服务的Web站点也需要推荐系统的大力支持。在日趋激烈的竞争环境下,个性化推荐系统能有效的保留客户,提高电子商务系统的服务能力。成功的推荐系统会带来巨大的效益。我们每天使用的许多网站中都可找到推荐系统。

	作为全球排名第一的社交网站(\href{https://code.facebook.com/posts/861999383875667/recommending-items-to-more-than-a-billion-people/}{https://code.facebook.com/}),Facebook利用分布式推荐系统来帮助用户找到他们可能感兴趣的页面、组、事件或者游戏等,代表了国外推荐系统的最高发展水平。Facebook中推荐系统所要面对的数据集包含了约1000亿个评分、超过10亿的用户以及数百万的物品,如何在在大数据规模情况下仍然保持良好性能已经成为世界级的难题。Facebook设计了一个全新的推荐系统。Facebook团队之前已经在使用一个分布式迭代和图像处理平台——Apache Giraph。因其能够很好的支持大规模数据,Giraph就成为了Facebook推荐系统的基础平台。在工作原理方面,Facebook推荐系统采用的是流行的协同过滤技术。CF技术的基本思路就是根据相同人群所关注事物的评分来预测某个人对该事物的评分或喜爱程度。从数学角度而言,该问题就是根据用户-物品的评分矩阵中已知的值来预测未知的值。其求解过程通常采用矩阵分解方法。MF方法把用户评分矩阵表达为用户矩阵和物品的乘积,用这些矩阵相乘的结果R’来拟合原来的评分矩阵R,使得二者尽量接近。如果把R和R’之间的距离作为优化目标,那么矩阵分解就变成了求最小值问题。对大规模数据而言,求解过程将会十分耗时。为了降低时间和空间复杂度,一些从随机特征向量开始的迭代式算法被提出。这些迭代式算法渐渐收敛,可以在合理的时间内找到一个最优解。随机梯度下降算法就是其中之一,其已经成功的用于多个问题的求解。SGD基本思路是以随机方式遍历训练集中的数据,并给出每个已知评分的预测评分值。用户和物品特征向量的调整就沿着评分误差越来越小的方向迭代进行,直到误差到达设计要求。因此,SGD方法可以不需要遍历所有的样本即可完成特征向量的求解。交替最小二乘法是另外一个迭代算法。其基本思路为交替固定用户特征向量和物品特征向量的值,不断的寻找局部最优解直到满足求解条件。

	为了利用上述算法解决Facebook推荐系统的问题,原本Giraph中的标准方法就需要进行改变。之前,Giraph的标准方法是把用户和物品都当作为图中的顶点、已知的评分当作边。那么,SGD或ALS的迭代过程就是遍历图中所有的边,发送用户和物品的特征向量并进行局部更新。该方法存在若干重大问题。首先,迭代过程会带来巨大的网络通信负载。由于迭代过程需要遍历所有的边,一次迭代所发送的数据量就为边与特征向量个数的乘积。假设评分数为1000亿、特征向量为100对,每次迭代的通信数据量就为80TB。其次,物品流行程度的不同会导致图中节点度的分布不均匀。该问题可能会导致内存不够或者引起处理瓶颈。假设一个物品有1000亿个评分、特征向量同样为100对,该物品对应的一个点在一次迭代中就需要接收80GB的数据。最后,Giraph中并没有完全按照公式中的要求实现SGD算法。真正实现中,每个点都是利用迭代开始时实际收到的特征向量进行工作,而并非全局最新的特征向量。因此Giraph中最大的问题就在于每次迭代中都需要把更新信息发送到每一个顶点。为了解决这个问题,Facebook发明了一种利用work-to-work信息传递的高效、便捷方法。该方法把原有的图划分为了由若干work构成的一个圆。每个worker都包含了一个物品集合和若干用户。在每一步,相邻的worker沿顺时针方法把包含物品更新的信息发送到下游的worker。这样,每一步都只处理了各个worker内部的评分,而经过与worker个数相同的步骤后,所有的评分也全部都被处理。该方法实现了通信量与评分数无关,可以明显减少图中数据的通信量。而且,标准方法中节点度分布不均匀的问题也因为物品不再用顶点来表示而不复存在。为了进一步提高算法性能,Facebook把SGD和ALS两个算法进行了揉合,提出了旋转混合式求解方法。

	接下来,Facebook在运行实际的A/B测试之间对推荐系统的性能进行了测量。首先,通过输入一直的训练集,推荐系统对算法的参数进行微调来提高预测精度。然后,系统针对测试集给出评分并与已知的结果进行比较。Facebook团队从物品平均评分、前1/10/100物品的评分精度、所有测试物品的平均精度等来评估推荐系统。此外,均方根误差(Root Mean Squared Error, RMSE)也被用来记录单个误差所带来的影响。

	此外,即使是采用了分布式计算方法,Facebook仍然不可能检查每一个用户/物品对的评分。团队需要寻找更快的方法来获得每个用户排名前K的推荐物品,然后再利用推荐系统计算用户对其的评分。其中一种可能的解决方案是采用ball tree数据结构来存储物品向量。all tree结构可以实现搜索过程10-100倍的加速,使得物品推荐工作能够在合理时间内完成。另外一个能够近似解决问题的方法是根据物品特征向量对物品进行分类。这样,寻找推荐评分就划分为寻找最推荐的物品群和在物品群中再提取评分最高的物品两个过程。该方法在一定程度上会降低推荐系统的可信度,却能够加速计算过程。

	最后,Facebook给出了一些实验的结果。在2014年7月,Databricks公布了在Spark上实现ALS的性能结果。Facebook针对Amazon的数据集,基于Spark MLlib进行标准实验,与自己的旋转混合式方法的结果进行了比较。实验结果表明,Facebook的系统比标准系统要快10倍左右。而且,前者可以轻松处理超过1000亿个评分。
\section{研究内容与研究方法}
	推荐系统问题之一是冷启动问题,冷启动问题有三种:用户冷启动、物品冷启动、系统冷启动,本文主要研究用户冷启动问题。经典的算法诸如最近邻的协同过滤算法、PageRank排序算法、关联规则挖掘等算法是给定用户对某些物品的行为数据,给每个用户推荐TOP N个其最喜欢的物品,这种思路对于新注册用户来讲效果不好,因为没有用户行为数据可供分析。解决这个问题的关键是对用户画像建模,实验发现融合用户画像的热门商品推荐是解决冷启动问题的最佳方式。

	推荐系统问题之二是马太效应,即热门商品越来越热,冷门商品越来越冷,在互联网指数级的爆发下信息量极大富余,这更加推动了马太效应的快速形成以及规模的无限扩大,现有的大多数推荐算法更是极大地加速马太效应的形成速度以及规模。我们提出利用用户兴趣探索解决商品的马太效应,提升推荐系统对物品长尾的发掘能力,主要思路是分析用户所有的行为数据,针对冷门商品(冷门商品包含的标签一般是小众标签)的行为会赋予一个倾斜因子,这样会导致兴趣探索标签候选集中的小众标签占大多数,而如果用户对其的满意度也很高,则说明这是一个成功的兴趣探索。这里涉及到的概念包括小众标签的定义和用户满意度的量化,将会在用户兴趣探索章节详细介绍。

	推荐系统问题之三是用户兴趣的动态变化问题,即时效性问题。笔者一直关心的一个问题就是不同系统的用户行为究竟有什么区别,并如何根据这些区别来选择合适的时间参数来预测用户的行为。如nytimes的时效性很短,大部分新闻都是在第一天被很多人关注,而后面就没有人关注了,所以即使很热门的新闻,其生命周期比不热门的新闻长不了太久。其次是blogspot,然后是youtube,最后是Wikipedia,根据用户兴趣时效性可得排序:NYTimes > BlogSpot > Youtube > Wikipedia。其中Wikipedia的斜率很接近最大理论斜率(0.5),这说明Wikipedia的热门的东西完全是因为生命周期长所以才热门,而不是因为在某天特别的火过。因此,正确把握用户兴趣变动的时效性对推荐结果影响很大,本文针对手机主题市场的特点,利用线性衰减算法融合用户画像和用户兴趣探索,其中用户画像代表了用户长期兴趣,用户兴趣探索代表了用户短期兴趣。

\section{论文结构}
	本文的其余正文内容由以下章节组成:
	\begin{itemize}
		\item 第二章首先介绍了推荐系统基本概念和排序模型,包括数据挖掘算法\citep{date-mining}和信息提取技术\citep{info-retrieval}的应用,然后详细介绍了用户画像和用户兴趣探索。
		\item 第三章主要讨论了如何利用用户画像建模解决推荐系统的冷启动问题,从而改善推荐系统的新用户留存率。最后给出了相关的实验结果及分析。
		\item 第四章主要讨论了如何利用用户兴趣探索跟踪用户动态并挖掘用户小众兴趣,从而提升推荐系统的长尾效应\citep{long-tail},文中给出了相关的实验结果及分析。
		\item 第五章是论文的结束语和展望,在对目前工作简要总结的基础上,提出了推荐系统下一步研究的任务和方向。
	\end{itemize}
   \chapter{基于用户画像的推荐系统综述}
	\section{引言}
	自从1992年著名的施乐公司的科学家们为了解决困扰已久的信息负载问题,第一次从概念上提出协同过滤的算法模型。1998年,林登及其同事们成功申请了item协同过滤技术的专利,经过多年的工程实践,美国电商亚马逊公司的工程师们骄傲的宣称:在公司所有的销售量,推荐系统占比已经占到整个Gross Merchandise Volume的百分之三十以上。不久之后的美国公司Netflix,因为其创始人与前任公司签署有若干年内不得从事同行工作的限制,于是通过举办推荐算法优化竞赛绕开限制,用以开发出更好的推荐算法。此次竞赛吸引了数以千计的团队参与角逐,期间进行了上百种的算法模型组合、优化的尝试,虽然Netflix公司为冠军团队支付了百万美金,但回报是Netflix推荐系统的快速发展以及营收的俩位数增长。其中冠军团队凭借Sigular Value Decomposition和Gavin Potter跨界引入的心理学方法进行的组合算法模型,在诸多优秀团队中脱颖而出。其中,矩阵分解的核心是将一个非常稀疏的用户评分矩阵R分解为两个更小的矩阵:只包含User特性的矩阵P和只包含Item特性的矩阵Q,利用P和Q相乘的结果R'来拟合原来的评分矩阵R,使得矩阵R'在R相同位置之间的损失函数值尽量的小,通过定义一个R和R'之间的距离定义(一般为曼哈顿距离),如果矩阵R'是正定矩阵,那么把矩阵分解转化成梯度下降求解的局部最优解,就是全局最优解。与此同时,Pandora、LinkedIn、Hulu等网站在个性化推荐领域都展开你争我抢的竞争势头,使得推荐系统在各个细分行业、垂直领域开始全面开花,都有了不少爆发性进展。但是,对于拥有全品类的综合性购物电商、广告营销,推荐系统的进展还是缓慢,主要原因是因为不同类型的商品,消费者的心态也是不同的,例如大型家电,消费者肯定是先看了又看、选了又选,从价格、定位、功能到噪声比、性价比,大多数都会先做足了调查,才会购买;与此相反,对于日常用品消费者可能眼睛都不眨就购买了,对于这俩种极端的消费情况,推荐系统需要做出截然不同的推荐策略,具体的,单个模型在母婴品类的推荐效果还比较好,但在其他品类就可能很差,很多时候需要根据场景、推荐栏位、品类等不同,设计不同的推荐模型。同时由于用户兴趣随时间会不停的变动,需要一种机制,使得推荐系统能定期对数据进行评估、分析,要命的是对于不同类型的商品有不同的更新频率,这就对推荐系统提出了更加智能化的挑战。还有,如果定期更新模型,则可能会因为计算资源的限制导致无损害推荐的实时性,因为模型训练也要相当cpu计算时间,而传统的Hadoop的方法实在是无法进行大的更新频率,spark框架又因为昂贵的内存限制了其计算容量,最终业务会到达一个数据量,此时的推荐效果会因为实时性问题达到第一个计算瓶颈。推荐算法包括基于人口统计学的推荐\citep{social-filter}、基于商品内容的推荐\citep{content-based}、基于user/item的协同过滤\citep{collab-filter}的推荐等。基于内容的推荐\citep{recmd_content_based}对物品冷启动问题免疫,但是无法解决用户冷启动问题\citep{cold-start},还有过拟合的问题:即在训练集上有比较好的表现,但在实际应用中效果往往不尽人意,推荐系统的通用性和移植性往往比较差,适合针对细分行业下的商品做推荐,一旦换了产品类型,往往需要构建新的模型。基于邻域的协同过滤算法,虽然没有领域知识要求,算法通用性好,但存在有冷启动问题、数据稀疏性问题。

	由此,笔者在实际工程中,针对传统推荐算法的种种弊端,选择了用户画像。伟大的数学家、计算机学家Knuth先生说:如果遇到一个不好搞定的问题,那么就该添加一层中间层,用以屏蔽掉问题。实际上,用户画像作为底层数据仓库和上层推荐系统的缓冲层,起的就是这种作用。

	\section{用户画像的研究现状}
		\subsection{用户画像的组成部分}
		基于内容和用户画像的个性化推荐,有两个实体:内容和用户。需要有一种文本机制联系这两者的东西,我们定义其为标签。内容特征文本化为标签即为内容特征化,用户兴趣文本化标签则称为用户特征化\citep{user-profile,user-profile1,user-profile2,user-profile3,user-profile4}。因此,对于基于用户画像的推荐,主要分为以下几个关键部分:
		\begin{enumerate}[(1)]
		\item 标签库

		标签是联系用户与物品、内容以及物品、内容之间的纽带,也是反应用户兴趣的重要数据源。标签库的最终用途在于对用户进行行为、属性标记。是将其他实体转换为计算机可以理解的语言关键的一步。标签库则是对标签进行聚合的系统,包括对标签的管理、更新等。在用户画像的过程中有一个很重要的概念叫做颗粒度,就是我们的用户画像应该细化到哪种程度。举一个极端的例子,如果“用户画像”最细的颗粒度应该是细到每一个用户每一具体的生活场景中,但是这基本上是一个不可能完成的任务,同时如果用户画像的颗粒度太大,对于产品设计的指导意义又相对变小了,所以把握好画像的总体丰富程度显得异常重要了。可通过调查问卷的形式来减小颗粒度。一般来说,标签是以层级的形式组织的。如体育为一级维度、篮球为二级维度、NBA篮球为三级维度等。

		\item 内容特征化

		内容特征化即给商品打标签。目前有两种方式:人工打标签和机器自动打标签。针对机器自动打标签,需要采取机器学习的相关算法来实现,即针对商品描述文本,生成一系列标签,为商品选取其中匹配度最高的几个标签。这不同于通常的分类和聚类算法\citep{recmd-kmeans}。可以采取使用分词 + Word2Vec来实现,过程:将文本语料进行分词,以空格,tab隔开都可以,使用结巴分词。使用word2vec训练词的相似度模型。使用tfidf提取内容的关键词A,B,C。对每个现存的标签,计算关键词与此标签的相似度之和。取出TopN相似度最高的标签即为此商品的标签。如对于《小羊肖恩》主题包,现有儿童、动漫俩个标签,描述文本有:一部史诗般的二次元欢乐片。经计算“二次元”关键字与现有标签相似度最高,则更新二次元到此商品的标签库中。

		\item 用户特征化

		用户特征化即为用户打文本标签。通过用户的行为日志和一定的模型算法得到用户的每个标签的权重。用户对内容的行为:点赞、不感兴趣、点击、浏览。对用户的反馈行为如点赞赋予权值1,默认为0,不感兴趣为-1;对于用户的浏览行为,则可使用点击、浏览作为权值。对商品发生的行为可以认为对此商品所有标签的行为。用户的兴趣是时间衰减的,即离当前时间越远的兴趣比重越低。时间衰减函数使用1/[log(t)+1], t为事件发生的时间距离当前时间的大小。要考虑到热门内容会干预用户的标签,需要对热门内容进行降权。
		\end{enumerate}

		\subsection{用户画像的构建周期}
		用户画像,即用户信息标签化,就是企业通过收集与分析消费者社会属性、生活习惯、消费行为等主要信息的数据之后,获得用户的数据标签库。构建周期如\autoref{pic:userprofile_process}。
		\begin{figure}
	    \centering
	      \framebox{\includegraphics[scale=0.45]{figures/userprofile_process}}
	      \figcaption{用户画像的构建周期示意图}
	      \label{pic:userprofile_process}
	    \end{figure}
	    \begin{enumerate}[(1)]
	    \item 数据收集

	    数据收集大致分为四类:1、网络行为数据包括页面浏览量、活跃人数、访问时长、浏览注册转化率、注册活跃转换率等。服务内行为数据:点击浏览路径、网页停留时长、滑屏次数、滑屏频率、滑屏时长。用户内容偏好数据:点击、浏览、收藏内容、评价、评分、评论内容、社交内容、品牌偏好等。用户交易数据(交易类服务):购买率、折扣率、导流率、流失率等。收集到的数据没必要是百分之百的准确,大体差不多即可。应用中,具体就是在数据清洗阶段过滤一部分不靠谱的异常值,验证、更新数据这块需要在后面的阶段中建模来再判断,比如某用户在性别一栏填的女,但其语言数据显示其为男的概率更大,根据业务再选择丢弃数据还是更新数据。
	    
	    \item 行为建模

	    该阶段是对收集到数据进行建模,目标是抽象出用户的文本标签,这个阶段不应该再纠结数据的正确性,而是应该注重大概率事件,通过统计学假设检验尽可能地排除用户的偶然行为。这时也要用到数据挖掘算法模型,对用户的行为进行回归预测,比如已有一个线性回归函数:y=kx+b,X 代表用户行为,y是函数拟合的用户喜好度,y'是用户真实偏好,我们通过不断的训练数据,利用参数k和参数b来得出最新损失函数下的值,用以精确模拟y'。

	    \item 用户画像基本成型

	    该阶段是行为建模的深化,需要利用用户的基本属性,如性别、地域、年龄,得出用户更高层的抽象概念:消费能力、忠诚度、活跃度、社交爱好等。因为用户画像永远也无法百分百地拟合现实中的一个人,只能做的就是不断地去减小拟合的损失函数,因此,用户画像需要根据变化的基础数据不断修正已有的更高层的抽象概念,尽可能模拟用户的变化趋势。

	    \item 数据可视化

	    最后是数据可视化分析,这部分是最能体现推荐系统的产出,因为人类对数据不如对图画来的敏感,在此步骤中一般是针对群体做进一步的抽象,按照消费习惯、消费能力、消费偏好把用户归类为一类人,比如可以根据用户对价格的敏感度细分出高价值用户、核心用户、高忠诚用户。而决策层所做出的评估也应该是基于某一群体的潜在价值分布。典型的用户画像如\autoref{pic:user_profile}
	    \end{enumerate}
		\begin{figure}
	    \centering
	      \framebox{\includegraphics[scale=0.4]{figures/user_profile}}
	      \figcaption{用户画像示意图}
	      \label{pic:user_profile}
	    \end{figure}

		\subsection{用户画像的建模}
		用户画像的建模包括内容标签化和标签权重量化。建模过程:1、内容分析,从原先的物品描述信息中提取有用的信息用一种规范化的标签表示,有时候这种信息源自于作者提供的描述,有时候源自于用户的评价,不管如何,都需要人工审核验证正确性;2、上传、记录用户注册信息,生成用户基本信息,这些信息基本是不会变化的;上传、记录用户行为数据,这些数据是不断变化着的,通常是采用数据挖掘算法从潜在物品集合中取出若干个结果表示用户喜好的模型。例如,一个网页推荐系统,可以通过分析用户过往浏览过的文章,得出用户喜欢浏览类似于范冰冰的花边新闻,如果用户点击了所推荐的文章,则说明分析正确,否则需要根据反馈重新训练模型,从而实现一个反馈-推荐-反馈的闭环;3、推荐系统得出推荐集合后往往需要取topN,因为推荐系统的本质在精不在多。通过定义一个距离算法,匹配用户标签和商品标签的相关度,相关度一般正则为0-1之间,结果是一个二元的离散量:<pid,score>。根据相关度将生成一个用户潜在感兴趣的物品评分列表,然后去掉用户之前看过的商品,取topN即可。例如在电影用户画像的建模中,首先分析用户打分比较高的电影的共同特性,包括导演、演员、风格等,这些电影的标签就会成为此用户画像的一部分,根据打分的多少,给定一个合适的权重值。用户-标签用矩阵A表示,电影-标签用矩阵B表示,A乘B得出矩阵C,C代表了用户与电影之间的相关度,固定一个用户,对所有相关度不为零的电影做排序,取topN即是推荐结果。用户画像建模的根本在于用户标签的获取和权重的定量分析。

		对于商品描述,也可以做进一步的处理,丰富商品的标签集合。其实和文本处理类似,笔者选择使用目前应用最广泛的方法:TF-IDF方法。设有N个文本文件,关键词$k_{i}$在$n_{i}$个文件中出现,设$f_{ij}$为关键词$k_i$在文件$d_j$中出现的次数,那么$k_i$在$d_j$中的词频TF$_{ij}$定义为:TF$_{ij}$=$f_{ij}$/max$_zf_{zj}$,其中分母中的最大值是通过计算这个文本j中所有关键词出现的频率得出。附图给出了3个短文和5个关键词,以关键词人为例,该关键词在文本1中出现了1次,而文本1中出现次数最多的关键词是事,一共出现了2次,因此TF$_{11}$=0.5。一个关键词经常在许多文件中出现,则该关键词能表示文件的特性的意义就会较小,试想我们考察关键词i出现次数的逆,也就是$IDF_{i}$=log(N/$n_{i}$),这个想法和Adamic-Adar指数思路基本相似,关键词i在文本文件j中的权重于是可以表示为$w_{ij}$=$TF_{ij}$*$IDF_{i}$,而文件j可以用一个向量$d_{j}$=($w1_{j}$,$w2_{j}$,…,$wk_{j}$),其中k是整个文本库中关键词的个数。一般而言,向量应该是一个稀疏向量,即其中很多元素都为0。如果把用户今日点击、浏览、购买的商品抽象成一个标签向量,则可以通过用户标签向量-商品标签向量的点乘得出一个数值,从所有数值中把相似性最大的那个产品的标签更新给该用户画像,第二大相似性的产品标签权重减半更新给该用户画像,以此类推,完成用户画像的建模过程。
		\begin{lstlisting}
		文本1:不做软事,不说硬话,对事不对人。
		文本2:多少事,从来急;天地转,光阴迫。一万年太久,只争朝夕。
		文本3:青春之所以幸福,就因为它有前途。

		关键字包括人、事、硬话、一万年、朝夕、青春、幸福、前途
		\end{lstlisting}

		\subsection{用户画像和推荐系统的评测}
		首先,用户画像作为一个工具,只用在运用到某一场景才有意义,并能评估出其产出,因此本节主要介绍推荐系统的评测,根据推荐系统的表现好坏才能评估出用户画像的推荐质量,本节介绍评测推荐系统常用的实验方法。
			\begin{enumerate}[(1)]
			\item 离线实验,从日志系统中直接取得用户最近单位时间的行为数据,然后将这些数据分成俩部分:训练数据和测试数据,一般来讲倆者的比例大致为:8比2,然后利用训练数据集迭代拟合用户的兴趣模型,在测试集上进行回归测试。过程简单、容易模式会管理,不需要人为干预,有很多的数值计算的开源软件库可以用:如google公司出品的TensorFlow。能方便快捷的测试大量不同的算法。
			\item 用户调查,又叫问卷调查。离线实验得出的是客观规律下的准确率,但是客观的准确率不等于用户实际的满意度,一般来说问卷调查需要只需要在小范围之内进行,即可得出大差不差的用户满意度调查,优点是可操作性强。
			\item AB测试,标准的AB测试是指通过一定的规则把类似的用户群随机分成俩组,采用旧模型的分组叫对照组,采用新模型的分组叫实验组。通过对用户展示不同的模型,得出用户的使用指标,关键是各种转化率,这样仅仅通过对比倆者的转化率即可得出各个模型的优劣。策略实验的难点在于如何找到合适的实验设计方案。通过时间交错能够在一定程度上减少由时间片带来的误差,这样就有一个难题:  如何选择合适长度的时间片。策略实验往往伴随着携带效应(carry-over effects),也就是上一个时间片的策略会对下一个时间片带来影响。笔者和同事们提出一个方案,当选择适当大的时间片的时候,通过A/A测试的数据调整A/B的结果,具体来说,如果A/A 的结果是 0.4\%, A/B 的结果是 1.2\%. 那么我们认为A/A  是真实的时间片之间的差异, 我们需要用 A/B - A/A 去调整时间片带来的影响,
			\end{enumerate}

	\section{用户画像在推荐系统的应用现状}
	Amazon的仓库里堆着数百万图书,Netflix的服务器中存储有数万部电影,淘宝平台上的小卖家总共拥有8亿件物品,除此之外,这三家公司都保留有数以亿计的用户行为数据。互联网电子商务开始积累了海量的用户数据,然后因为数据量过于庞大,有用信息如金矿中的金子一样很难挖掘利用,与此同时,用户发现常常需要面对过多的选择。心理学研究证实过多的选择会使人犹豫不决,导致消极等待,最终可能放弃消费的决定,这个问题严峻到可以造成肉眼可见的用户流失。近代统计学理论的发展加上最近几年的数据科学和数据挖掘工程的进步,为电子商务平台提供更有效的应对方案:推荐算法。推荐系统在帮助用户解决信息过载问题的同时,提升了企业价值。如今的企业不再局限于传统的推荐功能,通过建立完备的用户画像,推荐系统可以帮助企业更了解用户,在推广、反作弊、精细化运营等领域中发挥重要的作用。

	目前使用最广的推荐系统,主要是基于内容做推荐,根据RecSys大会(ACM Recommender Systems)中与会者的反馈,已经有不少公司和研究者先行一步,尝试基于用户画像做推荐。利用用户的画像,结合空间、时间、天气、环境、经纬度等上下文信息,可以给用户带来不一样的感受。用户画像是一种更高级的工具,在解决把数据转化为商业价值的问题上更甚一筹,相当于从海量数据中挖出俩倍的金银。用户画像中包含着高质量多维的数据,用以记录用户长期的行为,据此还原用户真实的消费特征、教育背景、兴趣偏好。科学中国网曾在《大数据揭秘:淘宝上的假货、次品都卖给了谁?》中报道了淘宝不良商家如果利用买家信息欺骗消费者\citep{liar_taobao}:1、分析数据看人下刀,宰用户没商量,真相就是消费者的消费记录、购买记录、客单价等都将作为参考数据被系统识别,商家会根据这些记录评估消费者能不能分辨假货,再把假货卖给对方。2、看退货率,专欺负老实人,消费者的退货率、投诉率也会被识别到系统里,这些数据帮助商家判断用户好不好惹,退货率低于百分之十的用户,会收到更多次品产品。3、看收货地址,决定给用户发什么货,一些淘宝店家还会根据用户收货地址所在城市,决定给用户发什么货。要是用户所在城市没有该品牌的专卖店,或者用户没有购买过该品牌的产品,那系统将会放心的把假货或者仿品发给用户。利用用户画像人们可以做到如此精准的销售,当然上述例子是用户画像极其错误的用法。
		\subsection{基于用户画像的推荐系统的商业应用}
		\begin{figure}
	    \centering
	      \framebox{\includegraphics[scale=0.45]{figures/recmd_facebook}}
	      \figcaption{Facebook个性化推荐用户界面}
	      \label{pic:recmd_facebook}
	    \end{figure}
		作为全球社交网站中的翘楚,Facebook在很早的时候就预言到了大数据+推荐系统+用户画像的无限前景。Facebook自己的推荐系统就是需要利用分布式计算框架快速的帮助用户找到他们可能感兴趣的人、文章、分析、用户组等。Facebook是个伟大的公司,一直为开源软件贡献着一份力量,最近在其官网就公布了facebook自己的推荐系统原理、性能及使用情况\citep{recmd-facebook}。Facebook的推荐系统需要面对的数据量应该是所有互联网公司中的数一数二,约包含了1000亿级别的评分数、10亿级别的用户数以及百万级别的虚拟商品,如何在如此庞大的数据规模下,仍然保持良好性能已经成为世界级的难题,而facebook解决了。即使是采用现在流行的分布式计算框架,Facebook仍然不可能穷举每一对用户-物品的评分。团队需要寻找效率更高、耗时更少的算法来获得每个用户topN的推荐物品,然后再利用推荐系统计算用户对其的评分,这与我们之前解释的恰好相反。解决方案是利用ball tree数据结构存储商品的权重向量。all tree可以贡献搜索过程10-100倍的加速率,使得推荐结果能够在合理时间内完成,典型的以空间换时间策略。最后,通过分析Facebook给出了一些实验的结果,表明,Facebook的系统比传统系统要快10倍左右。因此可以轻松愉快的处理1000亿级别的评分数据。目前,该方法已经用到Facebook的多个app应用中,包括用户、用户组的推荐。进一步的,为了能够减小系统负担,Facebook只是把稀疏度超过100的用户考虑为候选推荐集合。在初始迭代中,Facebook推荐系统直接把用户历史上喜欢过的主页、群组以及不喜欢的群组都作为输入。最重要的是Facebook还利用ALS算法,从用户获得间接的反馈,这样算是完成推荐-反馈-优化-推荐的一个完美闭环。未来Facebook会继续优化推荐系统,持续改进部分关键模块,包括社交图、用户跳转路径、自动化参数调整以及较好的机器负载均衡策略等。Facebook推荐主页如\autoref{pic:recmd_facebook}。
		
		Facebook的用户画像进展也十分可观,几乎是与推荐系统同步发展。2011年12月,Facebook发布了里程碑式的大数据产品——Timeline,通过开发API接口,允许用户自行编辑个人的时间轴:在什么时间、什么地点做了什么,遇到了谁,可以说在这条时间线记录这个人的全部生活故事。Timeline通过帮用户回忆自己的点点滴滴的同时,完成了用户数据捕获、存储,而一旦拥有了这些历史数据,Facebook就可以做进一步的数据分析、挖掘,这时的Facebook就如同和你从小长大的小伙伴,一个懂你的陌生人。可以说用户留下的数据越多,Facebook就越了解这个人,投放的广告就会更加精准,最终facebook利用庞大的用户数据生态赚足了钱。

		豆瓣网是国内互联网行业中的小清新,美誉度很高,这是一家致力于帮助用户发现美好事物为己任的公司\citep{recmd-douban}。不用费力设置播放列表,也不用费心思考要听啥,打豆瓣电台的推荐栏目,就能获得意想不到的快乐。如初恋般的音乐体验,让用户和音乐不期而遇,豆瓣电台坚持找到符合用户口味音乐。通过高度匹配的推荐结果,豆瓣电台为音乐爱好者提供了这样一种崭新的音乐盛宴,音乐本来就是件轻松的事,豆瓣电台回归了音乐最初定位。豆瓣电台的推荐算法综合用户的各种音乐行为\citep{recmd-doubanFM}。在豆瓣音乐中,通过量化音乐标签,谁喜欢哪些歌手,在听哪些,想听哪些,乐评,豆列等指标,会有相关的权重算法算出一个数值,最开始的时候只是一个最简单的计算公式,在经过多次产品迭代和用户反馈后,得出更靠谱的权重值累加算。其中权重最多的应该是电台本身的红心、踩、跳过、这些显性行为数据。豆瓣电台糅合了包括数据清洗、分析、挖掘、整合、用户画像建模、编辑与运营、后台架构等等大量的因素,如此庞大的架构中,即便是推荐算法也只是现实的一部分。豆瓣电台推荐页如\autoref{pic:recmd_doubanFM}。

		豆瓣电台的用户画像结构大致有俩快:享受时间和购买时间,用户画像的目标用户群体是在线观众,通过用户购买时间差区别并得出分类标签。如通过分析得出此类用户大多数是周末购买音乐工作日静下心慢慢享受。用户年龄、职业和地址,这是根据用户的经纬度和注册信息来对用户进行分类的用户画像建模。用户的经纬度分为一线城市,如北京、上、二线城市,如武汉、深圳、厦门、其他三类,如合肥、呼和浩特。年龄分为小于25岁、26到35岁、36到45岁和46岁以上四类。据统计结果表明:按人群经纬度分布,大致与橄榄球相似,二线城市的人群占中间的大部分,其他城市人数飞速增长;按用户年龄分布,九十后用户占主体地位。同时对情侣关系的用户推荐喜欢度接近的音乐。按活跃程度分布主要分为3档:100次以下;100-300次;300次以上。也可以同时考虑多个维度,包括活跃度、经纬度、年龄段,进行用户画像建模。
		\begin{figure}
	    \centering
	      \framebox{\includegraphics[scale=0.5]{figures/recmd_doubanFM}}
	      \figcaption{豆瓣电台个性化推荐用户界面}
	      \label{pic:recmd_doubanFM}
	    \end{figure}

		\subsection{推荐系统的主要方法}
		推荐系统主要有俩种思路:评分预测和Top-N预测,核心的目标都是找到最适合用户的候选集合s,从候选集合里挑选目标集合是一个非常复杂的非线性优化问题,通常采用的方案是用局部最优近似非线性最优,通过定义一个的损失函数,选取Top-N	\citep{recmd-Next}。
		\begin{enumerate}[(1)]
		\item 协同过滤的推荐

		推荐系统的算法基于统计学、概率论、线性代数、微积分技术,找出用户最有可能喜欢的商品,应该是现代互联网电商的明星应用。目前用的比较广泛的推荐算法还属协同过滤推荐算法,其基本思想是根据与他兴趣相近的用户的选择,得出推荐商品候选集,取topN推荐给目标用户,用维度为m×n的矩阵表示所有用户对所有物品的兴趣值,这个值应该是根据用户历史行为数据得出,值越高表示这个用户越喜欢,利用特殊值0表示没有接触过。图中行向量表示某个用户对所有商品的喜爱程度,列向量表示某个商品对所有用户的吸引程度,因此单个元素$U_{ij}$表示用户i对物品j的喜欢程度。协同过滤分为两个阶段:预测阶段和推荐阶段。预测阶段是基于所有原始集商品,预测这个用户有没有可能对其感兴趣,量化为一个数值,只要值不为零即可归为候选集中;推荐是根据预测结果,先去重后去除消费过的商品,然后根据某种算法去TopN推荐给用户。按照用户-商品数值的得出类型,协同过滤算法分为基于内容的和基于模型俩大类\citep{Wikipedia}。

		协同过滤算法的基本思路是基于一个假设:如果某类用户群对相同商品的打分比较类似,则表明他们的品味从某种程度上类似,由此可以推出在其它类项目的打分也应该类似才对。协同过滤推荐系统先定义好距离计算公式,然后搜索与目标用户相似的其他潜在类似用户,并根据类似用户的打分来量化潜在用户对指定商品的评分,最后选择评分最高的商品列表推荐给用户,同时可以给出令人信服的推荐理由:如某某人也购买过、评价过该商品。这种算法的优点很多:计算简单、精确度较高,能够自圆其说,因此被广泛采用。总之,基于User的协同过滤推荐算法的核心,就是先通过距离计算公式得出类似邻居,然后将最近邻的好评过的商品推荐给目标用户,很简单。

		例如,在\autoref{tab:User-based}所示的用户一商品评分矩阵中,行向量代表用户,列向量代表电影。表中的数值代表用户对电影的评价量化后的值。现在需要预测用户Hanmeimei对电影《教父》的评分(用户maggie对电影《X-Files 要你相信》的评分是缺失的数据)。
		由\autoref{tab:User-based}不难发现,Lane和Pony对电影的评分非常接近,Lane对《暮色3:月食》、《唐山大地震》、《X-Files 要你相信》的评分分别为3、4、4,Hanmeimei的评分分别为3、5、4,他们之间的相似度最高,因此Lane是Hanmeimei的最接近的邻居,Lane对《教父》的评分结果对预测值的影响占据最大比例。相比之下,用户Jackson和maggie不是Hanmeimei的最近邻居,因为他们对电影的评分存在很大差距,所以Jackson和maggie对《教父》的评分对预测值的影响相对小一些。
		\begin{table}[htp]
		\centering
		\tabcaption{用户-物品表}
		\label{tab:User-based}
		\begin{tabular}{ |c|p{2cm}|p{2cm}|p{2cm}|p{2cm}| } \hline
		 & 暮色3:月食 & 唐山大地震 & X-Files 要你相信 & 教父 \\ \hline
		Jackson & 4 & 4 & 5 & 4 \\ \hline
		Marry & 3 & 4 & 4 & 2 \\ \hline
		maggie & 2 & 3 &  & 3 \\ \hline
		Hanmeimei & 3 & 5 & 4 &  \\ \hline
		\end{tabular}
		\end{table}
		\end{enumerate}
		尽管有这么多的优点,协同过滤算法也存在两大问题:1、数据稀疏性。一个大型的电子商务平台一般有百万级别的物品,用户可能接触到的商品占所有商品的百分之一不到,因此用户之间购买过的物品重叠性非常小,以至于没办法做推荐,一个办法是利用算法添补部分值\citep{recmd-slopone}。2、扩展性较差,因为一般来讲,电子商务平台中的商品变动很小,用户流入流出、日益增加、变动很大,基于用户的协同过滤算法需要不停的跟新迭代保证跟上用户变动的步伐。遇到这种情况,可以考虑基于商品的协同过滤算法,其基本思想类似于基于用户的协同过滤算法,只是相似性计算对象是商品,而商品一般变动很小可以忽略不计。如果我们知道物品a和b相似,而一般喜欢a的用户也喜欢b,如果用户A喜欢a,那么我们有很大把握得知A也应该喜欢b,推荐了准没错。而物品之间的相似性比较固定,因此可以一次性计算出物品的相似度,将结果存储到redis中,推荐时查询redis即可。

		\subsection{推荐系统评测的测量指标}
		推荐系统存在三个参与方:用户、物品提供者和平台。好的推荐系统总体来说是一个能令三方共赢的系统。那么如何评价推荐系统功效呢? 从用户角度,推荐系统必须满足用户的需求,推荐的应该是那些令用户感兴趣的、之前又没有遇到过的商品,即推荐精度。推荐系统还应该有预测用户行为的功能,通过历史展望未来,帮助用户发现那些他们原本没机会发现的小众商品,即长尾效应。最后推荐系统也应该能引导用户兴趣,推荐一些商品,虽然与用户兴趣无关,但是用户看见可能会产生兴趣的商品,即惊喜度。从平台角度,推荐系统能够让平台的营收上一个台阶。
			\begin{enumerate}[(1)]
			\item 用户满意度

			用户满意度是最难量化的指标,也是最关键的指标。推荐系统的本意就是让用户满意。量化用户满意度可以采用用户问卷调查,还有一种更直接的方式,就是在推荐结果的侧栏设置俩个按钮,方便用户在线实时反馈意见,据笔者所知豆瓣的推荐物品旁边都有这类按钮,而亚马逊另辟蹊径,利用一些关键性指标衡量用户对推荐系统的满意度,一般用点击率,用户停留时间,转化率等指标来度量。
			\item 预测准确度

			如果是评分机制,则一般通过计算预测结果集合与用户实际消费集合直接的重合度,得出推荐系统的准确度。如果是Top-N推荐,则涉及到关键指标:召回率和准确率。准确率指在所有的推荐结果中有多少个是对的,其所占的比重,以推荐结果集合个数当除数。召回率则是指用户实际消费商品集合中,有多少物品出现在推荐结果中,已用户实际消费商品集合个数当除数。
			\item 覆盖率

			就是指推荐系统有没有照顾小众商品,而不是一个劲的推荐热门商品。方法就是统计推荐结果的类型个数,比上所有商品类型个数,得到的商越大代表覆盖率越好,其他方法就涉及到信息学中的熵和基尼系数。
			\item 多样性

			针对某一个用户,推荐结果中要变化多端,不能一根筋的推荐一种类型。比如电影,如果用户即格斗类的电影,同时又喜欢爱装小清新,那么推荐列表中就应该是两个类型的集合,除此之外,适当添补一些小众电影,三者比例按用户的爱好来推荐,比如用户爱格斗片多一点,文艺片也喜欢,历史片只是偶尔,那么推荐结果中最好也跟这个比例大差不差。
			\item 新颖性

			如果系统推荐的物品其实是用户知晓的,那么这就是一次失败的推荐,完全失去了推荐的意义。一般来讲,用户都是期望推荐一些自己暂时还不知道的商品或者没看过没买过的商品。方法之一是过滤掉用户已经看到过、购买过、点击过的物品,除此之外,物品的平均流行度与其新颖度成反比,越冷门的物品越会给用户新颖的感觉。比如用户是周星驰的粉丝,那么推荐《临岐》就是一个很棒的选择,因为很少人知道这是周星驰出演的。
			\item 信任度

			如果一个用户信任推荐系统,那么他不仅会频繁的选择查看推荐结果,还有适时的与推荐系统互动,包括反馈、评价、提建议等。如果用户信任推荐系统,从而获得更好的个性化推荐,这是一个良性循环。
			\item 实时性

			有时候一个推荐系统的实时性的重要性大过天\citep{temporal-cf},试想一下,如果一个用户要买睡袋,但不知道哪款睡袋好,推荐系统如果这是恰当好处的推荐结果,那么对于用户是很有意义的一件事情。反之,等用户买都买了,推荐系统在作出推荐,只会让用户难堪。
			\end{enumerate}

	\section{本章小结}
	本章简单概述了用户画像的研究现状,讨论了相关的建模过程。然后介绍了推荐系统的主要任务和问题,并从商业应用和学术研究两个角度介绍了推荐系统研究的现状,最后讨论了推荐系统的主要评测指标。
   
\chapter{手机主题推荐系统整体设计与实现}
    \section{前言}
    小米主题应用拥有成千上万款主题包,而一个用户整个活跃周期只能接触不到十分之一的主题,所以我们现在面临的一个问题是,如何帮助用户发现新的主题,这些主题同时满足俩个条件:1、不能和用户之前看过的、购买过的主题包重复。2、不能和用户之看买过的、购买过的主题不相关,而这也是我们开发的手机主题推荐系统所要达到的目标之一。除此之外,手机主题推荐系统要达到的目标之二是帮助第三方设计师推广其作品。手机主题应用本身既不生产主题包,也不消费主题包,我们的存在价值就是提供一个平台,能让用户、设计师和广告商从中受益。每个设计师都希望更多的用户体验、使用他\\她们的主题,尤其是对于刚出道的设计师。得益于个性化推荐系统的投入使用,我们现在可以把更多的主题包直接推送给那些潜在消费者面前。

    本章节主要介绍如何介绍手机主题推荐系统的完整架构。手机主题推荐由推荐模块、用户画像模型、用户兴趣探索模块组成。推荐过程流程为:首先,推荐系统把用户画像模型中兴趣需求信息和推荐主题模型中的特征信息匹配,然后使用排序算法进行计算筛选找到用户可能感兴趣的推荐主题,最后推荐给用户。
    \section{手机主题推荐系统设计}
    
    \begin{figure}
      \centering
        \framebox{\includegraphics[scale=0.6]{figures/hl_recmd_structure2}}
        \figcaption{推荐系统引擎框架总览图}
        \label{pic:hl_recmd_structure2}
      \end{figure}

    推荐系统框架如\autoref{pic:hl_recmd_structure2}。最顶层显示的是推荐系统对外的服务接口。由于不同展位的输入输出参数差异较大,因此这一层没有做过多的抽象,每个展位有自己特定的接口形式。接口层会调用abtest配置模块,对接入的流量按照uuid、城市等维度进行分流量的配置。Abtest配置模块之下,是推荐候选集的生成,排序和业务处理模块。候选集生成和排序模块,除了针对不同展位有不同逻辑以外,对同一展位的不同策略也有不同的逻辑。abtest模块在配置流量策略的时候,可以根据需要单独配置候选集策略和排序策略。从接口层接受到的每次响应请求会打印一些必要的日志,记录这次请求的一些必要的上下文信息以及用户及item相关的特征信息,以便生成用户行为数据。这些日志通过flume传输到HDFS上面。借助Hadoop、Hive、Spark等平台对原始日志进行处理,从而得到需要的各种数据及模型:包括用户的画像信息,用户之间的相似度,item之间的相似度。在推荐系统的候选集生成这一块,重度使用了传统的user based,item based协同过滤算法,协同过滤算法需要在用户行为较丰富的情况下才能奏效。而对于那些行为稀少的用户,需要根据平台的特点进行做好冷启动策略。这里面需要注意的是,推荐系统引入了时间衰减的因子,从而使新的行为起的作用大于老的行为,从结果来看确实对于效果会有提升。

    \subsection{数据集}
    我们的数据来源有俩部分。1、主动推送,推送有两个特点,一个是异步,可以在用户没有使用APP的时候,将消息推送给他,所以可以作为用户召回的一种手段;另一个是快且实时,因此它也是提高用户活跃度的一种方式。主动推送能够确保优秀的新主题包的时候非常及时的到达用户。2、被动响应,就是用户打开应用,跳转到主题推荐页面,这时才会给用户做推荐。这俩种数据来源都包含着用户行为数据,而用户行为数据是我们的手机主题推荐系统主要驱动数据。用户行为数据包括俩类:隐式反馈数据和显示反馈数据,隐式数据包括用户点击、浏览、搜索关键字等,显示数据主要是用户点赞和评星,显示数据在推荐算法中的权重占比很大,因为如果一个用户给一个主题评星为5星,我们就知道用户确确实实是喜欢这个主题,但是如果用户仅仅是浏览了一款主题,我们是没有办法知道用户真实的想法。通过我们的统计发现,显示数据大约占了1.2\%的比重,隐式数据占了98.8\%的比重,所以我们的推荐系统设计、实现主要是基于隐式用户数据的。

    \subsection{候选集的生成}
    通过用户历史行为数据生成推荐列表,我们把相似的主题包放在候选集中。我们主要利用Item-based Collaborative Filtering算法生成候选集,定义N$_u$表示用户u之前喜欢的主题集合,则用户u对主题i的偏好度根据\autoref{ItemCF}可得,
    \begin{equation}
    p(u,i)=\sum \limits _{j\in N(u)}^{} r(u,j)s(i,j)
    \label{ItemCF}
    \end{equation}

    其中,r$_{u,j}$表示用户u对主题j的偏好度,s$_{i,j}$表示主题i和主题j之间的相似度。Item based Collaborative Filtering算法定义俩个主题之间的相似度由集中在这个俩个主题的用户行为数据计算得出。N$_{i}$为看过主题i的用户集合,N$_{j}$为为看过主题j的用户集合,因此,主题i和主题j的相似度计算公式为\autoref{Item-item-similar}
    \begin{equation}
    s(i,j)=\frac{\left | N(i)\cap N(j) \right |}{\sqrt{\left | N(i) \parallel N(j) \right |}}
    \label{Item-item-similar}
    \end{equation}

    根据\autoref{Item-item-similar}可知,如果有很多用户同时看了主题i和主题j,那么主题i和主题j之间的相似度就会很高,不幸的是,这也会导致所有热门主题与所有主题的相似度都很高。通过A/B测试我们得知,根据用户最近行为作出的推荐比根据用户之前行为作出的推荐,点击转换率比例为1.8:1,因此如果用户最近行为和之前行为有冲突,推荐系统应该倾向于根据用户最近行为作出推荐,而不是反过来。

    \subsection{排序}
    排序之前先过滤掉用户之前接触过的主题包,对于剩余的主题,我们利用生成候选集时得到的用户-主题相关度对主题排序,作为最终推荐结果的重要依据,需要注意的是确保每种类型的主题都要或多或少的推荐一俩款,即保持推荐多样性。最后,推荐结果会对用户给出推荐理由,如一个用户之前买过一款《似水流年》的主题包,我们给他推荐了一款《青春不老 我们不散》的主题,给出的解释是:因为您之前购买过《似水流年》

  \section{用户画像与推荐系统}
  一个好的推荐系统要给用户提供个性化的、高效的、动态准确的推荐,那么推荐系统应能够获取反映用户多方面的、动态变化的兴趣偏好,推荐系统有必要为用户建立一个用户兴趣探索模型,该模型能获取、表示、存储和修改用户兴趣偏好,能进行推理,对用户进行分类和识别,帮助系统更好地理解用户特征和类别,这就是我们要引进用户画像的根本原因。用户画像模块和兴趣探索模块的关系如\autoref{pic:hl_iterate}所示。
  \begin{figure}
    \centering
      \framebox{\includegraphics[scale=0.4]{figures/hl_iterate}}
      \figcaption{用户画像架构示意图}
      \label{pic:hl_iterate}
  \end{figure}

  利用用户的画像,结合时间、天气等上下文信息,给用户做一些更加精准化的推荐是一个不错的方向。推荐系统根据用户画像进行推荐,所以用户画像对推荐系统的质量有至关重要的影响。建立用户画像模型之前需要考虑问题有:模型的输入数据有哪些,如何获取模型的输入数据;如何考虑用户的兴趣及需求的变化;建模的对象是谁以及如何建模;模型的输出是什么。用户画像模型的输入数据主构成包括:
  \begin{itemize}
  \item 用户属性,分为社会属性和自然属性,包括用户最基本的如用户的姓名、年龄、职业、收入、学历等信息。用户注册时的对自然属性和社会属性进行初始建模。 
  \item 用户手工输入的信息:是用户主动输出给系统的信息,包括用户在搜索引擎中打出的关键词,用户评论中发布的感兴趣的主题、频道。还有一类重要的信息就是用户反馈的信息,包括用户自己对推荐结果的满意程度;用户标注的浏览页面的感兴趣、不感兴趣或感兴趣的程度等。
  \item 用户的浏览行为和浏览内容:用户浏览的行为和内容体现了用户的兴趣和需求,它们包括浏览次数、频率、停留时间等,浏览页面时的操作(收藏、保存、复制等)、浏览时用户表情的变化等。服务器端保存的日志记录了用户的浏览行为和内容。
  \end{itemize}

  手机主题推荐系统对每个新注册用户会生成一个用户画像,刚开始只是包含最基本的用户人口信息,在维护过程中会逐渐增加用户行为和行为偏好,然后利用离线训练模型生成用户喜好的主题,详细内容将在第四章、第五章展开。
  \section{量化评估推荐系统}
  

  \section{本章小结}
  我们的手机主题推荐系统还遇到一种情况是,有时候推荐时,排序较高的主题不一定是用户需要的,而排序较低的主题有可能是用户期望看到的。例如,一个用户喜欢读弗洛伊德的《梦的解析》,那么他有可能会喜欢《异梦空间》这款主题,但是这款主题很冷门,推荐系统没办法挖掘出其价值。再比如,像《似水流年》这样常月霸占免费排行榜第一、第二位置的热门主题包,我们实际上是不需要对其作任何推广、推荐的。除此之外,我们也希望手机推荐系统能推陈出新,而不是一成不变。为了解决这些问题,本人基于手机主题推荐系统开发了用户画像功能模块和用户兴趣功能模块。
   
\chapter{用户画像模块}
\section{引言}
\label{chap:example}
Alan Cooper最早提出了用户画像的概念:Personas are a concrete representation of target users。Persona 是真实用户的虚拟代表,是建立在一系列真实数据之上的目标用户画像。通过用户历史行为去了解用户,根据他们的目标、行为和观点的差异,将他们区分为不同的类型,然后每种类型中抽取出典型特征,赋予名字、照片、一些人口统计学要素、兴趣标签等描述,就形成了一个人物原型,\autoref{pic:hl_userProfile}所示为一个典型的用户画像,标签面积越大代表其权重越高。

刻画每个用户,是任何一家社交类型的服务都需要面对的问题,不同的公司针对各自业务会有不同的需求,构建用户画像的动机和目标也会存在一定差异。从手机主题应用的业务特点来讲,构建用户画像的目的包括:

\begin{figure}
\centering
  \framebox{\includegraphics[scale=0.35]{figures/hl_userProfile}}
  \figcaption{用户画像标签示例图}
  \label{pic:hl_userProfile}
\end{figure}

\begin{itemize}
\item 完善及扩充用户信息:用户画像的首要动机就是了解用户,这样才能够提供更优质的服务。但是在实际中用户的信息提供得不尽完整,如对于没有填写性别信息的用户,用户画像通过分析用户语音数据识别其性别,尽可能多的为推荐系统提供正确的基础特征。
\item 打造健康的主题设计生态圈:在掌握用户信息的基础上,平台就可以对自身的状况进行分析,从相对宏观的基础上把握主题市场的生态环境,挖掘设计作品的最大价值,帮助设计师提高收入。例如通过对用户信息的聚类,能够对用户进行人群的划分,掌握不同人群的活跃程度、行为及兴趣偏好,热门主题的传播方式和流行引爆点等。
\item 支撑主题推荐系统的精准推荐:精准推荐的前提是对用户的清晰认知。以简单代金券发放为例,手机主题应用市场的历史数据呈现出两大类四种不同的消费习惯。代金券敏感型:发代金券才用、发代金券用的更多;代金券不敏感型:发不发都用,发代金券也不用。在推荐系统的用户画像系统中,上述四种群体会被分别冠以屌丝、普通、中产、土豪的标签。针对四类用户的运营策略也会全然不同,最直接的就是代金券的刺激频率以及刺激金额,而对“代金券”免疫的土豪群体,则更多地需要在优化服务上做文章。在实际场景中,影响用户对手机主题包的使用黏度的因素要远比代金券复杂得多,在这种情况下,利用用户画像可以对用户的“贴身跟踪”就能及时发现薄弱环节,因此从用户打开应用商店到退出使用,其间的每一步情况都被快的记录在案:哪一天退出的,哪一步退出的,退出之后“跳转”到什么软件等等。据此,用户画像也实现了用户另外一个纬度的归类,分清哪部分是忠实用户,哪部分可能是潜在的忠实用户,哪些则是已经流失的;更进一步来看流失的原因:因为代金券没有了流失?主题包质量不好流失?这些都是下一步精准推荐的依据,无论是基于兴趣的推荐提升用户价值,精准的广告投放提升商业价值,还是针对特定用户群体的内容运营,用户画像都是其必不可少的基础支撑。直接地,用户画像可以用于兴趣匹配、关系匹配的推荐和投放;间接地,可以基于用户画像中相似的兴趣、关系及行为模式去推动用户兴趣和设计师的无缝对接。
\item 主题市场安全领域的应用:随着手机主题市场的发展,商家会通过各种活动形式的补贴来获取用户、培养用户的消费习惯,但同时也催生一些通过刷排行榜、刷红包的用户,这些行为距离欺诈只有一步之遥,但他们的存在严重破环了市场的稳定,侵占了活动的资源。其中一个有效的解决方案就是利用用户画像沉淀方法设置促销活动门槛,即通过记录用户的注册时间、历史登陆次数、常用IP地址等,最大程度上隔离掉僵尸账号,保证市场的稳定发展。
\end{itemize}

    \section{用户画像数据类型}
    在个性化服务的用户画像建模中,一个完整、成熟的用户画像应该包含基础静态数据类型、基础行为数据类型和高维数据类型。
    \subsection{基础静态数据类型}
    当一个新用户注册时会填写人口基本信息,通过json格式从客户端传回服务器,格式如
    \begin{lstlisting}[language=json,firstnumber=1]
        {"registerLog": {
          "userId": "001",
          "gender": "male",
          "profession": "student",
          "phone": "null",
          "borthday": "19860820",
          "isWeiboUser": "no",
          "isWeixinUser": "yes",
          "city": "北京市",
          "timestamp": "1453700393",
          "...": "..."
        }}
    \end{lstlisting}
    有的用户会利用微信、微博提供的第三方免登陆API,第三方数据可以用来交叉验证用户填写的基础信息数据。用户每次登陆时应用程序还会获得其手机品牌、操作系统等信息。因此,通过解析server log得到基础静态数据形式:
      \begin{table}[htp]
      \centering
      \tabcaption{用户-基础静态数据矩阵表}
      \label{tab:tagweight}
      \begin{tabular}{|c|c|c|c|c|c|c|c|} \hline
       用户id & 性别 & 年龄 & 职业 & 电话号码 & 手机运营商 & 是否为微博用户 & ... \\ \hline
       001 & 女 & 23 & 学生 & 13948572214 & 移动 & 是 & ... \\ \hline
       002 & 男 & 30 & 学生 & 15811036703 & 移动 & 是 & ... \\ \hline
       ... & ... & ... & ... & ... & ... & ... & ... \\ \hline
      \end{tabular}
      \end{table}

    \subsection{基础行为数据类型}
    基础行为数据是指用户的一些行为,包括购买,试用,浏览,评价等的统计量,用户行为数据格式如
      \begin{lstlisting}[language=json,firstnumber=1]
        {"actionLog": {
          "userId": "001"
          "actions": [{
              {"itermId": "0822"},
              {"actionType": "jumpIn"},
              {"stayTime": "32000"},
              {"clickNum": "2"},
              {"scrollNum": "5"},
              {"timestamp": "1453701393"},
              {"...": "..."}
          }]
        }}
      \end{lstlisting}
    基础行为数据作为用户行为统计量可以反映用户的活跃度、消费能力和用户类型。基础行为数据形式如:
      \begin{table}[htp]
      \centering
      \tabcaption{用户-基础行为数据表}
      \label{tab:tagweight}
      \begin{tabular}{|c|c|c|c|c|c|c|c|} \hline
       用户id & 购买 & 试用数 & 浏览 & 未支付订单数 & 活跃时间段 & 日浏览时长 & ... \\ \hline
       001 & 2 & 7 & 118 & 0 & 20:00-22:00 & 120 & ... \\ \hline
       002 & 0 & 3 & 7 & 1 & 13:00-14:00 & 60 & ... \\ \hline
       ... & ... & ... & ... & ... & ... & ... & ... \\ \hline
      \end{tabular}
      \end{table}

    \subsection{高维数据类型}
    高维数据是用户画像模型从基础静态数据和基础行为数据统计、分析、抽象出来,用来衡量用户某一方面的价值,如用户信用是指是否有过作弊行为、退款次数过多等综合评估,用户价值是指购买次数、单笔消费额、消费频率的综合评估。高维数据可以用矩阵来表示:
      \begin{table}[htp]
      \centering
      \tabcaption{用户-高维数据表}
      \label{tab:tagweight}
      \begin{tabular}{|c|c|c|c|c|c|c|c|} \hline
       用户id & 信用 & 价值 & 忠诚度 & 活跃度 & 价格敏感度 & 奖励敏感度 & ... \\ \hline
       001 & 高 & 高 & 高 & 高 & 低 & 低 & ... \\ \hline
       002 & 中 & 中 & 高 & 高 & 高 & 高 & ... \\ \hline
       ... & ... & ... & ... & ... & ... & ... & ... \\ \hline
      \end{tabular}
      \end{table}

    \section{用户画像建模}
    用户画像建模的过程就是原始数据进过处理、分析得到可信度高的用户标签信息的过程,对于不同类型的用户数据其建模的侧重功能点也有所区别。
    \subsection{基础静态数据建模}
    用户基础静态数据的特点是数量不多,但在推荐系统中所占的权重较大,因此对其可信度要求较高,在对基础静态数据建模的时候主要实现俩个功能:根据上下文信息补全为为空的标签和根据上下文信息校验已有的标签。

    标签补全以用户性别标签为例,新用户注册时如未填写性别信息其值会默认设为Null,方便用户画像建模时判断。主要思路是通过分析用户上下文信息,包括第三方登入数据、用户语音和头像获得用户真实的性别,如以上方法都未成功获取用户性别,程序会利用线性回归算法挖掘出一个最有可能的性别标签值,代码:
    \begin{lstlisting}
      public String getUserGender(String log) {
          Gson gson = new Gson();
          UserProfile userProfile = gson.fromJson(log, UserProfile.class);

          if (userProfile.gender != null) {
              return userProfile.gender;
          }

          String useId = userProfile.useId;
          //通过第三方应用登陆数据得到用户信息
          UserProfile thirdPartUP = gson.fromJson(getThirdPartUserInfo(useId), UserProfile.class);
          if (thirdPartUP.gender != null) {
              return thirdPartUP.gender;
          }

          //通过分析用户语音数据得到用户信息
          UserProfile voiceUP = gson.fromJson(getUserVoiceUserInfo(useId), UserProfile.class);
          if (voiceUP.gender != null) {
              return voiceUP.gender;
          }

          //通过线性回归算法挖掘出用户信息
          UserProfile lrUP = gson.fromJson(getLinearRegressionUserInfo(useId), UserProfile.class);
          return lrUP.gender;

      }
    \end{lstlisting}

    标签校验是指虽然相关信息已经被填写,但程序认为其值具有随意性,需要根据上下文信息加以确认并校验,标签校验由于考虑的因素较多导致计算量大,使得其应用场景较少,还是以用户性别标签为例,代码:
    \begin{lstlisting}
        public String getRightUserGender(String log) {
        int[] count = {0, 0};
        Gson gson = new Gson();
        UserProfile userProfile = gson.fromJson(log, UserProfile.class);

        if (userProfile.gender != null) {
            if (userProfile.gender.equals("male")) {
                count[0]++;
            } else {
                count[1]++;
            }
        }

        String useId = userProfile.useId;
        UserProfile thirdPartUP = gson.fromJson(getThirdPartUserInfo(useId), UserProfile.class);
        if (thirdPartUP.gender != null) {
            if (thirdPartUP.gender.equals("male")) {
                count[0]++;
            } else {
                count[1]++;
            }
        }

        UserProfile voiceUP = gson.fromJson(getUserVoiceUserInfo(useId), UserProfile.class);
        if (voiceUP.gender != null) {
            if (voiceUP.gender.equals("male")) {
                count[0]++;
            } else {
                count[1]++;
            }
        }

        UserProfile lrUP = gson.fromJson(getLinearRegressionUserInfo(useId), UserProfile.class);
        if (lrUP.gender.equals("male")) {
            count[0]++;
        } else {
            count[1]++;
        }
        if (count[0] >= count[1]) {
            return "male";
        } else {
            return "female";
        }
    }
    \end{lstlisting}

    \subsection{基础行为数据建模}
    基础行为数据建模跟新频率较快,计算量较大,因此采用离线方式利用sql语句从hive表中得出用户在一段时间区间内特定行为的统计数据。需要注意一些用户行为的延迟性,如购买行为,从下单到支付成功可能跨越若干天,因此约定订单量以支付时间为准,有时候遇到网络故障相同订单会被用户提交多次,需要利用distinct做去重操作。统计特定用户某段时间的订单量的sql语句:
    \begin{lstlisting}
      set hiveconf:ymdwithline=2016-04-06;
      set hiveconf:userId=525108009;

      select count(distinct a.order_id) score
      from theme_dw.dw_v_order_base
      where concat_ws('-',year,month,day) between date_sub('${hiveconf:ymdwithline}',5) and '${hiveconf:ymdwithline}'
      and userId='${hiveconf:userId}'
      and finish_time like '${hiveconf:ymdwithline}%'
    \end{lstlisting}

    \subsection{高维数据建模}
    高维数据建模的数据来源包括基础静态数据、基础行为数据,数据类型包括累计量和趋势量,累计量包括用户浏览总数、用户购买总数等,趋势量是指用户最近登录时间、最近购买时间等,利用数据挖掘分类算法得出一个训练模型,需要注意的是用户行为类型、发生时间、发生位置会影响模型的权重计算,即weight = (行为类型 + 时间上下文 + 空间上下文) × 时间衰减因子。其中,用户行为类型包括浏览、购买、搜索、评论、购买、点击赞、收藏等,我们定义购买权重计为5,而浏览仅仅为1。空间上下文是指用户跳转入口方式,我们定义搜索入口权重3,排行榜入口为2。时间上下文是指用户之前是否接触过此类标签,接触频率等。时间衰减因子根据半衰期公式得出,如所示\autoref{equ-half_life},其中T取值为1,t为行为发生时间距离当前时间的天数。
    \begin{equation}
      score=(\frac{1}{2})^{(t/T)}
      \label{equ-half_life}
    \end{equation}

    以用户活跃度为例,由于日活跃变动过大,月活跃过于滞后,因此按周统计,模型选择线性回归算法,模型输入为基础静态数据、基础行为数据,模型输出为一个int型整数,值为[1,2,3],分别对应不活跃、较活跃、活跃。代码:
    \begin{lstlisting}
      public int getActivityScore(String userId) throws Exception {
          String userBaseInfo = getUserBaseInfo(userId);
          String userActionLog = getUserActionLog(userId);
          Gson gson = new Gson();
          String score = getLinearRegressionActivityScore(gson.fromJson(userBaseInfo, UserProfile
                  .class), gson.fromJson(userActionLog, UserActions.class));
          double activityScore = Double.parseDouble(score);
          if (activityScore >= 66) {
              return 3;
          } else if (activityScore >= 33) {
              return 2;
          } else {
              return 1;
          }
      }
    \end{lstlisting}

    \section{实验与分析}
    本节的研究目标是如何利用用户画像给新注册用户做出准确的Top-N推荐并提升用户留存率。
      \subsection{数据集准备}
      手机主题应用月新注册用户超过20万个用户,大部分用户的第一个月的行为记录少于10个,我们从2015年9月1号到2015年9月7号这段时间,筛选出所有注册信息相对完整的用户数据作为实验数据集,create table 格式:
      \begin{lstlisting}[language=json,firstnumber=1]
        {
        //静态数据
        user_id                  int    comment '用户id',
        user_name                int    comment '用户name',
        user_age                 int    comment '用户age',
        create_time              string comment '账号创建时间',
        city_id                  int    comment '城市id',
        city_name                string comment '城市名',
        phone                    int     comment '手机号',
        os_version               stringt comment '操作系统及版本',
        phonetype_serial         string  comment '手机品牌及型号',
        education_level          string  comment '学历',
        school                   string  comment '学校',

        //行为数据
        click_num int comment '点击次数',
        last_click_time int comment '最近点击时间',
        buy_num int     comment '购买次数',
        last_buy_time int     comment '最近购买时间',
        try_use int     comment '试用次数',
        last_tryuse_time int     comment '最近试用时间',
        browse_num int     comment '浏览次数',
        last_browse_time int     comment '最近浏览时间',
        browse_total_time int     comment '浏览总时长',
        login_num int     comment '登陆总次数',
        login_total_time int     comment '登陆总时长',
        comment_num int     comment '评论总次数',

        //高维数据
        use_time                 int     comment '使用时间段',
        not_use_time             int     comment '沉默天数',
        friendship               list<bigint>  comment '好友关系',
        friend_group             list<bigint>  comment '好友圈',
        coupon_sensitivity_score decimal(20,4) comment '券敏感及阈值',
        purchase_will_score      decimal(20,4) comment '消费意愿',
        loyal_score              decimal(20,4) comment '忠诚度',
        credit_score             decimal(20,4) comment '活跃度'
        }
      \end{lstlisting}
      \subsection{评测指标}
      本节使用线上A/B测试方案\citep{ab-test},利用用户留存率来评测推荐系统应对冷启动问题的效果。用户留存数是指在某段时间开始使用App应用,经过一段单位时间后仍然继续使用该App应用的用户,用户留存率是指用户留存数占当时新增用户的比例,这里的单位时间取天,实验时间区间为2015年9月7号到2015年9月30号。用户留存率研究对象为新注册用户,反映了推荐系统的转换能力,即由初期的不稳定的用户转化为活跃、稳定、忠诚的用户。
      \subsection{对比模型}
      基准模型为融合了用户画像的推荐模型,对照模型为单纯的推荐模型和推荐热门商品的简单推荐模型。每个推荐模型分流10\%的用户流量,推荐算法使用了开源软件spark MLlib的LogisticRegressionWithLBFGS模块,前俩个模型的推荐候选集为全部主题,简单推荐模型的推荐候选集为top20\%热度的主题。我们对比了单纯的推荐模型、推荐热门商品的简单推荐模型和融合了用户画像的推荐模型在2015年9月新注册用户数据集上的用户留存率。\autoref{pic:hl_saveRatio}展示了不同模型的实验结果。图中,横坐标是时间变量,单位为天,纵坐标是用户留存率,每一条曲线代表了一个模型的用户留存率随时间变化的曲线。通过观察曲线可以发现用户留存率随时间流动呈指数分布,头三天就流失了约90\%的新用户,从第四天用户留存率开始停留在一个比较稳定的阈值,实验结果显示,融合了用户画像的推荐模型相对其他模型有更高的留存率。截止到2015年9月30号,融合了用户画像的推荐模型的留存率是10.3\%,比推荐热门商品的简单推荐模型的留存率8.19\%高了2.11个百分点,相对于单纯的推荐模型的留存率5.76\%高了4.54个百分点。由此可见用户画像能够很好的解决冷启动问题并得到较高的新注册用户留存率。
      \begin{figure}
      \centering
        \framebox{\includegraphics[scale=0.55]{figures/hl_saveRatio}}
        \figcaption{新用户留存率实验对比图}
        \label{pic:hl_saveRatio}
      \end{figure}

    \section{本章小结}
      用户画像对于推荐系统来讲,主要几个方面的提升:提升推荐系统的精度,用户画像将用户的长期偏好融入到了推荐内容中,维护了推荐系统一致性。abtest显示,融合了用户画像的推荐模型比单纯的推荐模型在点击转化率指标提高了约2.8\%,考虑到300万用户的基数,2.8\%的提升是一个很大的进步;用户画像还解决新用户的冷启动问题,对于一个新注册用户来讲,推荐系统可以利用用户画像的静态信息,然后结合商品信息进行推荐;提高推荐系统的时效性,对用户行为的离线预处理,可以节约推荐系统的大部分计算时间。但是用户画像只是反映了用户长期的兴趣,所以无法动态的反映用户短期兴趣,因此我们引入了用户兴趣探索模块,将在下一章节件详细介绍。

  %自行添加
  
\chapter{用户兴趣探索}
\section{引言}
\label{chap:interestExplore}
电子商务产品的设计往往是数据驱动的,即许多产品方面的决策都是把用户行为数据量化后得出的。但就小米主题市场而言,只有那些热门主题才有足够多的数据可以挖掘,大部分主题只有很少的数据可供分析、挖掘,因此,需要一种方法对用户行为做针对性挖掘,利用少量数据获得出用户真正的偏好度,最终提升商品的销售量,这是问题之一。现有的推荐算法注重用户静态属性的同时却忽略了用户兴趣的动态变化,从而导致系统在时间维度上有偏离用户需求的趋势,这是问题之二。用户兴趣探索模块可以很好的解决此类问题。

探索用户兴趣的数据来源包括用户行为数据、用户画像和商品特征表。用户画像包括用户基本信息和兴趣标签等,商品特征表包括分类、属性标签等,过程分为几个步骤:首先,利用用户历史行为(评论,停留时长,评分,点赞,购买等)量化用户满意度,然后利用用户兴趣特征向量与商品特征矩阵得出相关分数,如果商品与用户的相关分数很低,但有很高的用户满意度,说明是一次成功的用户兴趣探索,更新用户画像。如果是热门商品,大量的用户都会点击,但商品与用户不是很相关,则认为其探索效果是有限的,反之如果是小众商品,考虑到长尾效应,则可以认为其是更成功的兴趣探索。这里涉及到的概念包括用户满意度的量化、用户和商品的关联度、商品属性标签的长尾性,下文会一一给出详细说明。

\section{用户行为数据的存储和处理}
手机主题用户行为数据的特点包括:1、用户基数庞大,手机主题注册用户达千万级,活跃用户达百万级。2、用户规模增长快,月新注册用户达10万数量级。3、每个用户的行为数量较小,即使是活跃用户,每天最多也只产生上百条行为记录。4、用户行为的计算较为复杂,计算用户的两次登录间隔天数、反复购买的商品、累积在线时间,这些都是针对用户行为的计算,通常具有一定的复杂性。5、用户行为数据格式不规整,字段丢失率较高。

  \subsection{数据预处理}
  数据预处理是数据挖掘过程中一个重要步骤,主要工作包括字段去重、无效日志过滤、多表字段的连接等。如统计2015年09月06号userId为001的投诉数,数据预处理过程如\autoref{code:data-predoing}。
  \begin{lstlisting}[language=java,firstnumber=1,label={code:data-predoing}, caption={数据预处理脚本}]
    set hiveconf:ymdwithline=2015-09-06;
    set hiveconf:metric=complaint_order_num;
    set hiveconf:user_id=001;

    select '${hiveconf:metric}' as metric, count(a.order_id) as score
    from (
        //去重
        select distinct order_id
        from theme.dw_v_order_base
        //以时间范围date_sub('${hiveconf:ymdwithline}',5) and '${hiveconf:ymdwithline}'为条件过滤掉不符合条件的订单
        where concat_ws('-',year,month,day) between date_sub('${hiveconf:ymdwithline}',5) and '${hiveconf:ymdwithline}'
        //无效订单过滤
        and order_id!=null
        //以用户id为条件过滤掉其他订单
        and user_id=${hiveconf:user_id}
    ) a
    inner join (
        //order_id 字段去重
        select distinct order_id
        from theme.g_comment_complaint
        //type = 3表示用户投诉
        where concat_ws('-',year,month,day) = '${hiveconf:ymdwithline}' and type = 3
        //多表字段的连接,如果有一个表有投诉记录,就算一次投诉。
        union
        select distinct order_id
        from theme.dwd_kefu_phone_complaint
        where concat_ws('-',year,month,day) = '${hiveconf:ymdwithline}'
    ) b
    on a.order_id = b.order_id
    group by metric;
  \end{lstlisting}

\section{用户兴趣探索模型}
用户兴趣探索主要功能模块包括:1,兴趣标签探测,在分析用户行为数据时,如果某些主题标签和用户相关度很低,那么这些标签会作为标签探索候选集。2,长尾标签提取,基于小众标签集,按照某种规则筛选出目标标签。3,用户满意度量化,根据用户所有对某一个主题的行为数据得出这个用户对这个主题的满意度。4,标签权重的更新,不管是不是一次成功的兴趣标签探索,都要对用户画像标签的权重做更新,更新算法利用了线性衰减思想。本章首先介绍一些基本概念,包括实体域、用户行为和用户满意度等。然后详细介绍用户兴趣探索功能模块的实现。
  \subsection{基本概念概述}
  实体域。如果一个模型是基于分析用户行为得出用户兴趣时,实体就是这个行为针对的对象。不同实体通过标签关联起来,对于手机主题应用市场来说,实体域还包括所有的背景图片,铃声,闹铃等。

  用户行为。包括浏览,点击,下载,试用,购买,评论。本文所指的用户行为都是指用户在某手机主题上的行为。

  用户满意度。用户满意度是指根据用户作用在主题上的不同行为动作及其属性值,反推得到的用户偏好度。

  小众标签集。小众标签集是指用户偏好频率低的主题标签的集合,需要补充是,长尾标签和小众标签大多数是一致的,只是长尾标签针对商品而言,小众标签针对用户而言。

  \subsection{兴趣标签探测功能模块}
  首先候选标签是与用户相关度低的标签,如用户001每次都会浏览动漫、美少女主题,但是有一天却购买了一款汽车手机主题,通过计算发现汽车标签对于用户001是从未遇到过的标签,即相关度为0,于是汽车标签将会是潜在的探索标签。事实上用户兴趣探索过程可以在很短的时间内完成,基于 Hive + HDFS 平台的时长维度为天,而基于 Kafka + Spark 平台可以将时长维度降到小时级别。

  \subsection{长尾标签抽取功能模块}
  首先需要介绍标签集中度(tagFocus)和标签热度(tagPopular),标签集中度针对商品而言,标签热度针对用户而言。

  标签集中度。如果某一类主题包集合中包含某兴趣标签的个数为tagInThemeNum,而其它类包含的总数为tagInOtherNum,当tagInThemeNum大的时候,就说明其集中度高。实际上,如果一个标签在同一类主题集合中频繁出现,则说明该标签能够很好代表这类主题集合的特征,这样的标签应该给它们赋予较高的权重,并选来作为该类主题的特征向量以区别于其它类主题,标签集中度公式如\autoref{equ:focus},我们很容易发现,如果一个标签只出现若干主题包,我们通过它就容易定位搜索目标,因此其权重也应该大。反之如果一个词在大量主题包中出现,其权重取较小为好。
  \begin{equation}
    tagFocus=\frac{|tagInThemeNum|}{|tagInThemeNum+tagInOtherNum|}
    \label{equ:focus}
  \end{equation}

  标签热度。标签热度指的是某一个给定标签在用户画像中出现的频率。例如在300万用户总数中,十分之一的用户标签中有“火影”标签,那么其热度为0.1。标签热度不是越大越好,有些标签如“精品”,“气质”等标签占了总词频的80\%以上,而它对区分主题类型几乎没有用,我们称这种词叫“应删标签”。即应删除词的权重应该是零,也就是说在度量相关性是不应考虑它们的频率。

  长尾标签定义为集中度和热度之比大于一个给定阈值的标签,且本身为小众标签。代码如\autoref{code:longtail-tag}。
  \begin{lstlisting}[language=java,firstnumber=1,label={code:longtail-tag}, caption={长尾标签抽取算法}]
    public Set<String> getLongTailTags(String userId, String itemId) throws Exception {
        Set<String> out = new HashSet<>();
        //获取所有小众标签
        HashSet<String> longTailTags = getColdTags();
        //获取所有当前用户画像没有的标签
        Set<String> rawTags = tagExplore(userId, itemId);
        for (String tag : rawTags) {
            if (!longTailTags.contains(tag)) {
                continue;
            }

            //获取标签的集中度
            long tagFocusScore = getTagFocusScore(tag);
            //获取标签的热度
            long tagPopularScore = getTagPopularScore(tag);
            if (tagFocusScore / tagPopularScore <= threshold) {
                continue;
            } else {
                out.add(tag);
            }
        }

        return out;
    }
  \end{lstlisting}

  \subsection{用户满意度量化功能模块}
    \autoref{tab:userAction}列举的用户行为包含了部分关键的行为类型,通过将不同行为反映为用户喜好的不同并进行加权累积,得到用户对于物品的总体喜好。显式的用户反馈比隐式的权值大,但比较稀疏,毕竟进行显示反馈的用户是少数;而隐式用户行为数据是用户在使用应用过程中产生的,它可能存在大量的噪音和用户的误操作,通过数据挖掘算法滤掉可能的噪音,这样使分析更加精确。然后是归一化操作,因为不同行为的数据取值可能相差很大,比如,用户的浏览数据必然比购买数据大的多,如何将各个行为的数据统一在一个相同的取值范围中,从而使得加权求和得到的总体喜好更加精确,就需要进行归一化处理使得数据取值在 [0,10] 范围中。

    \begin{table}[htp]
    \centering
    \tabcaption{用户行为权重对应表}
    \label{tab:userAction}
    \begin{tabular}{ |c|c|p{4cm}|p{5cm}|c|} \hline
     用户行为 & 类型 & 特征 & 作用 & 权重\\ \hline
     评分 & 显式 & 整数量化的偏好,可能的取值是 [0,5] & 通过用户对物品的评分,可以精确的得到用户的满意度,但是噪声比较大,比如遇到好评返现活动 & 1\\ \hline
     分享 & 显式 & 布尔量化的偏好,取值是 0 或 1 & 通过用户对物品的投票,可以精确的得到用户的喜好度,同时可以推理得到被转发人的兴趣取向 & 2\\ \hline
     评论 & 显式 & 一段文字,需要进行文本分析,得到偏好 & 通过分析用户的评论,可以得到用户的情感:喜欢还是讨厌 & 1\\ \hline
     赞/踩 & 显示 & 布尔量化的偏好,取值是 0 或 1 & 带有很强的个人喜好度 & 3 \\ \hline
     购买、试用 & 显式 & 布尔量化的偏好,取值是 0 或 1 & 用户的购买是很明确的说明这个项目它感兴趣。& 3 \\ \hline
     点击流 & 隐式 & 包括滑屏频率,滑屏次数,屏停留时长,用户对物品感兴趣,需要进行分析,得到偏好 & 用户的点击一定程度上反映了用户的注意力,所以它也可以从一定程度上反映用户的喜好。& 1 \\ \hline
     停留时长 & 隐式 & 一组时间信息,噪音大,需 要进行去噪,分析,得到偏 好 & 用户的页面停留时间一定程度上反映了用户的注意力和喜好,但噪音偏大,不好利用。比如说用户在浏览一个主题的时候,丢下手机和同学出去踢球去了,页面停留时长可能会很长 & 1 \\ \hline
    \end{tabular}
    \end{table}

  \section{用户画像和用户兴趣探索的融合}
  随着时间的变化,用户的兴趣会发生转移,时间越久远,标签的权重应该相应的下降,距离当前时间越近的兴趣标签应该得到适当突出。出于这样的考虑,一般会在标签权重值上叠加一个时间衰减函数,通过调节时间窗口大小和更新周期,体现不同的时效性。
  我们可以把用户画像权重想象成一个自然冷却的过程:任一时刻,用户画像中的标签都有一个当前温度,温度最高的标签权重值最高;如果该用户对某主题发生了一些正向行为,如点赞,该文章包含的标签在用户画像中的温度就会上升,否则温度下降;随着时间流逝,所有标签的温度都逐渐冷却,通过时间窗口向前滑动实现。

  这样假设的意义在于我们可以照搬物理学的牛顿冷却定律(\href{http://www.evanmiller.org/rank-hotness-with-newtons-law-of-cooling.html}{Newton's Law of Cooling}),建立标签权重与时间之间的函数关系:本期分数 = 上期分数 - 冷却系数$*$间隔天数。其中,冷却系数决定了标签融合的更新率,如果想放慢更新率,冷却系数就取一个较小的值,否则就取一个较大的值。

  标签权重的线性衰减算法结合了手机主题用户长期兴趣和短期兴趣,根据时间因素权重自动进行衰减,能准确反映用户兴趣的变化趋势。该模型是指用户对兴趣标签的权重仅代表评价当时的兴趣度,随着时间的推移,用户对该标签的权重将规律性地自动衰减,当权重衰减到 0 时,标签将被淘汰。


\section{实验与分析}
  \subsection{数据集准备}
  实验中我们利用2015年9月到2015年10月的用户行为数据和所有关联的手机主题包。这个数据集包含了110739个用户在这段时间对主题包的标签行为,数据集中包含了8936个主题包。该数据集每行是一条记录,每条记录包含的信息有:用户ID,主题ID,行为类型,行为值,日期,每一条记录代表了某个用户在某个时间点对某个主题包进行了某种行为。保证数据集具有一定的稠密程度,我们去除了用户行为记录少于10条的所有用户,最终用户集包含10646个用户,2033600条用户行为记录。
  \subsection{评测指标}
  使用线上A/B测试方案,利用点击购买转化率来评测推荐系统应对马太效应的效果。根据统计我们知道20\%的热门商品在占了80\%的曝光机会的同时却只占50\%的销售量,因为虽然热门商品销量很好但其整体数量偏少,很难满足大多数消费者的需求。相反,占据80\%的小众商品虽然曝光率低,但凭借其庞大数量和多样性,能满足多数消费者的需求。因此如果适度对小众商品增加曝光机就会大幅提升商品的销售量,这里我们选用的实验指标为商品的点击购买转换率。
  \subsection{对比模型}
  无兴趣探索模块的推荐模型,在实验中作为基准模型。对照模型包括融合了兴趣探索模块的推荐模型和推荐热门商品的简单推荐模型。
  \subsection{实验结果}
  我们对比了无兴趣探索模块的推荐模型、推荐热门商品的简单推荐模型和融合了兴趣探索模块的推荐模型在实验期间的有过至少一次销售记录的商品数itemCount。\autoref{pic:hl_themeNumber}展示了不同模型的实验结果。图中,横坐标是时间变量,单位为天,纵坐标是itemCount,每一条曲线代表了一个模型的itemCount随时间变化的曲线。通过观察曲线可知,融合了兴趣探索模块的推荐模型的itemCount月平均数是3136,推荐热门商品的简单推荐模型的itemCount月平均数是1935,无兴趣探索模块的推荐模型的itemCount月平均数是2679。实验说明融合了用户兴趣探索的推荐模型相对其他模型有更好的多样性。
  \begin{figure}
  \centering
    \framebox{\includegraphics[scale=0.55]{figures/hl_themeNumber}}
    \figcaption{推荐多样性实验对比图}
    \label{pic:hl_themeNumber}
  \end{figure}

  我们对比了无兴趣探索模块的推荐模型、推荐热门商品的简单推荐模型和融合了兴趣探索模块的推荐模型在实验期间的点击购买转化率。\autoref{pic:hl_buyLookRatio}展示了不同模型的实验结果。图中,横坐标是时间变量,单位为天,纵坐标是点击购买转化率,每一条曲线代表了一个模型的点击购买转化率随时间变化的曲线。实验结果显示,融合了兴趣探索模块的推荐模型相对其他模型有更高的点击购买转化率。融合了兴趣探索模块的推荐模型的平均点击购买转化率是32.74\%,比推荐热门商品的简单推荐模型的平均点击购买转化率9.63\%要高,相对于无兴趣探索模块的推荐模型的平均点击购买转化率17.54\%也高了不少。由此可见用户兴趣探索能够很好的提升点击购买转化率。
  \begin{figure}
  \centering
    \framebox{\includegraphics[scale=0.55]{figures/hl_buyLookRatio}}
    \figcaption{转化率实验对比图}
    \label{pic:hl_buyLookRatio}
  \end{figure}

\section{本章小结}
本章首先介绍了用户行为数据特点以及基于此的用户行为数据的的预处理。然后介绍了用户兴趣探索模块的组成内容,包括兴趣标签探测功能模块、长尾标签抽取功能模块和用户满意度量化功能模块,之后介绍了用户画像和用户兴趣探索的融合,最后给出了用户兴趣探索实验结果,即用户兴趣探索模块可以为推荐系统带来更好的多样性和更高的购买转化率。
  
\chapter{结束语}
  如果说过去的五年是推荐系统大行其道的时间,那么个性化将成为推荐技术未来五年中最重要的革新之一。一个好的推荐系统需要满足的目标有:能实时提供个性推荐服务,推荐结果满足新颖性、惊喜性和长尾性,而且推荐结果必须足够及时,这样才能在用户浏览之后、购买之前就获得推荐服务,笔者通过引入用户画像模块和用户兴趣探索模块来达到以上目标。

  本文介绍的推荐系统由三部分模块组成:用户画像模块,用户兴趣探索模块、推荐算法模块。用户画像模块记录了用户长期的信息,刻画用户的基础类型。用户兴趣探索模块负责记录能够体现用户喜好的行为,比如购买、下载、评分等,这部分看起来简单,其实需要非常仔细的设计,比如说购买和评分这两种行为表达潜在的喜好程度就不尽相同。完善的行为记录需要能够综合多种不同的用户行为,处理不同行为的累加。推荐算法模块的功能则实现了对用户行为记录的分析、计算和排序。

  推荐系统的主要方法是通过分析用户的过去预测未来,因此基于用户画像模型的研究是一个很好的突破口,因为用户画像天然支持用户历史信息的存储、查询,同时对用户兴趣探索的研究,无论是从促进用户画像模型的角度出发,还是从实际需求来看,都具有重要的意义,本文的研究工作正是在这一背景下展开。

  \section{研究工作总结}
    本文对推荐系统特别是与用户画像相关的动态推荐系统的相关工作做了总结和回顾之外,主要的工作包括以下几个方面:
    \begin{itemize}
      \item 设计了用户画像模型:通过对基础静态数据、基础行为数据类型和高维数据类型进行建模,得出较为完整、准确的兴趣标签,解决新用户的冷启动问题,提升了推荐系统的精度。
      \item 设计了用户兴趣探索模型:通过量化用户满意度和计算用户和商品的相关度,实现了用户小众兴趣的探索功能,提升了推荐系统的动态推荐效果。
      \item 利用线性衰减算法成功融合用户长期兴趣和短期兴趣:本文在研究用户画像建模和用户兴趣探索的基础上,结合电子商务用户兴趣偏好变化频繁的特点,提出了基于线性衰减的用户兴趣融合模型。标签权重的线性衰减算法结合了手机主题用户长期兴趣和短期兴趣,能准确反映用户兴趣的变化趋势。
    \end{itemize}

  \section{对未来工作的展望}
  本文对推荐系统的用户画像和用户兴趣探索模型进行了较深入的研究,但是针对用户兴趣变化的推荐模型的实现还有很多工作要做。本人认为有待解决的问题有:
    \begin{itemize}
      \item 用户行为的离线和在线计算的分配:用户行为每天产生的数据量很大,哪些行为需要在线实时计算反馈,哪些行为只需要离线计算即可,需要根据具体业务的特点和用户习惯赋予每种行为一个权重,然后根据权重排名决定计算方式。因此,用户行为的特征提取、分析将是我们将来工作的一个重要方面。
      \item 用户兴趣探索模型对推荐系统的影响:本文的所有工作的评估集中在点击购买转换率上。但点击购买转换率并不是推荐系统追求的唯一指标,比如,预测用户可能会去看,从而给用户推荐热门商品,这并不是一个好的推荐。因为热门商品本身的转化率就很高,这里涉及到了推荐系统的长尾性,即用户希望推荐系统能够给他们新颖的推荐结果,而不是那些他们已经知道的物品。此外,推荐系统还有多样性等指标。如何利用时间信息,在不牺牲转换率的同时, 提高推荐的其他指标,是笔者将来工作研究的一个重要方面。
      \item 推荐系统随时间的进化:用户的行为和兴趣是随时间变化的,意味着推荐系统本身也是一个不断演化的系统。其各项指标,包括长尾度、多样性和点击率都是随着数据的变化而演化。如何让推荐系统能够通过利用实时变化的用户反馈,向更好的方面发展是推荐系统研究的一个重要方面。
    \end{itemize}

  最后,希望本文的研究工作能够对动态推荐系统的发展作出一定的贡献,并真诚的希望老师们提出宝贵的批评意见和建议。

%%%%%%%%%%%%%%%%%%%%%%%%%%%%%%
%% 附件部分
%%%%%%%%%%%%%%%%%%%%%%%%%%%%%%
\backmatter

  % 参考文献
  % 使用 BibTeX
  % 选择参考文献的排版格式。注意ustcbib这个格式不保证完全符合要求,请自行决定是否使用
  %\bibliographystyle{plain}%{GBT7714-2005NLang-UTF8}
  %\bibliography{bib/tex}
  %\nocite{*} % for every item
  % 不使用 BibTeX
  %\renewcommand{\baselinestretch}{0.5}
\begin{thebibliography}{10}
\bibitem{long-tail}
O. Celma.
\newblock{\em Music Recommendation and Discovery in the Long Tail}.
\newblock Springer. 2010.

\bibitem{content-based}
Marko Balabanovi ́c and Yoav Shoham.
\newblock {\em Fab: content-based, collaborative recommendation}.
\newblock Commun. ACM, 40:66–72, March 1997.

\bibitem{cold-start}
Andrew I. Schein, Alexandrin Popescul, Lyle H. Ungar, David M. Pennock.
\newblock {\em Methods and Metrics for Cold-Start Recommendations}.
\newblock New York City, New York: ACM. pp. 253–260. 2002.

\bibitem{user-interest}
C\TeX{} Sia, K.C., Zhu, S., Chi, Y., Hino, K., Tseng, B.L.
\newblock {\em Capturing User Interests by Both Exploitation and Exploration}.
\newblock Technical report, NEC Labs America. 2006.

\bibitem{info-retrieval}
Jansen, B. J. and Rieh, S.
\newblock {\em The Seventeen Theoretical Constructs of Information Searching and Information Retrieval}.
\newblock Journal of the American Society for Information Sciences and Technology. 61(8), 2010.

\bibitem{date-mining}
Han, Jiawei; Kamber, Micheline.
\newblock {\em Data mining: concepts and techniques}.
\newblock Morgan Kaufmann. p. 5. 2001.recmd-system

\bibitem{recmd-system}
Francesco Ricci and Lior Rokach and Bracha Shapira.
\newblock {\em Introduction to Recommender Systems Handbook}.
\newblock Springer, pp. 1-35. 2011.

\bibitem{matthew-effect}
Robert K. Merton.
\newblock {\em The Matthew Effect in Science}.
\newblock Science, 159(3810):56– 63, January 1968.

\bibitem{matthew-effect:2}
Junghoo Cho and Sourashis Roy.
\newblock {\em Impact of search engines on page popularity}.
\newblock In Proceedings of the 13th international conference on World Wide Web, WWW ’04, pages 20–29, New York, NY, USA, ACM. 2004.

\bibitem{matthew-effect:3}
Daniel M. Fleder and Kartik Hosanagar.
\newblock {\em Recommender systems and their impact on sales diversity}.
\newblock In Proceedings of the 8th ACM conference on Electronic commerce, EC ’07, pages 192–199, New York, NY, USA, ACM. 2007.

\bibitem{social-filter}
Henry Kautz, Bart Selman, and Mehul Shah.
\newblock {\em Referral web: combining social networks and collaborative filtering}.
\newblock Commun. ACM, 40:63–65, March 1997.

\bibitem{collab-filter}
Jonathan L. Herlocker, Joseph A. Konstan, Loren G. Terveen, and John T. Riedl.
\newblock {\em Evaluating collaborative filtering recommender systems}.
\newblock ACM Trans. Inf. Syst., 22:5–53, January 2004.

\bibitem{ab-test}
Kohavi, Ron, Longbotham, Roger.
\newblock {\em Online Controlled Experiments and A/B Tests}.
\newblock In Sammut, Claude; Webb, Geoff. 2015.

\bibitem{cognitive-science}
Elaine Rich.
\newblock {\em Readings in intelligent user interfaces}.
\newblock chapter User modeling via stereotypes, pages 329–342. 1998.

\bibitem{info-overload}
Anne-F. Rutkowski and Carol S. Saunders.
\newblock {\em Growing pains with information overload}.
\newblock Computer, 43:96–95, June 2010.

\bibitem{Forecast-principle}
J. Scott Armstrong, editor.
\newblock {\em Principles of Forecasting - A Handbook for Researchers and Practitioners}.
\newblock Kluwer Academic, 2001.

\bibitem{cf-sn}
Henry Kautz, Bart Selman, and Mehul Shah.
\newblock {\em Referral web: combining social networks and collaborative filtering}.
\newblock Commun. ACM, 40:63–65, March 1997.

\bibitem{info-overload}
Greg Linden, Brent Smith, and Jeremy York. 
\newblock {\em Amazon.com recommendation- s: Item-to-item collaborative filtering}.
\newblock IEEE Internet Computing, 7:76–80, January 2003.

\bibitem{Amazon-cf}
Anne-F. Rutkowski and Carol S. Saunders.
\newblock {\em Growing pains with information overload}.
\newblock Computer, 43:96–95, June 2010.

\bibitem{temporal-cf}
Yehuda Koren.
\newblock {\em Collaborative filtering with temporal dynamics}.
\newblock In Proceedings of the 15th ACM SIGKDD international conference on Knowledge discovery and data mining, KDD ’09, pages 447–456, New York, NY, USA, 2009. ACM.

\bibitem{latent-cf}
Thomas Hofmann and Jan Puzicha.
\newblock {\em Latent class models for collaborative filtering}.
\newblock In Proceedings of the Sixteenth International Joint Conference on Artificial Intelligence, IJCAI ’99, pages 688–693, San Francisco, CA, USA, 1999. Morgan Kaufmann Publishers Inc.

\bibitem{demo-data}
Bruce Krulwich.
\newblock {\em Lifestyle finder: Intelligent user profiling using large-scale demographic data}.
\newblock AI Magazine, 18(2):37–45, 1997.

\bibitem{Trust-walker}
Mohsen Jamali and Martin Ester.
\newblock {\em Trustwalker: a random walk model for combining trust-based and item-based recommendation}.
\newblock In Proceedings of the 15th ACM SIGKDD international conference on Knowledge discovery and data mining, KDD ’09, pages 397–406, New York, NY, USA, ACM. 2009.

\end{thebibliography}


  % 附录,没有请注释掉
  %\begin{appendix}
  %  \include{chapter/chap-req}
  %\end{appendix}

  \makeatletter
  \ifustc@bachelor\relax\else
    % 致谢
	
\begin{thanks}

人生就是一个关于成长的漫长故事。而在中科大求学作为本人人生体验的一部分,亦是这样的一段故事。在此的俩年半,俯仰之间,科大的“问道”、“学术”于此,让我经历了这样的三段成长:学于师友,安于爱好,观于内心。

“古之学者必有师,师者,所以传道、授业、解惑也”。师友的教诲不可能一直跟着自己,可是他们治学态度却融入了我的人生观。授课的华保健老师的严谨、郭燕老师的认真、丁菁老师的直率、席菁老师的踏实都曾触动我,并给予我前进方向上的指引。

本论文内容为数据挖掘在电商行业的工程实现,因此有一段真实的、贴近数据挖掘领域的实习经历尤为重要。感谢我在苏州国云数据公司实习的 CEO 马晓东学长,让我有机会一窥大数据行业的内幕;感谢我在小米实习的导师方流博士,感谢我在滴滴出行工作的机器学习研究院李佩博士和袁森博士,让我成为大数据挖掘工程师的梦想又更近了一步;感谢我的导师周武旸教授和张四海教授,指导我完成论文。
向师友和书籍学习,是从外界汲取;只有回归到自己的内心和思绪才能沉淀.在每个夜幕深沉或是晨曦初露的时刻里,感受自己情绪的流动,反思自己的取舍得失,然后才有了融于师友和书籍时的奋进。这样的三段成长,如今已是一体,不断地相互印证与反馈!

“逝者如斯夫,不舍昼夜”。成长亦复如是,不断的和昨日的自己告别。但是,一路有你,真好!相会是缘,同行是乐,共事是福!

\vskip 18pt

\begin{flushright}

~~~~\ustc@author~~~~

\today
%\ustc@submitdate

\end{flushright}

\end{thanks}
%硕博致谢部分
    % 发表文章目录
    %\include{chapter/pub}
  \fi
  \makeatother

\end{document}

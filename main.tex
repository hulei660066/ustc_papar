\documentclass[master,oneside,euler,openright,macfonts]{ustcthesis}
% 默认twoside 双面打印
% 将master修改为bachelor, doctor or master
% 要使用adobe字体,添加adobefonts选项
% 要使用Mac系统的字体,添加macfonts选项
% 使用euler数学字体,如不愿使用,去掉euler
% 使用外文写作,请添加notchinese

% 设置图形文件的搜索路径
\graphicspath{{figures/}}

%仅用于本示例文档中显示特殊字符串
\usepackage{xltxtra}

%%%%%%%%%%%%%%%%%%%%%%%%%%%%%%
%% 封面部分
%%%%%%%%%%%%%%%%%%%%%%%%%%%%%%

  % 中文封面内容
  \title{基于用户画像的手机主题推荐系统}%一般情况下扉页和封皮、书脊共用一个标题文本,可以不用定义\spinetitle(仅硕博有用), \covertitle(本硕博均有用)和\encovertitle(仅本科有用)。特殊情况见下。
  %\spinetitle{\small{中国科学技术大学本硕博毕业论文模板示例文档\raisebox{-3pt}{(Beta)}}}
  %特殊情况1:本例中\title命令里含有换行控制字符,这会导致制作书脊的时候出现错误,例如如果你注释掉\spinetitle{...}这一行就会报错。这时需要定义一个不含换行等命令的\spinetitle,这并不表示\spinetitle里不能有任何命令——只能使用有限的命令。
  %特殊情况2:本例中标题过长,所以需要缩小书脊标题的字号。
  %特殊情况3:本例中中英文混排,由于tex竖排的原理限制,中英文基线不重合,所以需要人工调整英文的基线。具体调整量根据不同字体有所不同。
  %\covertitle{中国科学技术大学本硕博毕业\\论文模板示例文档(Beta)}
  %\covertitle{中文题目第一行\\中文题目第二行}
  %不要在此调整封皮字体大小! Do not set Cover Page font size here!
  %特殊情况4:本例中\title中含有多个换行,导致标题超过了两行。根据制本厂规定,封皮标题不能超过两行。因此需要定义封皮使用的标题\covertitle. 如果你注释掉这一行,就会发现封皮不符合规定。
 % \encovertitle{USTC Thesis Template for Bachelor, Master and Doctor User's Guide(Beta)}
  %\encovertitle{English Title Line 1\\English Title Line 2\\English Title Line 3}
  %不要在此调整封皮字体大小! Do not set Cover Page font size here!
  %特殊情况5:仅本科生有用。本科封皮中有英文标题,不超过三行。与上类似。

  \author{胡磊}
  \depart{软件学院}%系别,硕博请用系代号,本科请用全称如
  %\depart{数理化和信息工程系}
  \major{信息安全专业}%专业,硕博请用全称,本科不需要
  \advisor{周武旸\ 教授}
  \coadvisor{张四海\ 博士}%第二导师,没有请注释掉
  \studentid{SA13226110}%For bachelor only
  \submitdate{二〇一五年十二月}

  % 英文封面内容
  \entitle{The Phone Theme Recommendation System Based on User Profile}
  \enauthor{Lei Hu}
  \enmajor{Information Security}
  \enadvisor{Prof. Wuyang Zhou}
  \encoadvisor{Dr. Sihai Zhang}%另外一个导师
  \ensubmitdate{December 12th, 2015}
  
%%%%%%%%%%%%%%%%%%%%%%%%%%%%%%%%%%%%%%%%%%%%%%%%%%%%%%%%%%%%%%%%%%%%%

\begin{document}

  % 封面
  \maketitle

%特别注意,以下述顺序为准,在对应部分添加文档部件,切勿颠倒顺序:
%本科论文的文档部件顺序是:
%    frontmatter:致谢、目录、中文摘要、英文摘要、
%    mainmatter: 正文章节
%    backmatter: 参考文献或资料注释、附录
%硕博论文的文档部件顺序是:
%    frontmatter:中文摘要、英文摘要、目录、符号说明
%    mainmatter: 正文章节
%    backmatter: 参考文献、附录、致谢、发表论文
%%%%%%%%%%%%%%%%%%%%%%%%%%%%%%
%% 前言部分
%%%%%%%%%%%%%%%%%%%%%%%%%%%%%%
\frontmatter
\makeatletter
\ifustc@bachelor
	%%%%%%%%%%%%%%%%%
	%本科论文修改这里
	%%%%%%%%%%%%%%%%%
	% 致谢
	
\begin{thanks}

人生就是一个关于成长的漫长故事。而在中科大求学作为本人人生体验的一部分,亦是这样的一段故事。在此的俩年半,俯仰之间,科大的“问道”、“学术”于此,让我经历了这样的三段成长:学于师友,安于爱好,观于内心。

“古之学者必有师,师者,所以传道、授业、解惑也”。师友的教诲不可能一直跟着自己,可是他们治学态度却融入了我的人生观。授课的华保健老师的严谨、郭燕老师的认真、丁菁老师的直率、席菁老师的踏实都曾触动我,并给予我前进方向上的指引。

本论文内容为数据挖掘在电商行业的工程实现,因此有一段真实的、贴近数据挖掘领域的实习经历尤为重要。感谢我在苏州国云数据公司实习的 CEO 马晓东学长,让我有机会一窥大数据行业的内幕;感谢我在小米实习的导师方流博士,感谢我在滴滴出行工作的机器学习研究院李佩博士和袁森博士,让我成为大数据挖掘工程师的梦想又更近了一步;感谢我的导师周武旸教授和张四海教授,指导我完成论文。
向师友和书籍学习,是从外界汲取;只有回归到自己的内心和思绪才能沉淀.在每个夜幕深沉或是晨曦初露的时刻里,感受自己情绪的流动,反思自己的取舍得失,然后才有了融于师友和书籍时的奋进。这样的三段成长,如今已是一体,不断地相互印证与反馈!

“逝者如斯夫,不舍昼夜”。成长亦复如是,不断的和昨日的自己告别。但是,一路有你,真好!相会是缘,同行是乐,共事是福!

\vskip 18pt

\begin{flushright}

~~~~\ustc@author~~~~

\today
%\ustc@submitdate

\end{flushright}

\end{thanks}

	
	%目录部分
	%目录
	\tableofcontents
	%默认表格、插图、算法索引名称分别为“表格索引”、“插图索引”和“算法索引”
	%如果需要自行修改lot,lof,loa的名称,请定义
	%\ustclotname{...}
	%\ustclofname{...}
	%\ustcloaname{...}

	% 表格索引
	\ustclot
	% 插图索引
	\ustclof
	%算法索引 
	%如果需要使用算法环境并列出算法索引,请加入补充宏包。
	\ustcloa
	
	% 摘要
	 %\begin{cnabstract}
得益于互联网日新月异的演化和普及以及社交网络的流行,商业互联网公司可以获取大量的用户行为信息。大多数公司可以很容易地使用开源软件(OOS)如 Spark 来挖掘用户行为数据,分析、挖掘如此庞大的数据是几年前无法想象的。同样,因为互联网数据的爆发性成长,使得互联网公司迫切需要解决显示特定信息给哪些用户的问题。一种解决方式是根据用户之前的点击、浏览过的历史行为数据,分析、挖掘出用户可能感兴趣的其他信息并推送给用户,互联网公司如小米科技的手机主题个性化推荐系统可以做到千人千面,即根据用户使用偏好给每一个用户展示不同推荐结果。推荐系统的商业价值在于试图通过优化商品和用户的匹配度来销售更多的商品。

根据本人在小米科技的数据挖掘实习经历,基于互联网的商业推荐系统最大的挑战是:如何根据企业的商业目标和业务特点来优化推荐系统。对于小米科技手机主题业务,需要推荐系统能解决推荐社交化、长尾效应、冷启动等一系列综合问题,而这是一件具有挑战性的工作。

本论文的目的是研究一种融合了用户画像模块、用户兴趣探索模块的动态推荐系统。实验证明相比于传统推荐系统,融合了用户画像模块、用户兴趣探索模块的推荐系统具有多样性、长尾效应和冷启动周期短等优点。其中,用户画像建模使用了信息检索(IR)和数据挖掘(DM)技术,如常用的基于内容的协同推荐算法。基于推荐系统的数据驱动理念,本文也引入了如AB测试的说明。

\keywords{推荐系统\enskip 长尾效应\enskip 动态\enskip 用户画像建模\enskip 用户兴趣探索\enskip}
\end{cnabstract}

\begin{enabstract}
Because of the dynamic evolution and propagation of the Internet and the popularization of social networks, huge amounts of information about the users's behaviour are accessible to commercial internet companies. Most companies can be easily approaching OOS(open source software) such as spark to analyse and interpreting users's behaviour data-set,which just several years ago was not available or inaccessible. Also because of the evolution and popularization of the Internet and the vast amount of data stored, it has become desirable to moderate and select the content that is being displayed to the user,Mechanisms presenting goods on application try to present content that can interest the user based on his previous queries or browsing history. commercial internet companies such as e.g. Xiaomi filter the phone theme application visiting information and show only phone themes that the user potentially be interested in, trying to sell additional goods by recommender products based on user previous purchase behaviours.

Based on my working experience on Xiaomi as internship, The biggest challenge of recommender systems in commercial internet companies is: How to optimize a recommender system in accordance with the true business objective. for commercial internet company like Xiaomi the fitness recommender system is something that mix the social part, the long-tail, the cold-start and many other factors to finally aproximate what the user really wants. It's really a fascinating and complex working. 

The aim of this paper is analyse the long tail feature and dynamic feature of a android phone theme recommender system with the help of user profiles and user interests exploration. In order to build the user profile, Information Retrieval (IR) and Data Mining (DM) techniques such as content-based Collaborative Recommendation have been used. Because of the subjective nature of such a solution, verification process such as A/B test is also introduced.

\enkeywords{recommender systems, long tail, dynamic, user profile, user interests exploration}
\end{enabstract}
%此文件中含有中英文摘要
\else
	%%%%%%%%%%%%%%%%%
	%硕博论文修改这里
	%%%%%%%%%%%%%%%%%
	% 摘要
	 \begin{cnabstract}
得益于互联网日新月异的演化和普及以及社交网络的流行,商业互联网公司可以获取大量的用户行为信息。大多数公司可以很容易地使用开源软件(OOS)如 Spark 来挖掘用户行为数据,分析、挖掘如此庞大的数据是几年前无法想象的。同样,因为互联网数据的爆发性成长,使得互联网公司迫切需要解决显示特定信息给哪些用户的问题。一种解决方式是根据用户之前的点击、浏览过的历史行为数据,分析、挖掘出用户可能感兴趣的其他信息并推送给用户,互联网公司如小米科技的手机主题个性化推荐系统可以做到千人千面,即根据用户使用偏好给每一个用户展示不同推荐结果。推荐系统的商业价值在于试图通过优化商品和用户的匹配度来销售更多的商品。

根据本人在小米科技的数据挖掘实习经历,基于互联网的商业推荐系统最大的挑战是:如何根据企业的商业目标和业务特点来优化推荐系统。对于小米科技手机主题业务,需要推荐系统能解决推荐社交化、长尾效应、冷启动等一系列综合问题,而这是一件具有挑战性的工作。

本论文的目的是研究一种融合了用户画像模块、用户兴趣探索模块的动态推荐系统。实验证明相比于传统推荐系统,融合了用户画像模块、用户兴趣探索模块的推荐系统具有多样性、长尾效应和冷启动周期短等优点。其中,用户画像建模使用了信息检索(IR)和数据挖掘(DM)技术,如常用的基于内容的协同推荐算法。基于推荐系统的数据驱动理念,本文也引入了如AB测试的说明。

\keywords{推荐系统\enskip 长尾效应\enskip 动态\enskip 用户画像建模\enskip 用户兴趣探索\enskip}
\end{cnabstract}

\begin{enabstract}
Because of the dynamic evolution and propagation of the Internet and the popularization of social networks, huge amounts of information about the users's behaviour are accessible to commercial internet companies. Most companies can be easily approaching OOS(open source software) such as spark to analyse and interpreting users's behaviour data-set,which just several years ago was not available or inaccessible. Also because of the evolution and popularization of the Internet and the vast amount of data stored, it has become desirable to moderate and select the content that is being displayed to the user,Mechanisms presenting goods on application try to present content that can interest the user based on his previous queries or browsing history. commercial internet companies such as e.g. Xiaomi filter the phone theme application visiting information and show only phone themes that the user potentially be interested in, trying to sell additional goods by recommender products based on user previous purchase behaviours.

Based on my working experience on Xiaomi as internship, The biggest challenge of recommender systems in commercial internet companies is: How to optimize a recommender system in accordance with the true business objective. for commercial internet company like Xiaomi the fitness recommender system is something that mix the social part, the long-tail, the cold-start and many other factors to finally aproximate what the user really wants. It's really a fascinating and complex working. 

The aim of this paper is analyse the long tail feature and dynamic feature of a android phone theme recommender system with the help of user profiles and user interests exploration. In order to build the user profile, Information Retrieval (IR) and Data Mining (DM) techniques such as content-based Collaborative Recommendation have been used. Because of the subjective nature of such a solution, verification process such as A/B test is also introduced.

\enkeywords{recommender systems, long tail, dynamic, user profile, user interests exploration}
\end{enabstract}
%此文件中含有中英文摘要
	% 目录
	\tableofcontents
	%默认表格、插图、算法索引名称分别为“表格索引”、“插图索引”和“算法索引”
	%如果需要自行修改lot,lof,loa的名称,请定义
	%\ustclotname{...}
	%\ustclofname{...}
	%\ustcloaname{...}

	% 表格索引
	\ustclot
	% 插图索引
	\ustclof
	%算法索引 
	%如果需要使用算法环境并列出算法索引,请加入补充宏包。
	\ustcloa
	
	%符号说明,需要加入补充包
	 %\begin{denotation}

\item[] 
\end{denotation}
%不是必需的,如果不想列出请注释掉
\fi
\makeatother

%%%%%%%%%%%%%%%%%%%%%%%%%%%%%%
%% 正文部分
%%%%%%%%%%%%%%%%%%%%%%%%%%%%%%
\mainmatter

   
\chapter{绪论}
\label{chap:introduction}
\section{研究背景与意义}
	自互联网诞生以来,用户寻找信息的方法经历了几个阶段。早期的用户主要靠直接记住感兴趣网站的网址来寻找内容,直接促使Yahoo!提出了分类目录系统,将网站分门别类方便用户查询。但随着信息越来越多,分类目录也只能记录少量的网站,于是产生了搜索引擎。以Google为代表的搜索引擎可以让用户通过关键词找到自己需要的信息,但是,搜索引擎需要用户主动的提供显式关键词来寻找信息,因此它不能解决用户的更多的潜在需求,当用户无法精准描述自己的需求时,搜索引擎就无能为力了,于是又催生出推荐系统\citep{recmd-system}。以亚马逊电商官网为代表的推荐系统是一种帮助用户快速发现有用信息的工具,和搜索引擎不同的是推荐系统不需要提供明确的需求,而是通过分析用户的历史行为来给用户画像建模\citep{demo-data}从而主动给用户推荐出能够满足他们兴趣和需求的信息。因此,从某种意义上说推荐系统和搜索引擎是两个互补的工具。搜索引擎满足用户显式的需求,而推荐系统能够在用户没有明确目的的时候帮助他们发现潜在的需要。随着物联网和用户终端设备的发展,人们逐渐从信息的匮乏时代走进了信息的过载时代。无论是作为信息消费者的普通用户,还是作为信息生产者的提供商面临着数据爆炸时代的挑战。作为用户,如何从充斥着大量噪声的大数据中找到自己感兴趣的信息是一件非常耗时费力的事情,笔者曾有过这样的一种购物体验:在淘宝商城购买一台笔记本电脑,花费了一上午的时间才浏览、比较完所有的 thinkpad 品牌商家店面,如\autoref{fig:hl_taobao}。
	\begin{figure}
		\centering
		\includegraphics[width=0.9\textwidth]{hl_taobao}
		\figcaption{淘宝购物搜索图}
		\label{fig:hl_taobao}
	\end{figure}

	而近年来淘宝的交易额增长规模巨大,2005年淘宝交易额为80亿,2010年为4000亿,而到2015年淘宝双十一单日交易额就为912亿元,可见未来几年内笔者的这种关键字搜索+逐条浏览的购物方式已经不再具有可行性。而作为提供商,如何让自己生产的信息不埋没在大数据洪流中而受到潜在用户的充分关注,这也是其所要解决的一个课题,很多企业已经或者正在开发适合本公司的推荐系统(Recommender System)来解决这一矛盾。

	推荐系统广泛应用于电子商务领域,通过分析用户的数据,帮助用户找到喜欢和感兴趣的商品,然后推荐给他们。推荐系统的最大优点在于它能收集用户的兴趣信息并根据用户的不同偏好,主动的为用户做出个性化推荐,而且此推荐信息是动态更新的,也就是说随着时间的推移,用户的兴趣在逐渐改变,推荐系统的推荐结果也会随之改变。因此,推荐系统大大的提高了网站的用户体验,方便了用户对资源信息的查询。推荐系统的主要任务就是联系用户和信息,一方面协助用户发现自己潜在感兴趣的信息从而提升用户的满意度,另一方面让信息针对性的展现在只对它有兴趣的用户面前从而提升商品的转化率,于是实现了消费者和生产者的双赢。

	\subsection{推荐系统的定义}
	推荐系统的研究和很多早期的研究相关,比如认知科学(cognitive science)\citep{cognitive-science},信息检索(information retrieval)和预测理论\citep{Forecast-principle}。随着互联网的兴起,研究人员开始研究如何利用用户对物品行为数据来预测用户的兴趣并给用户做推荐\citep{cf-sn}。推荐系统开始成为一个比较独立的研究问题。到2006年为止推荐系统的研究主要集中在基于邻域的协同过滤算法,目前工业界应用最广泛、最知名的算法应该就是亚马逊开发并使用的协同过滤算法\citep{Amazon-cf}。推荐系统推荐给用户的商品首先不能与用户购买过的商品重复,其次也不能与用户刚浏览过的商品太相关。推荐系统的形式化定义如:设C是所有用户的集合,S是所有可以推荐给用户的主题的集合。实际上,C和S集合的规模通常很大,如上百万的顾客以及上万款手机主题。设函数u()可以计算主题s对用户c的推荐度R,即$u=C\times S \rightarrow R$,R是一定范围内的全序的非负实数,推荐要研究的问题就是找到推荐度R最大的那些主题S*,如\autoref{equ:fromal}。
	\begin{equation}
	\forall c \in C,S^{*}=arg  max_{s \in S} u(c,s)
	\label{equ:fromal}
	\end{equation}

	\subsection{推荐系统的产生与发展}
	随着科学技术与信息传播的迅猛发展,人类社会进入了一个全新的大数据时代,互联网和物联网无处不在的影响着人类生活的方方面面,并颠覆性改变了人们的生活方式,互联网用户既代表了网络信息的消费者,也代表了网络内容的生产者。尤其是随着Web 2.0时代的到来,社交化网络媒体的异军突起,互联网中的信息量呈指数级增长,而由于用户的辨别能力有限,使得其在庞大且复杂的互联网信息中找寻有用信息的成本巨大,这就是所谓的信息过载问题\citep{info-overload, info-overload:1}。搜索引擎和推荐系统的出现为用户解决信息过载提供了非常重要的技术手段。搜索引擎是被动的,用户在搜索互联网中的信息时需要在搜索引擎中输入关键词,搜索引擎根据输入在系统后台进行信息匹配,将与用户查询相关的信息展示给用户。但是当用户无法精确描述自己需求时,搜索引擎就无能为力了。推荐系统是主动的,用户不需要提供明确的需求,而是通过分析用户的历史行为来对用户进行分析,从而主动给用户推荐可能满足他们兴趣和需求的信息。因此搜索引擎和推荐系统是两个互补的技术手段。

	推荐系统概念是1995年在美国人工智能协会\citep{recmd-history}上由CMU大学的教授Robert Armstrong首先提出并推出了推荐系统的原型系统——Web Watcher。随后推荐系统的研究工作开始慢慢壮大。第一个正式商用的推荐系统是1996年Yahoo网站推出的个性化入口MyYahoo。21新世纪推荐系统的研究与应用随着电子商务的快速发展而风起云涌,各大电子商务网站都开发、部署了推荐系统,Amazon公司称其网站中35\%的营业额来自于自身的推荐系统。2006年美国的DVD租赁公司Netflix\citep{recmd-netflix}在网上公开设立了一个推荐算法竞赛并公开了真实网站中的一部分数据,包含用户对电影的评分。Netflix竞赛有效地推动了学术界和产业界对推荐算法的兴趣,很多有效的算法在此阶段被提了出来。

	自从1992年施乐的科学家为了解决信息负载的问题,第一次提出协同过滤算法,个性化推荐已经经过了二十几年的发展。1998年,林登和他的同事申请了item-to-item协同过滤技术的专利,经过多年的实践,亚马逊宣称销售的推荐占比可以占到整个销售GMV(Gross Merchandise Volume,即年度成交总额)的30\%以上。随后Netflix举办的推荐算法优化竞赛,吸引了数万个团队参与角逐,期间有上百种的算法进行融合尝试,加快了推荐系统的发展,其中SVD(Sigular Value Decomposition,即奇异值分解,一种正交矩阵分解法)和Gavin Potter跨界的引入心理学的方法进行建模,在诸多算法中脱颖而出。其中,矩阵分解的核心是将一个非常稀疏的用户评分矩阵R分解为两个矩阵:User特性的矩阵P和Item特性的矩阵Q,用P和Q相乘的结果R'来拟合原来的评分矩阵R,使得矩阵R'在R的非零元素那些位置上的值尽量接近R中的元素,通过定义R和R'之间的距离,把矩阵分解转化成梯度下降等求解的局部最优解问题。与此同时,Pandora、LinkedIn、Hulu、Last.fm等一些网站在个性化推荐领域都展开了不同程度的尝试,使得推荐系统在垂直领域有了不少突破性进展,但是在全品类的电商、综合的广告营销上,进展还是缓慢,仍然有很多的工作需要探索。特别是在全品类的电商中,单个模型在母婴品类的效果还比较好,但在其他品类就可能很差,很多时候需要根据品类、推荐栏位、场景等不同,设计不同的模型。同时由于用户、SKU不停地增加,需要定期对数据进行重新分析,对模型进行更新,但是定期对模型进行更新,无法保证推荐的实时性,一段时间后,由于模型训练也要相当时间,可利用传统的批处理的Hadoop的方法是无法再缩短更新频率,最终推荐效果会因为实时性问题达到一个瓶颈。推荐算法主要有基于人口统计学的推荐、基于内容的推荐、基于协同过滤的推荐等,而协同过滤算法又有基于邻域的方法(又称基于记忆的方法)、隐语义模型、基于图的随机游走算法等。基于内容的推荐解决了商品的冷启动问题,但是解决不了用户的冷启动问题,并且存在过拟合问题(往往在训练集上有比较好的表现,但在实际预测中效果大打折扣),对领域知识要求也比较高,通用性和移植性比较差,换一个产品形态,往往需要重新构建一套,对于多媒体文件信息特征提取难度又比较大,往往只能通过人工标准信息。基于邻域的协同过滤算法,虽然也有冷启动问题和数据稀疏性等问题,但是没有领域知识要求,算法通用性好,增加推荐的新颖性,并且对行为丰富的商品,推荐准确度较高。基于模型的协同过滤算法在一定程度上解决了基于邻域的推荐算法面临的一些问题,在RMSE(Root Mean Squared Error,即均方根误差)等推荐评价指标上更优,但是通常算法复杂,计算开销大,所以目前基于邻域的协同过滤算法仍然是最为流行的推荐算法。

	自推荐系统诞生后学术界对其关注的兴趣度也越来越大。从1999年开始美国计算机学会每年召开电子商务研讨会以来,发表的与推荐系统相关的论文数以千计。ACM信息检索专业组在2001年开始把推荐系统作为该会议的一个独立研究主题。同年召开的人工智能联合大会也将推荐系统作为一个单独的主题。目前为止数据库、数据挖掘、人工智能、机器学习方面的重要国际会议(如KDD、AAAI、ICML等)都有大量与推荐系统相关的研究成果发表。同时第一个以推荐系统命名的国际会议ACM Recommender Systems Conference 于2007年首次举办。在近几年的数据挖掘及知识发现国际会议举办的竞赛中,连续两年的竞赛主题都是推荐系统。2011年的KDD CUP 竞赛中,两个竞赛题目分别为音乐评分预测和识别音乐是否被用户评分(\href{http://www.kdd.org/kdd2011/kddcup.shtml}{www.kddcup2011.org})。2012年的KDD CUP 竞赛中,两个竞赛题目分别为腾讯微博中的好友推荐和计算广告中的点击率预测。(\href{www.kddcup2012.org}{www.kddcup2012.org})

	\subsection{推荐系统的作用}
	推荐系统改变了没有活力的网站与其用户通信的方式。无需提供一种静态体验,让用户搜索并可能购买产品,推荐系统加强了交互,以提供内容更丰富的体验。推荐系统根据用户过去的购买和搜索历史,以及其他用户的行为,自主地为各个用户识别推荐内容。个性化推荐的最大的优点在于它能收集用户特征资料并根据用户特征,如兴趣偏好,为用户主动作出个性化的推荐。而且,系统给出的推荐是可以实时更新的,即当系统中的商品库或用户特征库发生改变时,给出的推荐序列会自动改变。这就大大提高了电子商务活动的简便性和有效性,同时也提高了企业的服务水平。总体说来,一个成功的个性化推荐系统的作用主要表现在以下几个方面:
	\begin{enumerate}[(1)]
	\item 将电子商务网站的浏览者转变为购买者:电子商务系统的访问者在浏览过程中经常并没有购买欲望,个性化推荐系统能够向用户推荐他们感兴趣的商品,从而促成购买过程。
	\item 提高电子商务网站的交叉销售能力:个性化推荐系统在用户购买过程中向用户提供其他有价值的商品推荐,用户能够从系统提供的推荐列表中购买自己确实需要但在购买过程中没有想到的商品,从而有效提高电子商务系统的交叉销售。
	\item 提高客户对电子商务网站的忠诚度:与传统的商务模式相比,电子商务系统使得用户拥有越来越多的选择,用户更换商家极其方便,只需要点击一两次鼠标就可以在不同的电子商务系统之间跳转。个性化推荐系统分析用户的购买习惯,根据用户需求向用户提供有价值的商品推荐。如果推荐系统的推荐质量很高,那么用户会对该推荐系统产生依赖。因此,个性化推荐系统不仅能够为用户提供个性化的推荐服务,而且能与用户建立长期稳定的关系,从而有效保留客户,提高客户的忠诚度,防止客户流失。
	\end{enumerate}

	\subsection{推荐系统与电子商务}
	近几年随着电子商务蓬勃发展,推荐系统在互联网中的优势地位也越来越明显。在国外比较著名的电子商务网站有Amazon和eBay,其中Amazon平台中采用的推荐算法是非常成功的。在国内比较典型的电子商务平台网站有淘宝网、网页云音乐、爱奇艺PPS等。在这些电子商务平台中,网站提供的商品数量不计其数,网站中的用户规模也非常巨大。据不完全统计天猫商城中的商品数量已经超过了5000万。在商品数量如此庞大的电商网站中,如果用户仅仅根据自己的购买意图输入关键字查询只会得到很多用户很难区分的相似结果,也不便用户做出选择。因此推荐系统作为能够根据用户兴趣\citep{user-interest}为用户推荐商品的主要途径,从而为用户在购物的选择中提供建议的需求非常明显。目前比较成功的电子商务网站中,都不同程度地利用推荐系统在用户购物的同时为用户推荐一些商品,从而提高商品的销售额。另一方面,随着以智能手机为代表的物联网推动了移动互联网的发展。在用户在连入移动互联网的过程中,其所处的地理位置信息可以非常准确地被获取,并由此出现了大量的基于用户位置信息的网站。国外比较著名的有Uber和Coupons。国内著名的有滴滴出行和美团网。例如,在美团网这种基于位置服务的网站中,用户可以根据自己的当前位置搜索餐馆、酒店、影院、旅游景点等信息服务。同时,可以对当前位置下的各类信息进行点评,为自己在现实世界中的体验打分,分享自己的经验与感受。当用户使用这类基于位置的网站服务时,同样会遭遇信息过载问题。推荐系统可以根据用户的位置信息为用户推荐当前位置下用户感兴趣的内容,为用户提供符合其真正需要的内容,提升用户对网站的满意度。

	随着社交网络的深入人心,用户在互联网中的行为不再局限于获取信息,更多的是与网络上的其他用户进行互动。国外著名的社交网络有Facebook、Twitter等,国内的社交网络有微信、米聊等。在社交网站中用户不再是单个的个体,而是与网络中的很多人具有了错综复杂的社交关系链。社交网络中最重要的资源就是用户与用户之间的这种联系。社交网络中用户间的关系是多维度的,建立社交关系的因素可能是在现实世界中是亲人、同学、同事、朋友关系,也可能只是网络中的虚拟朋友,比如都是有着共同爱好的会员成员。在社交网络中用户与用户之间的联系紧密度反映了用户之间的信任关系,用户不在是一个个体存在,其在社交网络中的行为或多或少地会受到其他用户关系的影响。因此推荐系统在这类社交网站中的研究与应用应该考虑用户社交的影响。

	现如今推荐系统在很多领域得到了广泛的应用,如出租车推荐、商品推荐、美餐推荐、电影推荐和音乐推荐,几乎囊括了人类的吃住行穿四大领域,团购网站美团网早已经利用推荐系统提供面向不同业务的个性化服务:1,猜你喜欢:美团最重要的推荐产品,目标是让用户打开美团App的时候,可以最快找到用户想要的团购服务;2,首页频道推荐:若干频道是固定的,若干频道是根据用户的个人偏好推荐出来的;3,今日推荐个性化推送:美团的个性化推送的产品,目的是在用户打开美团App前,就把用户最感兴趣的服务推送给用户,促使用户点击及下单,从而提高用户的活跃度;4,品类列表的个性化排序:美团首页的那些品类频道区。

\section{大数据时代下的推荐系统}
	虽然推荐系统己经被成功运用在很多大型系统、网站,但是在当前大数据的时代下,推荐系统的面临的场景越来越复杂,推荐系统不仅需要解决传统的数据稀疏、冷启动和动态兴趣问题,还面临由大数据引发的更多、更复杂的实际问题,例如数以亿计的用户数目和海量用户同时访问推荐系统所造成的性能压力,使传统的基于单节点架构的推荐系统不再适用。同时Web服务器处理系统请求在大数据集下变得越来越多,Web服务器响应速度缓慢制约了当前推荐系统为大数据集提供推荐。基于实时模式的推荐在大数据集下也面临着严峻考验,用户难以忍受超过秒级的推荐结果返回时间。传统推荐系统的单一数据库存储技术在大数据集下变得不再适用,急需一种对外提供统一接口、对内采用多种混合模式存储的存储架构来满足大数据集下各种数据文件的存储。并且传统推荐系统在推荐算法上采取的是单机节点的计算方式也不能满足海量用户行为数据的计算需求。大数据本身具有的复杂性、不确定性也给推荐系统带来诸多新的挑战,传统推荐系统的时间效率、空间效率和推荐准确度都遇到严重的瓶颈。

	\subsection{推荐系统的关键技术}
	分布式文件系统。传统的推荐系统技术主要处理小文件存储和少量数据计算,大多是面向服务器的架构,中心服务器需要收集用户的浏览记录、购买记录、评分记录等大量的交互信息来为单个用户定制个性化推荐。当数据规模过大,数据无法全部载入服务器内存时,就算采用外存置换算法和多线程技术,依然会出现I/O上的性能瓶颈,致使任务执行效率过低,产生推荐结果的时间过长。对于面向海量用户和海量数据的推荐系统,基于集中式的中心服务器的推荐系统在时间和空间复杂性上无法满足大数据背景下推荐系统快速变化的需求。大数据推荐系统采用基于集群技术的分布式文件系统管理数据。建立一种高并发、可扩展、能处理海量数据的大数据推荐系统架构是非常关键的,它能为大数据集的处理提供强有力的支持。Hadoop 的分布式文件系统架构是其中的典型。与传统的文件系统不同,数据文件并非存储在本地单一节点上,而是通过网络存储在多台节点上。并且文件的位置索引管理一般都由一台或几台中心节点负责。客户端从集群中读写数据时,首先通过中心节点获取文件的位置,然后与集群中的节点通信,客户端通过网络从节点读取数据到本地或把数据从本地写入节点。在这个过程中由HDFS来管理数据冗余存储、大文件的切分、中间网络通信、数据出错恢复等,客户端根据HDFS 提供的接口进行调用即可,非常方便。
	
	分布式计算框架。集群上实现分布式计算的框架很多,Spark作为推荐算法并行化的依托平台,既是一种分布式的计算框架,也是一种新型的分布式计算编程模型,是一种常见的开源计算框架。其基于内存的MapReduce算法的核心思想是分而治之,把对大规模数据集的操作,分发给一个主节点管理下的各个分节点共同完成,然后通过整合各个节点的中间结果,得到最终结果。计算框架负责处理并行编程中分布式存储、工作调度、负载均衡、容错均衡、容错处理以及网络通信等复杂问题,把处理过程高度抽象为两个函数: map和reduce。map负责把任务分解成多个任务,reduce负责把分解后多任务处理的结果汇总起来。
	
	推荐算法并行化。大型企业所需的推荐算法要处理的数据量非常庞大,从TB级别到PB级甚至更高,腾讯Peacock主题模型分析系统需要进行高达十亿文档、百万词汇、百万主题的主题模型训练,仅一个百万词汇乘以百万主题的矩阵,其数据存储量已达3TB。面对如此庞大的数据,若采用传统串行推荐算法,时间开销太大。当数据量较小时,时间复杂度高的串行算法能有效运作,但数据量极速增加后,这些串行推荐算法的计算性能过低,无法应用于实际的推荐系统中。因此,面向大数据集的推荐系统从设计上就应考虑到算法的分布式并行化技术,使得推荐算法能够在海量的、分布式、异构数据环境下得以高效实现。

	\subsection{推荐系统算法简介}
	现有的推荐算法类型很多,但是各有各的局限,因此推荐系统经常采用组合推荐算法,即融合了协同过滤推荐、聚类算法和其他算法的组合推荐算法。
	\begin{enumerate}[(1)]
	\item 协同过滤算法。

	利用用户的历史喜好信息计算用户之间的距离,然后利用目标用户的最近邻居用户对评价的加权评价值来预测目标用户对特定手机主题的喜好程度,系统从而根据这一喜好程度来对目标用户进行推荐。协同过滤是基于这样的假设:为一用户找到他真正感兴趣的内容的好方法是首先找到与此用户有相似兴趣的其他用户,然后将他们感兴趣的内容推荐给此用户。协同过滤正是把这一思想运用到手机推荐系统中来,基于其他用户对某一类手机主题的评价来向目标用户进行推荐。基于协同过滤的推荐系统可以说是从用户的角度来进行相应推荐的,而且是自动的,即用户获得的推荐是系统从购买模式或浏览行为等隐式获得的,不需要用户努力地找到适合自己兴趣的推荐信息,如填写一些调查表格等。

	协同过滤的根本原理是,人们可以从和自己有相同品味、习性的人群那里获得高质量的推荐。协同过滤算法主要研究如何聚类具有相似兴趣特征的人群并基于此做出推荐,因为算法本身是基于用户社交群体,因此往往会涉及到大规模的用户行为数据的计算。协同过滤的应用领域也很广:电子商务,金融信贷,搜素引擎,互联网企业,网络社区等需要对用户提供个性化体验的服务商。因为中国现有的人口国情,协同过滤算法往往需要面对亿万级用户和海量的用户-主题交互数据。作为输入数据,一个用户是以一个N维度的向量来表示,N代表所有的主题数量。向量内容可以为正也可为负,分别表示了用户喜欢、讨厌该主题的程度。对于热门主题,给其打分的用户会很多,其分数应该乘以一个因子u得到有效的分数,u代表所有给其打分的用户个数的倒数,大多数用户向量是稀疏的。在协同过滤算法中关键性的一步就是要选择测量的距离,描述集合相似度算法有欧氏距离、闵可夫斯基距离、汉明距离等,其中最常用的有余弦距离公式(cosine similiarity),公式描述如,
	其中$similarity_{uv}$代表用户u与v之间的兴趣相似度,N(u)表示用户u曾经喜欢过的物品集合,N(v)表示用户v曾经喜欢过的物品集合。
	\begin{equation}
	similarity_{uv} = \frac{|N(u)\cdot N(v)|}{||N(u)||\cdot||N(v)||}
	\label{cosine-similiarity}
	\end{equation}
	
	然后利用相似度算法把用户分类成独立的集合,每个用户有且只属于其中的一个集合,对于每个集合,取这个集合最受欢迎的top N 个主题,作为推荐内容推荐给集合的所有用户。大多数情况下协同过滤算法面都临着一个问题:最坏情况下需要遍历所有的用户和所有的主题,算法计算复杂度为O(MN),M是用户数N是主题数,解决方法可以借助一种简单的降维思想加以解决:通过去掉那些非常冷门的主题对N做降维,通过去掉那些非常不活跃的用户对M做降维,计算维度下降的代价是降低了推荐系统的准确性。
	\item 聚类算法

	聚类分析是对于统计数据分析的一门技术,和分类算法一个主要的区别就是聚类不需要人工参与打标签,基于聚类的协同过滤方法,也可以在一定程度上解决传统协同过滤算法用户评分矩阵稀疏和冷启动问题,在降低用户评分矩阵稀疏性的同时提高目标用户最近邻居的查询速度。聚类是把相似的对象通过静态分类的方法分成不同的组别或者更多的子集,这样让在同一个子集中的成员对象都有相似的一些属性,聚类结果不仅可以揭示数据间的内在联系与区别,还可以为进一步的数据分析与知识发现提供重要依据。在结构性聚类中关键性的一步就是要选择测量的距离。一个简单的测量就是使用曼哈顿距离,它相当于每个变量的绝对差值之和。该名字的由来起源于在纽约市区测量街道之间的距离就是由人步行的步数来确定的。聚类模块可以是对用户兴趣属性相似度做聚类,也可以对用户社交属性相似度做聚类,或者俩种兼有。

	在现实社会中人们的兴趣和选择往往受到身边亲朋好友的影响。在互联网中随着诸如国内的腾讯,国外的Twitter等社会网络网站的兴起,如何利用用户的社会属性做推荐是近几年推荐领域比较热门的研究问题。基于社会网络的推荐算法被称为社会化推荐。近几年在工业界已经有了很多社会化推荐系统。最简单的社会化过滤算法是基于邻域的算法。给定用户u,令F(u)为用户u的好友集合,N(u)为用户u喜欢的物品集合。那么用户u对物品i的喜好程度定义为用户u的好友中喜欢物品i的好友个数,如公\autoref{Social-Rec}。
	\begin{equation}
		P_{vi} = \sum_{v\in F(v),i\in N(u)}^{} 1
		\label{Social-Rec}
	\end{equation}

	聚类算法在许多领域受到广泛应用,包括机器学习,数据挖掘,模式识别,图像分析以及生物信息,最常用的k-means算法\citep{recmd-kmeans}表示以空间中k个点为中心进行聚类,对最靠近他们的对象归类。
	\item 基于内容的推荐算法。

	基于内容的推荐是信息过滤技术的延续与发展,它是建立在对手机主题的标签信息上作出推荐的,而不需要依据用户对手机主题的评价意见,需要用机器学习的方法从关于内容的特征描述的事例中得到用户的兴趣资料。手机主题是通过相关的特征的属性来定义,系统基于用户评价对象的特征,学习用户的兴趣,考察用户资料与待预测手机主题的相匹配程度。用户的资料模型取决于所用学习方法,采用了综合决策树、神经网络和基于向量的组合方法。 基于内容的用户资料是需要有用户的历史数据,用户资料模型可能随着用户的偏好改变而发生变化。基于内容推荐方法的优点是:不需要其它用户的数据,没有冷开始问题和稀疏问题。能为具有特殊兴趣爱好的用户进行推荐。能推荐新的或不是很流行的手机主题,没有产品问题。通过列出推荐手机主题的内容特征,可以解释为什么推荐那些手机主题。

	本节利用spark mllib中ALS算法解释基于内容的推荐。首先,给出一个(用户,主题,评分)三元组的数据集,ALS会建立一个user$\ast$product的m$\ast$n的矩阵,其中,m为用户的数量,n为商品的数量。这个矩阵的每一行代表一个用户 ($u_1$,$u_2$,…,$u_9$)、每一列代表一个产品 ($v_1$,$v_2$,…,$v_9$)。用户的打分在0到10之间。但是在这个数据集中,并不是每个用户都对每个产品进行过评分,所以这个矩阵往往是稀疏的,所以需要做预处理将其填满,然后开始训练:
    假设m$\ast$n的评分矩阵R,可以被近似分解成$U*V^{T}$ ,U为m$\ast$d的用户特征向量矩阵,V为n$\ast$d的产品特征向量矩阵,d为用户和商品的特征值的数量。

    \begin{center} 
	$\begin{Bmatrix}
	  & u_1 & u_2\\ 
	p_1 & 8 & 7\\ 
	p_2 & 44 & 39
	\end{Bmatrix} =
	\begin{Bmatrix}
	  & f_1 & f_2\\ 
	p_1 & 0 & 1\\ 
	p_2 & 2 & 3
	\end{Bmatrix} \ast 
	\begin{Bmatrix}
	  & u_1 & u_2\\ 
	p_1 & 10 & 9\\ 
	p_2 & 8 & 7
	\end{Bmatrix}$
	\end{center}

	对于电影类型的手机主题,可以从d个角度进行评价,如主角,铃声,背景,特效4个角度来评价,那么d就等于4。矩阵V由n个product$\ast$d个特征值组成。对于矩阵U,假设对于任意的用户A,该用户对一款手机主题的综合评分和主题的特征值存在一定的线性关系,综合评分=(a1*d1+a2*d2+a3*d3+a4*d4) ,其中$a_{i}$为用户A的特征值,$d_{i}$为之前所说的主题的特征值。ALS算法认为m*n的评分矩阵R可以被近似分解成$U*V^{T}$,得到目标函数:
    \begin{equation}
    L(U,V)=\sum_{i,j}(R_{ij}-U_{i}^{T}V_{j})^{2}
    \label{F-Measure}
    \end{equation}

    其中a表示评分数据集中用户i对产品j的真实评分,另外一部分表示用户i的特征向量和产品j的特征向量,加上正则化参数$\lambda (||U_i||^2+||V_j||^2)$以防止过度拟合,固定V对U求导得到公式:
    \begin{equation}
    U_{t}=R_{t}V_{ut}(V_{ut}^TV_{ut}+\lambda n_{ut}I)^{-1}, i \in [1,m]
    \label{equa-least2}
    \end{equation}

    其中$R_{t}$表示用户i评过的手机主题的评分向量,$V_{ut}$表示用户i评过的手机主题的特征向量组成的特征矩阵。$n_{ut}$表示用户i评过的手机主题数量。同理,固定U,可以得到求解$V_{j}$的公式:
    \begin{equation}
    V_{j}=R_{j}^{T}U_{mj}(U_{mj}^{T}U_{mj}+\lambda n_{mj}I)^{-1}
    \label{equa-least3}
    \end{equation}

    $R_{j}$表示评过手机主题j的用户向量,$U_{mj}$表示评过手机主题j的用户特征向量组成的矩阵,$m_{mj}$表示评过电影j的用户的数量。

    首先用一个小于1的随机数初始化V,根据\autoref{equa-least2}求U,此时就可以得到初始的UV矩阵了,根据计算得到的U和\autoref{equa-least3}重新计算并覆盖V,反复进行以上两步的计算,直到目标函数和小于一个预设的值,或者迭代次数满足要求则停止。
	\item 组合推荐。

	由于各种推荐方法都有优缺点,手机主题推荐采用了组合推荐方式。研究和应用最多的是基于内容的推荐和协同过滤推荐的组合。最简单的做法就是分别用基于内容的方法和协同过滤推荐方法去产生一个推荐预测结果,然后用某方法组合其结果。组合推荐一个最重要原则就是通过组合后要能避免或弥补各自推荐技术的弱点。在组合方式上使用了几种组合思路:加权(Weight):加权多种推荐技术结果。变换(Switch):根据问题背景和实际情况或要求决定变换采用不同的推荐技术。混合(Mixed):同时采用多种推荐技术给出多种推荐结果为用户提供参考。特征组合(Feature combination):组合来自不同推荐数据源的特征被另一种推荐算法所采用。层叠(Cascade):先用一种推荐技术产生一种粗糙的推荐结果,第二种推荐技术在此推荐结果的基础上进一步作出更精确的推荐。特征扩充(Feature augmentation):一种技术产生附加的特征信息嵌入到另一种推荐技术的特征输入中。
	\end{enumerate}

	\subsection{推荐系统面临的问题}
	\begin{enumerate}[(1)]
	\item 特征提取问题。

	推荐系统的推荐对象种类丰富,例如新闻、博客等文本类对象,视频、图片、音乐等多媒体对象以及可以用文本描述的一些实体对象等。如何对这些推荐对象进行特征提取一直是学术界和工业界的热门研究课题。对于文本类对象,可以借助信息检索领域己经成熟的文本特征提取技术来提取特征。对于多媒体对象,由于需要结合多媒体内容分析领域的相关技术来提取特征,而多媒体内容分析技术目前在学术界和工业界还有待完善,因此多媒体对象的特征提取是推荐系统目前面临的一大难题。此外推荐对象特征的区分度对推荐系统的性能有非常重要的影响。目前还缺乏特别有效的提高特征区分度的方法。
	\item 数据稀疏问题。

	现有的大多数推荐算法都是基于用户—物品协同过滤矩阵数据,数据的稀疏性问题主要是指用户—物品评分矩阵的稀疏性,即用户与物品的交互行为太少。一个大型网站可能拥有上亿数量级的用户和物品,用户评分数据总量在面对增长更快的“用户—物品评价矩阵”时,仍然表现出稀疏性,推荐系统研究中的经典数据集MovieLens的稀疏度仅4.5\%, Netflix百万大赛中提供的音乐数据集的稀疏度是1.2\%。这些都是已经处理过的数据集,实际上真实数据集的稀疏度都远远低于1\%。例如, Bibsonomy的稀疏度是0.35\%,Delicious的稀疏度是0.046\%,淘宝网数据的稀疏度甚至仅在0.01\%左右。根据经验,数据集中用户行为数据越多,推荐算法的精准度越高,性能也越好。若数据集非常稀疏,只包含极少量的用户行为数据,推荐算法的准确度会大打折扣,极容易导致推荐算法的过拟合,影响算法的性能。
	\item 冷启动问题。

	冷启动问题是推荐系统所面临的最大问题之一。冷启动问题总的来说可以分为3类:系统冷启动问题、新用户问题和新物品问题。系统冷启动问题指的是由于数据过于稀疏,“用户—物品评分矩阵”的密度太低,导致推荐系统得到的推荐结果准确性极低。新物品问题是由于新的物品缺少用户对该物品的评分,这类物品很难通过推荐系统被推荐给用户,用户难以对这些物品评分,从而形成恶性循环,导致一些新物品始终无法有效推荐。新物品问题对不同的推荐系统影响程度不同:对于用户可以通过多种方式查找物品的网站,新物品问题并没有太大影响,如电影推荐系统等,因为用户可以有多种途径找到电影观看并评分;而对于一些推荐是主要获取物品途径的网站,新物品问题会对推荐系统造成严重影响。通常解决这个问题的途径是激励或者雇佣少量用户对每一个新物品进行评分。新用户问题是目前对现实推荐系统挑战最大的冷启动问题:当一个新的用户使用推荐系统时,他没有对任何项目进行评分,因此系统无法对其进行个性化推荐;即使当新用户开始对少量项目进行评分时,由于评分太少,系统依然无法给出精确的推荐,这甚至会导致用户因为推荐体验不佳而停止使用推荐系统。当前解决新用户问题主要是通过结合基于内容和基于用户特征的方法,掌握用户的统计特征和兴趣特征,在用户只有少量评分甚至没有评分时做出比较准确的推荐。

	\item 马太效应。

	马太效应(Mattnew Effect)是指强者愈强、弱者愈弱的现象,在互联网中引申为热门的产品受到更多的关注,冷门内容则愈发的会被遗忘的现象。很不幸的是推荐系统的出现加剧了互联网商品的马太效应,因为很多商品只有很少的评分,因此很难在推荐系统中应用,导致推荐结果大部分为热门商品。与马太效应相对于的是长尾理论,由美国人克里斯·安德森提出。长尾理论认为,由于成本和效率的因素,当商品储存流通展示的场地和渠道足够宽广,商品生产成本急剧下降以至于个人都可以进行生产,并且商品的销售成本急剧降低时,几乎任何以前看似需求极低的产品,只要有卖,都会有人买。这些需求和销量不高的产品所占据的共同市场份额,可以和主流产品的市场份额相比,甚至更大。
	\end{enumerate}

	\subsection{推荐系统开源项目介绍}
	工欲善其事,必先利器,关于大数据,有很多令人兴奋的事情,但如何分析、利用它也带来了很多困惑。好在开源观念盛行的今天,有一些在大数据领域领先的免费开源技术可供利用。
	\begin{itemize}
		\item Apache Hadoop:Hadoop是一个由Apache基金会所开发的分布式系统基础架构,是一种用于分布式存储和处理商用硬件上大型数据集的开源框架,可让各企业迅速从海量结构化和非结构化数据中获得洞察力。Hadoop的框架最核心的设计就是HDFS和MapReduce。HDFS为海量的数据提供了存储,则MapReduce为海量的数据提供了计算。HDFS有高容错性的特点,并且设计用来部署在低廉的硬件上;而且它提供高吞吐量来访问应用程序的数据,适合那些有着超大数据的应用程序。MapReduce 本身就是用于并行处理大数据集的软件框架,其根源是函数性编程中的 map 和 reduce 函数。它由两个可能包含有许多实例的操作组成。Map 函数接受一组数据并将其转换为一个键/值对列表,输入域中的每个元素对应一个键/值对。
		\item Apache Hive:Hive是建立在 Hadoop 上的数据仓库基础构架。它提供了一系列的工具,可以用来进行数据提取转化加载,这是一种可以存储、查询和分析存储在Hadoop中的大规模数据的机制。Hive定义了简单的类SQL查询语言,称为HQL,它允许熟悉SQL的用户查询数据。同时,这个语言也允许熟悉 MapReduce 开发者的开发自定义的 mapper 和 reducer 来处理内建的 mapper 和 reducer 无法完成的复杂的分析工作,十分适合数据仓库的统计分析。
		\item Apache Spark:Spark是加州大学伯克利分校所开源的类Hadoop的通用并行框架,Spark拥有 Hadoop 所具有的优点;但不同于 Hadoop 的是Job中间输出结果可以保存在内存中,从而不再需要读写HDFS,因此Spark能更好地适用于数据挖掘与机器学习等需要迭代的MapReduce的算法。
		\item Apache Kafka:Kafka 是一种高吞吐量的分布式发布订阅消息系统,它可以处理消费者规模的网站中的所有用户行为流数据。这种用户行为流数据是在现代网络上的许多社会功能的一个关键因素。这些数据通常是由于吞吐量的要求而通过处理日志和日志聚合来解决。对于像Hadoop的一样的日志数据和离线分析系统,但又要求实时处理的限制,Kafka一个可行的解决方案。其目的是通过Hadoop的并行加载机制来统一线上和离线的消息处理,也是为了通过集群机来提供实时的消费。
	\end{itemize}

	\subsection{推荐系统的应用案例}
	近几年随着社会化网络的发展,推荐系统在工业界广泛应用并且取得了显著进步。比较著名的推荐系统应用有:淘宝网的电子商务推荐系统、Youtube的视频推荐系统\citep{recmd-youtube}、网易云音乐推荐系统以及Facebook好友推荐系统。个性化推荐系统具有良好的发展和应用前景。目前,几乎所有的大型电子商务系统,如Amazon、eBay等,都不同程度的使用了各种形式的推荐系统。各种提供个性化服务的Web站点也需要推荐系统的大力支持。在日趋激烈的竞争环境下,个性化推荐系统能有效的保留客户,提高电子商务系统的服务能力。成功的推荐系统会带来巨大的效益。我们每天使用的许多网站中都可找到推荐系统。

	作为全球排名第一的社交网站(\href{https://code.facebook.com/posts/861999383875667/recommending-items-to-more-than-a-billion-people/}{https://code.facebook.com/}),Facebook利用分布式推荐系统来帮助用户找到他们可能感兴趣的页面、组、事件或者游戏等,代表了国外推荐系统的最高发展水平。Facebook中推荐系统所要面对的数据集包含了约1000亿个评分、超过10亿的用户以及数百万的物品,如何在在大数据规模情况下仍然保持良好性能已经成为世界级的难题。Facebook设计了一个全新的推荐系统。Facebook团队之前已经在使用一个分布式迭代和图像处理平台——Apache Giraph。因其能够很好的支持大规模数据,Giraph就成为了Facebook推荐系统的基础平台。在工作原理方面,Facebook推荐系统采用的是流行的协同过滤技术。CF技术的基本思路就是根据相同人群所关注事物的评分来预测某个人对该事物的评分或喜爱程度。从数学角度而言,该问题就是根据用户-物品的评分矩阵中已知的值来预测未知的值。其求解过程通常采用矩阵分解方法。MF方法把用户评分矩阵表达为用户矩阵和物品的乘积,用这些矩阵相乘的结果R’来拟合原来的评分矩阵R,使得二者尽量接近。如果把R和R’之间的距离作为优化目标,那么矩阵分解就变成了求最小值问题。对大规模数据而言,求解过程将会十分耗时。为了降低时间和空间复杂度,一些从随机特征向量开始的迭代式算法被提出。这些迭代式算法渐渐收敛,可以在合理的时间内找到一个最优解。随机梯度下降算法就是其中之一,其已经成功的用于多个问题的求解。SGD基本思路是以随机方式遍历训练集中的数据,并给出每个已知评分的预测评分值。用户和物品特征向量的调整就沿着评分误差越来越小的方向迭代进行,直到误差到达设计要求。因此,SGD方法可以不需要遍历所有的样本即可完成特征向量的求解。交替最小二乘法是另外一个迭代算法。其基本思路为交替固定用户特征向量和物品特征向量的值,不断的寻找局部最优解直到满足求解条件。

	为了利用上述算法解决Facebook推荐系统的问题,原本Giraph中的标准方法就需要进行改变。之前,Giraph的标准方法是把用户和物品都当作为图中的顶点、已知的评分当作边。那么,SGD或ALS的迭代过程就是遍历图中所有的边,发送用户和物品的特征向量并进行局部更新。该方法存在若干重大问题。首先,迭代过程会带来巨大的网络通信负载。由于迭代过程需要遍历所有的边,一次迭代所发送的数据量就为边与特征向量个数的乘积。假设评分数为1000亿、特征向量为100对,每次迭代的通信数据量就为80TB。其次,物品流行程度的不同会导致图中节点度的分布不均匀。该问题可能会导致内存不够或者引起处理瓶颈。假设一个物品有1000亿个评分、特征向量同样为100对,该物品对应的一个点在一次迭代中就需要接收80GB的数据。最后,Giraph中并没有完全按照公式中的要求实现SGD算法。真正实现中,每个点都是利用迭代开始时实际收到的特征向量进行工作,而并非全局最新的特征向量。因此Giraph中最大的问题就在于每次迭代中都需要把更新信息发送到每一个顶点。为了解决这个问题,Facebook发明了一种利用work-to-work信息传递的高效、便捷方法。该方法把原有的图划分为了由若干work构成的一个圆。每个worker都包含了一个物品集合和若干用户。在每一步,相邻的worker沿顺时针方法把包含物品更新的信息发送到下游的worker。这样,每一步都只处理了各个worker内部的评分,而经过与worker个数相同的步骤后,所有的评分也全部都被处理。该方法实现了通信量与评分数无关,可以明显减少图中数据的通信量。而且,标准方法中节点度分布不均匀的问题也因为物品不再用顶点来表示而不复存在。为了进一步提高算法性能,Facebook把SGD和ALS两个算法进行了揉合,提出了旋转混合式求解方法。

	接下来,Facebook在运行实际的A/B测试之间对推荐系统的性能进行了测量。首先,通过输入一直的训练集,推荐系统对算法的参数进行微调来提高预测精度。然后,系统针对测试集给出评分并与已知的结果进行比较。Facebook团队从物品平均评分、前1/10/100物品的评分精度、所有测试物品的平均精度等来评估推荐系统。此外,均方根误差(Root Mean Squared Error, RMSE)也被用来记录单个误差所带来的影响。

	此外,即使是采用了分布式计算方法,Facebook仍然不可能检查每一个用户/物品对的评分。团队需要寻找更快的方法来获得每个用户排名前K的推荐物品,然后再利用推荐系统计算用户对其的评分。其中一种可能的解决方案是采用ball tree数据结构来存储物品向量。all tree结构可以实现搜索过程10-100倍的加速,使得物品推荐工作能够在合理时间内完成。另外一个能够近似解决问题的方法是根据物品特征向量对物品进行分类。这样,寻找推荐评分就划分为寻找最推荐的物品群和在物品群中再提取评分最高的物品两个过程。该方法在一定程度上会降低推荐系统的可信度,却能够加速计算过程。

	最后,Facebook给出了一些实验的结果。在2014年7月,Databricks公布了在Spark上实现ALS的性能结果。Facebook针对Amazon的数据集,基于Spark MLlib进行标准实验,与自己的旋转混合式方法的结果进行了比较。实验结果表明,Facebook的系统比标准系统要快10倍左右。而且,前者可以轻松处理超过1000亿个评分。
\section{研究内容与研究方法}
	推荐系统问题之一是冷启动问题,冷启动问题有三种:用户冷启动、物品冷启动、系统冷启动,本文主要研究用户冷启动问题。经典的算法诸如最近邻的协同过滤算法、PageRank排序算法、关联规则挖掘等算法是给定用户对某些物品的行为数据,给每个用户推荐TOP N个其最喜欢的物品,这种思路对于新注册用户来讲效果不好,因为没有用户行为数据可供分析。解决这个问题的关键是对用户画像建模,实验发现融合用户画像的热门商品推荐是解决冷启动问题的最佳方式。

	推荐系统问题之二是马太效应,即热门商品越来越热,冷门商品越来越冷,在互联网指数级的爆发下信息量极大富余,这更加推动了马太效应的快速形成以及规模的无限扩大,现有的大多数推荐算法更是极大地加速马太效应的形成速度以及规模。我们提出利用用户兴趣探索解决商品的马太效应,提升推荐系统对物品长尾的发掘能力,主要思路是分析用户所有的行为数据,针对冷门商品(冷门商品包含的标签一般是小众标签)的行为会赋予一个倾斜因子,这样会导致兴趣探索标签候选集中的小众标签占大多数,而如果用户对其的满意度也很高,则说明这是一个成功的兴趣探索。这里涉及到的概念包括小众标签的定义和用户满意度的量化,将会在用户兴趣探索章节详细介绍。

	推荐系统问题之三是用户兴趣的动态变化问题,即时效性问题。笔者一直关心的一个问题就是不同系统的用户行为究竟有什么区别,并如何根据这些区别来选择合适的时间参数来预测用户的行为。如nytimes的时效性很短,大部分新闻都是在第一天被很多人关注,而后面就没有人关注了,所以即使很热门的新闻,其生命周期比不热门的新闻长不了太久。其次是blogspot,然后是youtube,最后是Wikipedia,根据用户兴趣时效性可得排序:NYTimes > BlogSpot > Youtube > Wikipedia。其中Wikipedia的斜率很接近最大理论斜率(0.5),这说明Wikipedia的热门的东西完全是因为生命周期长所以才热门,而不是因为在某天特别的火过。因此,正确把握用户兴趣变动的时效性对推荐结果影响很大,本文针对手机主题市场的特点,利用线性衰减算法融合用户画像和用户兴趣探索,其中用户画像代表了用户长期兴趣,用户兴趣探索代表了用户短期兴趣。

\section{论文结构}
	本文的其余正文内容由以下章节组成:
	\begin{itemize}
		\item 第二章首先介绍了推荐系统基本概念和排序模型,包括数据挖掘算法\citep{date-mining}和信息提取技术\citep{info-retrieval}的应用,然后详细介绍了用户画像和用户兴趣探索。
		\item 第三章主要讨论了如何利用用户画像建模解决推荐系统的冷启动问题,从而改善推荐系统的新用户留存率。最后给出了相关的实验结果及分析。
		\item 第四章主要讨论了如何利用用户兴趣探索跟踪用户动态并挖掘用户小众兴趣,从而提升推荐系统的长尾效应\citep{long-tail},文中给出了相关的实验结果及分析。
		\item 第五章是论文的结束语和展望,在对目前工作简要总结的基础上,提出了推荐系统下一步研究的任务和方向。
	\end{itemize}
   \chapter{基于用户画像的推荐系统综述}
	\section{引言}
	自从1992年著名的施乐公司的科学家们为了解决困扰已久的信息负载问题,第一次从概念上提出协同过滤的算法模型。1998年,林登及其同事们成功申请了item协同过滤技术的专利,经过多年的工程实践,美国电商亚马逊公司的工程师们骄傲的宣称:在公司所有的销售量,推荐系统占比已经占到整个Gross Merchandise Volume的百分之三十以上。不久之后的美国公司Netflix,因为其创始人与前任公司签署有若干年内不得从事同行工作的限制,于是通过举办推荐算法优化竞赛绕开限制,用以开发出更好的推荐算法。此次竞赛吸引了数以千计的团队参与角逐,期间进行了上百种的算法模型组合、优化的尝试,虽然Netflix公司为冠军团队支付了百万美金,但回报是Netflix推荐系统的快速发展以及营收的俩位数增长。其中冠军团队凭借Sigular Value Decomposition和Gavin Potter跨界引入的心理学方法进行的组合算法模型,在诸多优秀团队中脱颖而出。其中,矩阵分解的核心是将一个非常稀疏的用户评分矩阵R分解为两个更小的矩阵:只包含User特性的矩阵P和只包含Item特性的矩阵Q,利用P和Q相乘的结果R'来拟合原来的评分矩阵R,使得矩阵R'在R相同位置之间的损失函数值尽量的小,通过定义一个R和R'之间的距离定义(一般为曼哈顿距离),如果矩阵R'是正定矩阵,那么把矩阵分解转化成梯度下降求解的局部最优解,就是全局最优解。与此同时,Pandora、LinkedIn、Hulu等网站在个性化推荐领域都展开你争我抢的竞争势头,使得推荐系统在各个细分行业、垂直领域开始全面开花,都有了不少爆发性进展。但是,对于拥有全品类的综合性购物电商、广告营销,推荐系统的进展还是缓慢,主要原因是因为不同类型的商品,消费者的心态也是不同的,例如大型家电,消费者肯定是先看了又看、选了又选,从价格、定位、功能到噪声比、性价比,大多数都会先做足了调查,才会购买;与此相反,对于日常用品消费者可能眼睛都不眨就购买了,对于这俩种极端的消费情况,推荐系统需要做出截然不同的推荐策略,具体的,单个模型在母婴品类的推荐效果还比较好,但在其他品类就可能很差,很多时候需要根据场景、推荐栏位、品类等不同,设计不同的推荐模型。同时由于用户兴趣随时间会不停的变动,需要一种机制,使得推荐系统能定期对数据进行评估、分析,要命的是对于不同类型的商品有不同的更新频率,这就对推荐系统提出了更加智能化的挑战。还有,如果定期更新模型,则可能会因为计算资源的限制导致无损害推荐的实时性,因为模型训练也要相当cpu计算时间,而传统的Hadoop的方法实在是无法进行大的更新频率,spark框架又因为昂贵的内存限制了其计算容量,最终业务会到达一个数据量,此时的推荐效果会因为实时性问题达到第一个计算瓶颈。推荐算法包括基于人口统计学的推荐\citep{social-filter}、基于商品内容的推荐\citep{content-based}、基于user/item的协同过滤\citep{collab-filter}的推荐等。基于内容的推荐\citep{recmd_content_based}对物品冷启动问题免疫,但是无法解决用户冷启动问题\citep{cold-start},还有过拟合的问题:即在训练集上有比较好的表现,但在实际应用中效果往往不尽人意,推荐系统的通用性和移植性往往比较差,适合针对细分行业下的商品做推荐,一旦换了产品类型,往往需要构建新的模型。基于邻域的协同过滤算法,虽然没有领域知识要求,算法通用性好,但存在有冷启动问题、数据稀疏性问题。

	由此,笔者在实际工程中,针对传统推荐算法的种种弊端,选择了用户画像。伟大的数学家、计算机学家Knuth先生说:如果遇到一个不好搞定的问题,那么就该添加一层中间层,用以屏蔽掉问题。实际上,用户画像作为底层数据仓库和上层推荐系统的缓冲层,起的就是这种作用。

	\section{用户画像的研究现状}
		\subsection{用户画像的组成部分}
		基于内容和用户画像的个性化推荐,有两个实体:内容和用户。需要有一种文本机制联系这两者的东西,我们定义其为标签。内容特征文本化为标签即为内容特征化,用户兴趣文本化标签则称为用户特征化\citep{user-profile,user-profile1,user-profile2,user-profile3,user-profile4}。因此,对于基于用户画像的推荐,主要分为以下几个关键部分:
		\begin{enumerate}[(1)]
		\item 标签库

		标签是联系用户与物品、内容以及物品、内容之间的纽带,也是反应用户兴趣的重要数据源。标签库的最终用途在于对用户进行行为、属性标记。是将其他实体转换为计算机可以理解的语言关键的一步。标签库则是对标签进行聚合的系统,包括对标签的管理、更新等。在用户画像的过程中有一个很重要的概念叫做颗粒度,就是我们的用户画像应该细化到哪种程度。举一个极端的例子,如果“用户画像”最细的颗粒度应该是细到每一个用户每一具体的生活场景中,但是这基本上是一个不可能完成的任务,同时如果用户画像的颗粒度太大,对于产品设计的指导意义又相对变小了,所以把握好画像的总体丰富程度显得异常重要了。可通过调查问卷的形式来减小颗粒度。一般来说,标签是以层级的形式组织的。如体育为一级维度、篮球为二级维度、NBA篮球为三级维度等。

		\item 内容特征化

		内容特征化即给商品打标签。目前有两种方式:人工打标签和机器自动打标签。针对机器自动打标签,需要采取机器学习的相关算法来实现,即针对商品描述文本,生成一系列标签,为商品选取其中匹配度最高的几个标签。这不同于通常的分类和聚类算法\citep{recmd-kmeans}。可以采取使用分词 + Word2Vec来实现,过程:将文本语料进行分词,以空格,tab隔开都可以,使用结巴分词。使用word2vec训练词的相似度模型。使用tfidf提取内容的关键词A,B,C。对每个现存的标签,计算关键词与此标签的相似度之和。取出TopN相似度最高的标签即为此商品的标签。如对于《小羊肖恩》主题包,现有儿童、动漫俩个标签,描述文本有:一部史诗般的二次元欢乐片。经计算“二次元”关键字与现有标签相似度最高,则更新二次元到此商品的标签库中。

		\item 用户特征化

		用户特征化即为用户打文本标签。通过用户的行为日志和一定的模型算法得到用户的每个标签的权重。用户对内容的行为:点赞、不感兴趣、点击、浏览。对用户的反馈行为如点赞赋予权值1,默认为0,不感兴趣为-1;对于用户的浏览行为,则可使用点击、浏览作为权值。对商品发生的行为可以认为对此商品所有标签的行为。用户的兴趣是时间衰减的,即离当前时间越远的兴趣比重越低。时间衰减函数使用1/[log(t)+1], t为事件发生的时间距离当前时间的大小。要考虑到热门内容会干预用户的标签,需要对热门内容进行降权。
		\end{enumerate}

		\subsection{用户画像的构建周期}
		用户画像,即用户信息标签化,就是企业通过收集与分析消费者社会属性、生活习惯、消费行为等主要信息的数据之后,获得用户的数据标签库。构建周期如\autoref{pic:userprofile_process}。
		\begin{figure}
	    \centering
	      \framebox{\includegraphics[scale=0.45]{figures/userprofile_process}}
	      \figcaption{用户画像的构建周期示意图}
	      \label{pic:userprofile_process}
	    \end{figure}
	    \begin{enumerate}[(1)]
	    \item 数据收集

	    数据收集大致分为四类:1、网络行为数据包括页面浏览量、活跃人数、访问时长、浏览注册转化率、注册活跃转换率等。服务内行为数据:点击浏览路径、网页停留时长、滑屏次数、滑屏频率、滑屏时长。用户内容偏好数据:点击、浏览、收藏内容、评价、评分、评论内容、社交内容、品牌偏好等。用户交易数据(交易类服务):购买率、折扣率、导流率、流失率等。收集到的数据没必要是百分之百的准确,大体差不多即可。应用中,具体就是在数据清洗阶段过滤一部分不靠谱的异常值,验证、更新数据这块需要在后面的阶段中建模来再判断,比如某用户在性别一栏填的女,但其语言数据显示其为男的概率更大,根据业务再选择丢弃数据还是更新数据。
	    
	    \item 行为建模

	    该阶段是对收集到数据进行建模,目标是抽象出用户的文本标签,这个阶段不应该再纠结数据的正确性,而是应该注重大概率事件,通过统计学假设检验尽可能地排除用户的偶然行为。这时也要用到数据挖掘算法模型,对用户的行为进行回归预测,比如已有一个线性回归函数:y=kx+b,X 代表用户行为,y是函数拟合的用户喜好度,y'是用户真实偏好,我们通过不断的训练数据,利用参数k和参数b来得出最新损失函数下的值,用以精确模拟y'。

	    \item 用户画像基本成型

	    该阶段是行为建模的深化,需要利用用户的基本属性,如性别、地域、年龄,得出用户更高层的抽象概念:消费能力、忠诚度、活跃度、社交爱好等。因为用户画像永远也无法百分百地拟合现实中的一个人,只能做的就是不断地去减小拟合的损失函数,因此,用户画像需要根据变化的基础数据不断修正已有的更高层的抽象概念,尽可能模拟用户的变化趋势。

	    \item 数据可视化

	    最后是数据可视化分析,这部分是最能体现推荐系统的产出,因为人类对数据不如对图画来的敏感,在此步骤中一般是针对群体做进一步的抽象,按照消费习惯、消费能力、消费偏好把用户归类为一类人,比如可以根据用户对价格的敏感度细分出高价值用户、核心用户、高忠诚用户。而决策层所做出的评估也应该是基于某一群体的潜在价值分布。典型的用户画像如\autoref{pic:user_profile}
	    \end{enumerate}
		\begin{figure}
	    \centering
	      \framebox{\includegraphics[scale=0.4]{figures/user_profile}}
	      \figcaption{用户画像示意图}
	      \label{pic:user_profile}
	    \end{figure}

		\subsection{用户画像的建模}
		用户画像的建模包括内容标签化和标签权重量化。建模过程:1、内容分析,从原先的物品描述信息中提取有用的信息用一种规范化的标签表示,有时候这种信息源自于作者提供的描述,有时候源自于用户的评价,不管如何,都需要人工审核验证正确性;2、上传、记录用户注册信息,生成用户基本信息,这些信息基本是不会变化的;上传、记录用户行为数据,这些数据是不断变化着的,通常是采用数据挖掘算法从潜在物品集合中取出若干个结果表示用户喜好的模型。例如,一个网页推荐系统,可以通过分析用户过往浏览过的文章,得出用户喜欢浏览类似于范冰冰的花边新闻,如果用户点击了所推荐的文章,则说明分析正确,否则需要根据反馈重新训练模型,从而实现一个反馈-推荐-反馈的闭环;3、推荐系统得出推荐集合后往往需要取topN,因为推荐系统的本质在精不在多。通过定义一个距离算法,匹配用户标签和商品标签的相关度,相关度一般正则为0-1之间,结果是一个二元的离散量:<pid,score>。根据相关度将生成一个用户潜在感兴趣的物品评分列表,然后去掉用户之前看过的商品,取topN即可。例如在电影用户画像的建模中,首先分析用户打分比较高的电影的共同特性,包括导演、演员、风格等,这些电影的标签就会成为此用户画像的一部分,根据打分的多少,给定一个合适的权重值。用户-标签用矩阵A表示,电影-标签用矩阵B表示,A乘B得出矩阵C,C代表了用户与电影之间的相关度,固定一个用户,对所有相关度不为零的电影做排序,取topN即是推荐结果。用户画像建模的根本在于用户标签的获取和权重的定量分析。

		对于商品描述,也可以做进一步的处理,丰富商品的标签集合。其实和文本处理类似,笔者选择使用目前应用最广泛的方法:TF-IDF方法。设有N个文本文件,关键词$k_{i}$在$n_{i}$个文件中出现,设$f_{ij}$为关键词$k_i$在文件$d_j$中出现的次数,那么$k_i$在$d_j$中的词频TF$_{ij}$定义为:TF$_{ij}$=$f_{ij}$/max$_zf_{zj}$,其中分母中的最大值是通过计算这个文本j中所有关键词出现的频率得出。附图给出了3个短文和5个关键词,以关键词人为例,该关键词在文本1中出现了1次,而文本1中出现次数最多的关键词是事,一共出现了2次,因此TF$_{11}$=0.5。一个关键词经常在许多文件中出现,则该关键词能表示文件的特性的意义就会较小,试想我们考察关键词i出现次数的逆,也就是$IDF_{i}$=log(N/$n_{i}$),这个想法和Adamic-Adar指数思路基本相似,关键词i在文本文件j中的权重于是可以表示为$w_{ij}$=$TF_{ij}$*$IDF_{i}$,而文件j可以用一个向量$d_{j}$=($w1_{j}$,$w2_{j}$,…,$wk_{j}$),其中k是整个文本库中关键词的个数。一般而言,向量应该是一个稀疏向量,即其中很多元素都为0。如果把用户今日点击、浏览、购买的商品抽象成一个标签向量,则可以通过用户标签向量-商品标签向量的点乘得出一个数值,从所有数值中把相似性最大的那个产品的标签更新给该用户画像,第二大相似性的产品标签权重减半更新给该用户画像,以此类推,完成用户画像的建模过程。
		\begin{lstlisting}
		文本1:不做软事,不说硬话,对事不对人。
		文本2:多少事,从来急;天地转,光阴迫。一万年太久,只争朝夕。
		文本3:青春之所以幸福,就因为它有前途。

		关键字包括人、事、硬话、一万年、朝夕、青春、幸福、前途
		\end{lstlisting}

		\subsection{用户画像和推荐系统的评测}
		首先,用户画像作为一个工具,只用在运用到某一场景才有意义,并能评估出其产出,因此本节主要介绍推荐系统的评测,根据推荐系统的表现好坏才能评估出用户画像的推荐质量,本节介绍评测推荐系统常用的实验方法。
			\begin{enumerate}[(1)]
			\item 离线实验,从日志系统中直接取得用户最近单位时间的行为数据,然后将这些数据分成俩部分:训练数据和测试数据,一般来讲倆者的比例大致为:8比2,然后利用训练数据集迭代拟合用户的兴趣模型,在测试集上进行回归测试。过程简单、容易模式会管理,不需要人为干预,有很多的数值计算的开源软件库可以用:如google公司出品的TensorFlow。能方便快捷的测试大量不同的算法。
			\item 用户调查,又叫问卷调查。离线实验得出的是客观规律下的准确率,但是客观的准确率不等于用户实际的满意度,一般来说问卷调查需要只需要在小范围之内进行,即可得出大差不差的用户满意度调查,优点是可操作性强。
			\item AB测试,标准的AB测试是指通过一定的规则把类似的用户群随机分成俩组,采用旧模型的分组叫对照组,采用新模型的分组叫实验组。通过对用户展示不同的模型,得出用户的使用指标,关键是各种转化率,这样仅仅通过对比倆者的转化率即可得出各个模型的优劣。策略实验的难点在于如何找到合适的实验设计方案。通过时间交错能够在一定程度上减少由时间片带来的误差,这样就有一个难题:  如何选择合适长度的时间片。策略实验往往伴随着携带效应(carry-over effects),也就是上一个时间片的策略会对下一个时间片带来影响。笔者和同事们提出一个方案,当选择适当大的时间片的时候,通过A/A测试的数据调整A/B的结果,具体来说,如果A/A 的结果是 0.4\%, A/B 的结果是 1.2\%. 那么我们认为A/A  是真实的时间片之间的差异, 我们需要用 A/B - A/A 去调整时间片带来的影响,
			\end{enumerate}

	\section{用户画像在推荐系统的应用现状}
	Amazon的仓库里堆着数百万图书,Netflix的服务器中存储有数万部电影,淘宝平台上的小卖家总共拥有8亿件物品,除此之外,这三家公司都保留有数以亿计的用户行为数据。互联网电子商务开始积累了海量的用户数据,然后因为数据量过于庞大,有用信息如金矿中的金子一样很难挖掘利用,与此同时,用户发现常常需要面对过多的选择。心理学研究证实过多的选择会使人犹豫不决,导致消极等待,最终可能放弃消费的决定,这个问题严峻到可以造成肉眼可见的用户流失。近代统计学理论的发展加上最近几年的数据科学和数据挖掘工程的进步,为电子商务平台提供更有效的应对方案:推荐算法。推荐系统在帮助用户解决信息过载问题的同时,提升了企业价值。如今的企业不再局限于传统的推荐功能,通过建立完备的用户画像,推荐系统可以帮助企业更了解用户,在推广、反作弊、精细化运营等领域中发挥重要的作用。

	目前使用最广的推荐系统,主要是基于内容做推荐,根据RecSys大会(ACM Recommender Systems)中与会者的反馈,已经有不少公司和研究者先行一步,尝试基于用户画像做推荐。利用用户的画像,结合空间、时间、天气、环境、经纬度等上下文信息,可以给用户带来不一样的感受。用户画像是一种更高级的工具,在解决把数据转化为商业价值的问题上更甚一筹,相当于从海量数据中挖出俩倍的金银。用户画像中包含着高质量多维的数据,用以记录用户长期的行为,据此还原用户真实的消费特征、教育背景、兴趣偏好。科学中国网曾在《大数据揭秘:淘宝上的假货、次品都卖给了谁?》中报道了淘宝不良商家如果利用买家信息欺骗消费者\citep{liar_taobao}:1、分析数据看人下刀,宰用户没商量,真相就是消费者的消费记录、购买记录、客单价等都将作为参考数据被系统识别,商家会根据这些记录评估消费者能不能分辨假货,再把假货卖给对方。2、看退货率,专欺负老实人,消费者的退货率、投诉率也会被识别到系统里,这些数据帮助商家判断用户好不好惹,退货率低于百分之十的用户,会收到更多次品产品。3、看收货地址,决定给用户发什么货,一些淘宝店家还会根据用户收货地址所在城市,决定给用户发什么货。要是用户所在城市没有该品牌的专卖店,或者用户没有购买过该品牌的产品,那系统将会放心的把假货或者仿品发给用户。利用用户画像人们可以做到如此精准的销售,当然上述例子是用户画像极其错误的用法。
		\subsection{基于用户画像的推荐系统的商业应用}
		\begin{figure}
	    \centering
	      \framebox{\includegraphics[scale=0.45]{figures/recmd_facebook}}
	      \figcaption{Facebook个性化推荐用户界面}
	      \label{pic:recmd_facebook}
	    \end{figure}
		作为全球社交网站中的翘楚,Facebook在很早的时候就预言到了大数据+推荐系统+用户画像的无限前景。Facebook自己的推荐系统就是需要利用分布式计算框架快速的帮助用户找到他们可能感兴趣的人、文章、分析、用户组等。Facebook是个伟大的公司,一直为开源软件贡献着一份力量,最近在其官网就公布了facebook自己的推荐系统原理、性能及使用情况\citep{recmd-facebook}。Facebook的推荐系统需要面对的数据量应该是所有互联网公司中的数一数二,约包含了1000亿级别的评分数、10亿级别的用户数以及百万级别的虚拟商品,如何在如此庞大的数据规模下,仍然保持良好性能已经成为世界级的难题,而facebook解决了。即使是采用现在流行的分布式计算框架,Facebook仍然不可能穷举每一对用户-物品的评分。团队需要寻找效率更高、耗时更少的算法来获得每个用户topN的推荐物品,然后再利用推荐系统计算用户对其的评分,这与我们之前解释的恰好相反。解决方案是利用ball tree数据结构存储商品的权重向量。all tree可以贡献搜索过程10-100倍的加速率,使得推荐结果能够在合理时间内完成,典型的以空间换时间策略。最后,通过分析Facebook给出了一些实验的结果,表明,Facebook的系统比传统系统要快10倍左右。因此可以轻松愉快的处理1000亿级别的评分数据。目前,该方法已经用到Facebook的多个app应用中,包括用户、用户组的推荐。进一步的,为了能够减小系统负担,Facebook只是把稀疏度超过100的用户考虑为候选推荐集合。在初始迭代中,Facebook推荐系统直接把用户历史上喜欢过的主页、群组以及不喜欢的群组都作为输入。最重要的是Facebook还利用ALS算法,从用户获得间接的反馈,这样算是完成推荐-反馈-优化-推荐的一个完美闭环。未来Facebook会继续优化推荐系统,持续改进部分关键模块,包括社交图、用户跳转路径、自动化参数调整以及较好的机器负载均衡策略等。Facebook推荐主页如\autoref{pic:recmd_facebook}。
		
		Facebook的用户画像进展也十分可观,几乎是与推荐系统同步发展。2011年12月,Facebook发布了里程碑式的大数据产品——Timeline,通过开发API接口,允许用户自行编辑个人的时间轴:在什么时间、什么地点做了什么,遇到了谁,可以说在这条时间线记录这个人的全部生活故事。Timeline通过帮用户回忆自己的点点滴滴的同时,完成了用户数据捕获、存储,而一旦拥有了这些历史数据,Facebook就可以做进一步的数据分析、挖掘,这时的Facebook就如同和你从小长大的小伙伴,一个懂你的陌生人。可以说用户留下的数据越多,Facebook就越了解这个人,投放的广告就会更加精准,最终facebook利用庞大的用户数据生态赚足了钱。

		豆瓣网是国内互联网行业中的小清新,美誉度很高,这是一家致力于帮助用户发现美好事物为己任的公司\citep{recmd-douban}。不用费力设置播放列表,也不用费心思考要听啥,打豆瓣电台的推荐栏目,就能获得意想不到的快乐。如初恋般的音乐体验,让用户和音乐不期而遇,豆瓣电台坚持找到符合用户口味音乐。通过高度匹配的推荐结果,豆瓣电台为音乐爱好者提供了这样一种崭新的音乐盛宴,音乐本来就是件轻松的事,豆瓣电台回归了音乐最初定位。豆瓣电台的推荐算法综合用户的各种音乐行为\citep{recmd-doubanFM}。在豆瓣音乐中,通过量化音乐标签,谁喜欢哪些歌手,在听哪些,想听哪些,乐评,豆列等指标,会有相关的权重算法算出一个数值,最开始的时候只是一个最简单的计算公式,在经过多次产品迭代和用户反馈后,得出更靠谱的权重值累加算。其中权重最多的应该是电台本身的红心、踩、跳过、这些显性行为数据。豆瓣电台糅合了包括数据清洗、分析、挖掘、整合、用户画像建模、编辑与运营、后台架构等等大量的因素,如此庞大的架构中,即便是推荐算法也只是现实的一部分。豆瓣电台推荐页如\autoref{pic:recmd_doubanFM}。

		豆瓣电台的用户画像结构大致有俩快:享受时间和购买时间,用户画像的目标用户群体是在线观众,通过用户购买时间差区别并得出分类标签。如通过分析得出此类用户大多数是周末购买音乐工作日静下心慢慢享受。用户年龄、职业和地址,这是根据用户的经纬度和注册信息来对用户进行分类的用户画像建模。用户的经纬度分为一线城市,如北京、上、二线城市,如武汉、深圳、厦门、其他三类,如合肥、呼和浩特。年龄分为小于25岁、26到35岁、36到45岁和46岁以上四类。据统计结果表明:按人群经纬度分布,大致与橄榄球相似,二线城市的人群占中间的大部分,其他城市人数飞速增长;按用户年龄分布,九十后用户占主体地位。同时对情侣关系的用户推荐喜欢度接近的音乐。按活跃程度分布主要分为3档:100次以下;100-300次;300次以上。也可以同时考虑多个维度,包括活跃度、经纬度、年龄段,进行用户画像建模。
		\begin{figure}
	    \centering
	      \framebox{\includegraphics[scale=0.5]{figures/recmd_doubanFM}}
	      \figcaption{豆瓣电台个性化推荐用户界面}
	      \label{pic:recmd_doubanFM}
	    \end{figure}

		\subsection{推荐系统的主要方法}
		推荐系统主要有俩种思路:评分预测和Top-N预测,核心的目标都是找到最适合用户的候选集合s,从候选集合里挑选目标集合是一个非常复杂的非线性优化问题,通常采用的方案是用局部最优近似非线性最优,通过定义一个的损失函数,选取Top-N	\citep{recmd-Next}。
		\begin{enumerate}[(1)]
		\item 协同过滤的推荐

		推荐系统的算法基于统计学、概率论、线性代数、微积分技术,找出用户最有可能喜欢的商品,应该是现代互联网电商的明星应用。目前用的比较广泛的推荐算法还属协同过滤推荐算法,其基本思想是根据与他兴趣相近的用户的选择,得出推荐商品候选集,取topN推荐给目标用户,用维度为m×n的矩阵表示所有用户对所有物品的兴趣值,这个值应该是根据用户历史行为数据得出,值越高表示这个用户越喜欢,利用特殊值0表示没有接触过。图中行向量表示某个用户对所有商品的喜爱程度,列向量表示某个商品对所有用户的吸引程度,因此单个元素$U_{ij}$表示用户i对物品j的喜欢程度。协同过滤分为两个阶段:预测阶段和推荐阶段。预测阶段是基于所有原始集商品,预测这个用户有没有可能对其感兴趣,量化为一个数值,只要值不为零即可归为候选集中;推荐是根据预测结果,先去重后去除消费过的商品,然后根据某种算法去TopN推荐给用户。按照用户-商品数值的得出类型,协同过滤算法分为基于内容的和基于模型俩大类\citep{Wikipedia}。

		协同过滤算法的基本思路是基于一个假设:如果某类用户群对相同商品的打分比较类似,则表明他们的品味从某种程度上类似,由此可以推出在其它类项目的打分也应该类似才对。协同过滤推荐系统先定义好距离计算公式,然后搜索与目标用户相似的其他潜在类似用户,并根据类似用户的打分来量化潜在用户对指定商品的评分,最后选择评分最高的商品列表推荐给用户,同时可以给出令人信服的推荐理由:如某某人也购买过、评价过该商品。这种算法的优点很多:计算简单、精确度较高,能够自圆其说,因此被广泛采用。总之,基于User的协同过滤推荐算法的核心,就是先通过距离计算公式得出类似邻居,然后将最近邻的好评过的商品推荐给目标用户,很简单。

		例如,在\autoref{tab:User-based}所示的用户一商品评分矩阵中,行向量代表用户,列向量代表电影。表中的数值代表用户对电影的评价量化后的值。现在需要预测用户Hanmeimei对电影《教父》的评分(用户maggie对电影《X-Files 要你相信》的评分是缺失的数据)。
		由\autoref{tab:User-based}不难发现,Lane和Pony对电影的评分非常接近,Lane对《暮色3:月食》、《唐山大地震》、《X-Files 要你相信》的评分分别为3、4、4,Hanmeimei的评分分别为3、5、4,他们之间的相似度最高,因此Lane是Hanmeimei的最接近的邻居,Lane对《教父》的评分结果对预测值的影响占据最大比例。相比之下,用户Jackson和maggie不是Hanmeimei的最近邻居,因为他们对电影的评分存在很大差距,所以Jackson和maggie对《教父》的评分对预测值的影响相对小一些。
		\begin{table}[htp]
		\centering
		\tabcaption{用户-物品表}
		\label{tab:User-based}
		\begin{tabular}{ |c|p{2cm}|p{2cm}|p{2cm}|p{2cm}| } \hline
		 & 暮色3:月食 & 唐山大地震 & X-Files 要你相信 & 教父 \\ \hline
		Jackson & 4 & 4 & 5 & 4 \\ \hline
		Marry & 3 & 4 & 4 & 2 \\ \hline
		maggie & 2 & 3 &  & 3 \\ \hline
		Hanmeimei & 3 & 5 & 4 &  \\ \hline
		\end{tabular}
		\end{table}
		\end{enumerate}
		尽管有这么多的优点,协同过滤算法也存在两大问题:1、数据稀疏性。一个大型的电子商务平台一般有百万级别的物品,用户可能接触到的商品占所有商品的百分之一不到,因此用户之间购买过的物品重叠性非常小,以至于没办法做推荐,一个办法是利用算法添补部分值\citep{recmd-slopone}。2、扩展性较差,因为一般来讲,电子商务平台中的商品变动很小,用户流入流出、日益增加、变动很大,基于用户的协同过滤算法需要不停的跟新迭代保证跟上用户变动的步伐。遇到这种情况,可以考虑基于商品的协同过滤算法,其基本思想类似于基于用户的协同过滤算法,只是相似性计算对象是商品,而商品一般变动很小可以忽略不计。如果我们知道物品a和b相似,而一般喜欢a的用户也喜欢b,如果用户A喜欢a,那么我们有很大把握得知A也应该喜欢b,推荐了准没错。而物品之间的相似性比较固定,因此可以一次性计算出物品的相似度,将结果存储到redis中,推荐时查询redis即可。

		\subsection{推荐系统评测的测量指标}
		推荐系统存在三个参与方:用户、物品提供者和平台。好的推荐系统总体来说是一个能令三方共赢的系统。那么如何评价推荐系统功效呢? 从用户角度,推荐系统必须满足用户的需求,推荐的应该是那些令用户感兴趣的、之前又没有遇到过的商品,即推荐精度。推荐系统还应该有预测用户行为的功能,通过历史展望未来,帮助用户发现那些他们原本没机会发现的小众商品,即长尾效应。最后推荐系统也应该能引导用户兴趣,推荐一些商品,虽然与用户兴趣无关,但是用户看见可能会产生兴趣的商品,即惊喜度。从平台角度,推荐系统能够让平台的营收上一个台阶。
			\begin{enumerate}[(1)]
			\item 用户满意度

			用户满意度是最难量化的指标,也是最关键的指标。推荐系统的本意就是让用户满意。量化用户满意度可以采用用户问卷调查,还有一种更直接的方式,就是在推荐结果的侧栏设置俩个按钮,方便用户在线实时反馈意见,据笔者所知豆瓣的推荐物品旁边都有这类按钮,而亚马逊另辟蹊径,利用一些关键性指标衡量用户对推荐系统的满意度,一般用点击率,用户停留时间,转化率等指标来度量。
			\item 预测准确度

			如果是评分机制,则一般通过计算预测结果集合与用户实际消费集合直接的重合度,得出推荐系统的准确度。如果是Top-N推荐,则涉及到关键指标:召回率和准确率。准确率指在所有的推荐结果中有多少个是对的,其所占的比重,以推荐结果集合个数当除数。召回率则是指用户实际消费商品集合中,有多少物品出现在推荐结果中,已用户实际消费商品集合个数当除数。
			\item 覆盖率

			就是指推荐系统有没有照顾小众商品,而不是一个劲的推荐热门商品。方法就是统计推荐结果的类型个数,比上所有商品类型个数,得到的商越大代表覆盖率越好,其他方法就涉及到信息学中的熵和基尼系数。
			\item 多样性

			针对某一个用户,推荐结果中要变化多端,不能一根筋的推荐一种类型。比如电影,如果用户即格斗类的电影,同时又喜欢爱装小清新,那么推荐列表中就应该是两个类型的集合,除此之外,适当添补一些小众电影,三者比例按用户的爱好来推荐,比如用户爱格斗片多一点,文艺片也喜欢,历史片只是偶尔,那么推荐结果中最好也跟这个比例大差不差。
			\item 新颖性

			如果系统推荐的物品其实是用户知晓的,那么这就是一次失败的推荐,完全失去了推荐的意义。一般来讲,用户都是期望推荐一些自己暂时还不知道的商品或者没看过没买过的商品。方法之一是过滤掉用户已经看到过、购买过、点击过的物品,除此之外,物品的平均流行度与其新颖度成反比,越冷门的物品越会给用户新颖的感觉。比如用户是周星驰的粉丝,那么推荐《临岐》就是一个很棒的选择,因为很少人知道这是周星驰出演的。
			\item 信任度

			如果一个用户信任推荐系统,那么他不仅会频繁的选择查看推荐结果,还有适时的与推荐系统互动,包括反馈、评价、提建议等。如果用户信任推荐系统,从而获得更好的个性化推荐,这是一个良性循环。
			\item 实时性

			有时候一个推荐系统的实时性的重要性大过天\citep{temporal-cf},试想一下,如果一个用户要买睡袋,但不知道哪款睡袋好,推荐系统如果这是恰当好处的推荐结果,那么对于用户是很有意义的一件事情。反之,等用户买都买了,推荐系统在作出推荐,只会让用户难堪。
			\end{enumerate}

	\section{本章小结}
	本章简单概述了用户画像的研究现状,讨论了相关的建模过程。然后介绍了推荐系统的主要任务和问题,并从商业应用和学术研究两个角度介绍了推荐系统研究的现状,最后讨论了推荐系统的主要评测指标。
   
\chapter{用户画像建模}
\label{chap:example}
Alan Cooper(交互设计之父)最早提出了用户画像(persona)的概念:“Personas are a concrete representation of target users”。Persona 是真实用户的虚拟代表,是建立在一系列真实数据(Marketing data,Usability data)之上的目标用户画像。通过用户历史行为去了解用户,根据他们的目标、行为和观点的差异,将他们区分为不同的类型,然后每种类型中抽取出典型特征,赋予名字、照片、一些人口统计学要素、兴趣标签等描述,就形成了一个人物原型(personas),\autoref{pic:hl_userProfile}所示为一个典型的用户画像,标签面积越大代表其权重越高。一些大公司很喜欢用personas做用研究,比如阿里,腾讯,微软等,刻画每个用户,是任何一家社交类型的服务都需要面对的问题,不同的公司针对各自业务会有不同的需求,构建用户画像的动机和目标也会存在一定差异。从手机主题应用商城的角度来讲,构建用户画像的目的包括:

\begin{figure}
\centering
  \framebox{\includegraphics[scale=0.35]{figures/hl_userProfile}}
  \figcaption{用户画像标签化}
  \label{pic:hl_userProfile}
\end{figure}

\begin{itemize}
\item 完善及扩充用户信息。用户画像的首要动机就是了解用户,这样才能够提供更优质的服务。但是在实际中用户的信息提供得不尽完整,有些是因为平台的引导机制造成的,有时候又是用户不愿意或懒得提供,而且对于用户自行输入的内容又很难进行规范化此外,一些隐性或变化频繁的信息也需要通过用户的行为挖掘出来。
\item 打造健康的主题设计生态圈。在掌握用户信息的基础上,平台就可以对自身的状况进行分析,从相对宏观的基础上把握主题市场的生态环境,挖掘设计作品的最大价值,帮助设计师提高收入,\autoref{pic:hl_income}。例如通过对用户信息的聚类,能够对用户进行人群的划分,掌握不同人群的活跃程度、行为及兴趣偏好,热门主题的传播方式和流行引爆点等。
\begin{figure}
\centering
  \framebox{\includegraphics[scale=0.45]{figures/hl_income}}
  \figcaption{2015年Q1热销主题排行榜}
  \label{pic:hl_income}
\end{figure}

\item 支撑主题推荐系统的精准推荐。精准推荐的前提是对用户的清晰认知。以简单代金券发放为例,手机主题应用市场的历史数据呈现出两大类四种不同的消费习惯。代金券敏感型:发代金券才用、发代金券用的更多;代金券不敏感型:发不发都用,发代金券也不用。在推荐系统的用户画像系统中,上述四种群体会被分别冠以屌丝、普通、中产、土豪的标签。针对四类用户的运营策略也会全然不同,最直接的就是代金券的刺激频率以及刺激金额,而对“代金券”免疫的土豪群体,则更多地需要在优化服务上做文章。在实际场景中,影响用户对手机主题包的使用黏度的因素要远比代金券复杂得多,在这种情况下,利用用户画像可以对用户的“贴身跟踪”就能及时发现薄弱环节,因此从用户打开应用商店到退出使用,其间的每一步情况都被快的记录在案:哪一天退出的,哪一步退出的,退出之后“跳转”到什么软件等等。据此,用户画像也实现了用户另外一个纬度的归类,分清哪部分是忠实用户,哪部分可能是潜在的忠实用户,哪些则是已经流失的;更进一步来看流失的原因:因为代金券没有了流失?主题包质量不好流失?这些都是下一步精准推荐的依据。其实手机主题市场中的各项业务都与用户画像有着直接与间接的关系,无论是基于兴趣的推荐提升用户价值,精准的广告投放提升商业价值,还是针对特定用户群体的内容运营,用户画像都是其必不可少的基础支撑。直接地,用户画像可以用于兴趣匹配、关系匹配的推荐和投放;间接地,可以基于用户画像中相似的兴趣、关系及行为模式去推动用户兴趣和设计师的无缝对接。
\item 主题市场安全领域的应用。随着手机主题市场的发展,商家会通过各种活动形式的补贴来获取用户、培养用户的消费习惯,但同时也催生一些通过刷排行榜、刷红包的用户,这些行为距离欺诈只有一步之遥,但他们的存在严重破环了市场的稳定,侵占了活动的资源。其中一个有效的解决方案就是利用用户画像沉淀方法设置促销活动门槛,即通过记录用户的注册时间、历史登陆次数、常用IP地址等,最大程度上隔离掉僵尸账号,保证市场的稳定发展。
\end{itemize}

用户画像的目的是将用户信息标签化,本文介绍针对主题应用商店本身的特点介绍用户画像的构建,该用户画像主要还是从电子商务的角度出发,完善用户信息和发掘用户兴趣,区分兴趣和购买意愿,并形式化、结构化表达出来。数据的来源也主要是主题平台本身,并没有采用更多的第三方数据。

    \section{用户画像的数据来源}
    手机主题用户画像的信息来源可以有如下几种方式:
    \begin{itemize}
    \item 显式用户行为。显式方法主要是通过获取用户注册信息中的有关的兴趣和偏好或允许用户自己定义和修改用户画像来实现,一般获取的是用户相对静态和稳定的属性,例如:性别、年龄区间、地域、受教育程度、学校、公司等。主题应用商店本身就有比较完整的用户注册引导、用户信息完善任务、认证用户审核等,在收集和清洗用户属性的过程中,需要注意的主要是标签的规范化以及不同来源信息的交叉验证。
    \item 隐式用户行为。隐式方法则是通过跟踪用户的行为和交互来评估和推测用户画像,一般获取的是用户更加动态和易变化的兴趣特征,首先,用户兴趣会受到环境、热点事件、季节等方面的影响,一旦这些因素发生变化,用户的兴趣容易产生迁移;其次,用户的行为多样且碎片化,不同行为反映出来的兴趣差异较大。
    \item 第三方应用数据。一些功能性应用如微信、微博提供的第三方免注册登陆API接口,可以直接获取第三方应用账号提供的用户基本数据。
    \item 自然语言处理技术。利用自然语言处理技术提取用户购买评价、评论语句中的关键词,作为用户画像标签的一部分。
    \end{itemize}

    在个性化服务的用户画像建模中,最常用的方式是将以上几种或多种方法结合起来,通过显式方式来获取静态用户信息如姓名、性别、职业等;通过隐式方式来获取动态用户信息如用户兴趣、爱好等;通过第三方登陆接口获取用户的分享、动态信息等;通过自然语言处理技术分析用户的当前心态、满意度、消费心情等。

    \section{标签权重计算}
    推荐本质上是一种个性化排序,因此在收集到一个用户可能存在的标签后,还需要给标签赋一定的权重,用来区分不同标签对于该用户的重要程度。一个标签对于特定用户的权重值可以大致表示为:标签权重 = (行为类型 + 时空上下文 + 长尾因子) × 时间衰减因子。举例,用户小磊昨天购买了一款win8风格的主题包,计算过程如\autoref{tab:tagweight}所示。
    \begin{table}[htp]
    \centering
    \tabcaption{标签权重计算公式}
    \label{tab:tagweight}
    \begin{tabular}{|c|p{8cm}|} \hline
     标签 & win8风格,比较大众化,长尾因子记为 1 \\ \hline
     时间 & 昨天,衰减因子为 0.95。 \\ \hline
     行为 & 购买行为,记为权重 5 \\ \hline
     上下文 & 用户通过关键字搜索进入,最近几天有多次浏览行为,记为权重 2+2 \\ \hline
     标签权重 &  (5+1+4)*0.95=9.5 \\ \hline
    \end{tabular}
    \end{table}

    其中,用户行为类型一般有浏览、添加购物车、搜索、评论、购买、点击赞、收藏等,不同的行为类型具有不同的权重,如购买权重计为5,浏览计为1。空间上下文是指用户跳转入口方式,如通过搜索入口权重高一些,排行榜入口低一些,时间上下文是指用户之前是否接触过此类标签,接触频率等。长尾因子是指,如果标签本身是一个非常常见的词,那么它用于刻画用户的兴趣的区分性是比较差的,相反如果是一个长尾词,则区分性较强。出于这样的考虑,越是长尾词,标签的权重值会越高。标签的权重也随着时间的流逝而变化,用户的兴趣会发生转移,时间越久远,标签的权重应该相应的下降,距离当前时间越近的兴趣标签应该得到适当突出。出于这样的考虑,一般会在标签权重值上叠加一个时间衰减函数,这个时间衰减函数被设计成如\autoref{pic:hl_timedecay}所示的指数衰减的形式,通过定义衰减幅度和半衰期,调节衰减的程度,体现不同的时效性。此外,针对用户的兴趣,还会设定一个较小的时间窗口来获取用户的短期兴趣,短期兴趣更新周期会较长期兴趣更短,兴趣更集中,但是能够比较及时地反应用户兴趣的变化。
    \begin{figure}
    \centering
      \framebox{\includegraphics[scale=1]{figures/hl_timedecay}}
      \figcaption{时间衰减函数分布}
      \label{pic:hl_timedecay}
    \end{figure}

    \section{用户画像建模方式}
    根据用户在建模过程中的参与程度,用户兴趣建模可以分为用户手工定制建模、示例用户建模和自动用户建模。
    \begin{itemize}
    \item 用户手工定制建模。指用户画像由用户自己手工输入或选择的用户建模方法,如用户手工输入感兴趣信息的关键词列表,或者是选择感兴趣的栏目等。在手机主题市场早期,用户手工定制建模是用户建模的主要方法。用户手工定制建模方法实现简单、效果也不错,但它存在以下三个方面问题:(1)完全依赖于用户,容易降低用户使用系统的积极性。(2)即使用户乐意手工输入用户画像,用户也难以全面,准确的罗列自己感兴趣的栏目或关键词,导致用户标签的质量有好有坏。(3)当用户兴趣发生变化时,用户必须重新输入用户画像,这给用户带来了额外的负担。
    \item 示例用户建模。指由用户提供与自己兴趣相关的示例及其类别属性来建立用户画像的建模方法。由于用户对自己的兴趣和偏好等最有发言权,因而用户提供的有关自己兴趣的示例最能集中、准确地反映用户的兴趣和偏好等特点。示例一般通过要求用户在浏览过程中标注自己的兴趣兴趣度,如喜欢、赞、踩和收藏等。
    \item 自动用户建模.指根据用户的浏览内容和浏览行为自动构建用户画像,自动用户建模由于无需用户主动提供信息,不会显示干扰用户,有利于提高个性化服务系统的亲和度,因此,自动建模技术当前用户画像领域热门研究方向。
    \end{itemize}

    \section{用户画像的维度分析}
    一个用户可以从多个方面去刻画,也就是说用户画像可以从多个维度来考虑和构建。作为虚拟电子商品交易平台,手机主题市场的用户在平台上通过某些行为(点击、浏览、购买)生产或获取信息,也通过其它一些行为(如转发、评论、赞)将信息传播出去,信息的传播是通过用户之间的社交关系所进行的,并且在生产、消费、传播信息的过程中对信息的选择和过滤体现了用户在兴趣方面的倾向性。由此,我们可以将用户画像按照\autoref{pic:hl_userDimension}所示的四个维度进行划分,即属性维度、兴趣维度、社交维度和行为维度。用户属性和用户兴趣是传统用户画像中包含的两个维度。前者刻画用户的静态属性特征,例如用户的身份信息(性别、年龄、受教育程度、学校等),后者则用于刻画用户在信息筛选方面的倾向(例如用户的购买能力、兴趣标签、能力标签等)。社交维度是从社交关系及信息传播的角度来刻画用户的。在社区中用户不在仅仅是一个个体,用户和用户之间的社交关系构成了一张网络,信息在这张网络中高速流动,但是这种流动并不是无差别的,信息的起始点,所经历的关键节点以及这些节点构成的关系圈都是影响信息流动的重要因素。行为维度是一个比较新的研究方向,目的是发现影响用户属性、信息变化的行为因素,分析典型用户群体的行为模式。一方面可以通过行为模式的复用来促进用户在手机主题应用平台的成长;另一方面也有利于平台认识用户,和发现新的或异常的用户行为。    
    \begin{figure}
    \centering
      \framebox{\includegraphics[scale=0.6]{figures/hl_userDimension}}
      \figcaption{用户画像维度划分}
      \label{pic:hl_userDimension}
    \end{figure}

        \subsection{属性维度}
        属性维度属于传统用户画像的范畴,即对用户的信息进行标签化。一方面,标签化是对用户信息进行结构化,方便计算机的识别和处理;另一方面,标签本身也具有准确性和非二义性,也有利于人工的整理、分析和统计。用户属性指相对静态和稳定的人口属性,例如:性别、年龄区间、地域、受教育程度、学校、公司等信息的收集和建立主要依靠产品本身的引导、调查、第三方提供等,在此基础上需要进行补充和交叉验证。
        \begin{itemize}
        \item 标签来源:不是所有的词都适合充当用户标签,这些词本身应该具有区分性和非二义性;此外,还需要考虑来源的全面性,除了用户主动提供的兴趣标签外,用户在使用的过程中的行为,构建的用户关系等也能够反应用户的兴趣,因此也要将其考虑在内。
        \item 权重计算:得到了用户的兴趣标签,还需要针对用户给这些标签进行权重赋值,用来区分不同标签对于该用户的重要程度。
        \end{itemize}

        \subsection{兴趣维度}
        由于用户兴趣维度的重要性,因此有一个独立于用户画像模块的兴趣探索模块,下一章节将会详细介绍到。用户兴趣是更加动态和易变化的特征,首先兴趣受到人群、环境、热点事件、行业等方面的影响,一旦这些因素发生变化,用户的兴趣容易产生迁移;其次,用户的行为多样且碎片化,不同行为反映出来的兴趣差异较大,在用户画像建模的过程中,主要考虑如下几个方面:
        \begin{itemize}
        \item 时效性:随着时间的变化,用户的兴趣会发生转移,有些兴趣会贯穿用户使用社交媒体的全过程,而有些兴趣则是受热点时间、环境因素等的影响。
        \item 长尾性:对于电商领域来讲,那些冷门的用户兴趣的总和可以和那些为数不多的大众化兴趣所占的市场份额相匹配或胜出。
        \item 兴趣和购买意愿的区分:用户具有某方面的兴趣,只代表了他愿意接受这方面的信息,并不能代表他具有购买相关内容的意愿。例如对于一些只看不买的用户,我们认为其购买意愿很小,因此对其会尽可能多的展示免费主题。
        \end{itemize}

        \subsection{社交维度}
        如果将主题应用平台的用户视作节点,用户之间的关系视作节点之间的边,那么这些节点和边将构成一个社交的网络拓扑结构,或称作社交图谱。消费信息就是在这个图谱上进行传播。从社交的维度建立用户画像,需要从不同的角度细致和全面地描述这个消费图谱的特征,反应影响信息传播的各层面上的因素,寻找节点之间的关联度,以及刻画图谱本身的结构特征。其中包括:
        \begin{itemize}
        \item 用户个体对消费信息传播的影响:不同用户在信息传播过程中的重要性不一样,影响大的用户对于信息的传播较影响小的用户更具有促进作用。
        \item 量化用户关系紧密度:存在社交关联的用户,关系越近的用户之间越容易产生相同的消费行为。
        \item 寻找相似的用户:消费中非对等的关系本身可以认为是一种认证,用户基于兴趣、消费态度等原因反应到线上的一种关联。那么在消费维度上的相似用户至少能反应他们在某种因素上的一致性。
        \item 识别关系圈:从关系图谱的本身的结构出发,从中发掘关联紧密的群体,有助于促销广告的精准投放和主题包的推广。以上关于关系建模的任务可以看作是逐步深入的,从“个体”-->“关联”-->“相似”-->“群体”的逐渐深入。
        \end{itemize}

        \subsection{行为维度}
        分析用户的行为,建立行为模式有两个任务:针对典型个体行为进行时序分片,分析用户成长的相关因素;针对典型群体的行为进行统计,为其构建通用的用户画像。
        \begin{itemize}
        \item 典型个体的行为时序分析。所谓典型个体是指某段时间内,成长比较突出的用户。例如从一个新用户从新注册到点击过百、浏览过千需要有一个积累过程,有些用户积累较快,有些较慢,而这些积累较快的用户可以作为典型个体;或者某些用户在某一阶段消费有限,但在某时刻消费激增,无论是消费金额还是数量都变化很大,这种也可以作为典型个体。针对典型个体,需要挖掘与其用户成长相关的行为因素。基本方法是对时间进行分片,获取用户在不同时间片上的行为统计,以及在各个时间分片上的用户成长指标(点击量、购买量、点击转换比等)。在此基础上针对用户行为的统计量的变化,利用关联性分析或回归来分析用户成长与哪些因素有关。
        \item 典型群体行为模式分析。针对典型个体,从用户的基本信息、人口信息、兴趣维度,可以将相似的典型用户划分为同一的群体,称作典型群体,针对典型群体中的用户按照成长程度进行划分,按不同的成长阶段统计用户行为,即建立了该典型群体的行为模型。例如,对于“年龄在20~30岁,女性,付费用户”这样的典型群体,从日点击量、月消费额等维度将其划分到初创、成长、快速提升、成熟等阶段,针对不同成长阶段内的行为组合进行统计,结果构成该群体的行为模式。如\autoref{pic:hl_usergroup}
        \end{itemize}

        \begin{figure}
        \centering
          \framebox{\includegraphics[scale=0.35]{figures/hl_usergroup}}
          \figcaption{手机主题市场用户群体分布}
          \label{pic:hl_usergroup}
        \end{figure}

    \section{用户画像评估方法}
    评价方法分为以下俩种:线下测试和线上测试。首先介绍线下测试基本概念,然后具体介绍工业界常用的线上A/B测试。
      \subsection{线下测试}
      笔者曾参加过2014年阿里举办的大数据竞赛,当时主要用线下测试评估算法模型优劣,总结出的基本准则是:不要过早优化模型调参和模型融合,这两部分应当留到中后期来做;不要一次性添加大量特征,最好一部分一部分添加,这样会对添加的特征效果有个大体的认识。线下测试具体步骤如下:
      \begin{itemize}
      \item 选定数据集选择和应用相关的数据集。数据需要是无偏的(unbiased),通过随机抽样能满足要求。将现有数据集分成训练集(train set) 验证集(validation set) 测试集(test set)。其中训练集用来估计模型,验证集用来确定网络结构或者控制模型复杂程度的参数,而测试集则检验最终选择最优的模型的性能如何。一个典型的划分是训练集占总样本的50%,而其它各占25%。样本少的时候可以留少部分做测试集。然后对其余N个样本采用K折交叉验证法。就是将样本打乱,然后均匀分成K份,轮流选择其中K-1份训练,剩余的一份做验证,计算预测误差平方和,最后把K次的预测误差平方和再做平均作为选择最优模型结构的依据。
      \item 建立算法模型.
      \item 准确度的评估、反馈。准确度的评估方法有Mean Absolute Error和Root Mean Squared Error,对于一个元素是 user-item 对(u, i)的集合 T,实际评分为 $r_{ui}$,预测评分为 $\hat{r}_{ui}$,无论是通过 MAE 还是 RMSE 计算,最终的结果值越小证明结果越准确。但从公式可以看出,RMSE 通过平方扩大了偏离量,同样的两组结果用RMSE得出的差异值将比MAE更大。对应公式如下:
      \begin{equation}
        MAE = \frac{1}{|T |} \sum_{(u,i)\in T}|\hat{r}_{ui}-r_{ui}|
        \label{cosine-similiarity}
      \end{equation}
      \begin{equation}
        RMSE = \sqrt{\frac{1}{|T |} \sum_{(u,i)\in T}(\hat{r}_{ui}-r_{ui})^2}
        \label{cosine-similiarity}
      \end{equation}
      \end{itemize}
      

      \subsection{线上A/B测试}
      在太平洋东部加拉帕戈斯(Galapagos)的一个小岛上有一种名叫达尔文雀的鸟,一部分生活在岛的西部,另一部分生活在岛的东部,由于生活环境的细微不同它们进化出了不同的喙,如\autoref{pic:hl_abbird}所示,这被认为是自然选择学说上的一个重要例证。同样一种鸟,究竟哪一种喙更适合生存呢?自然界给出了她的解决方案,让鸟儿自己变异(设计多个方案),然后优胜劣汰。具体到达尔文雀这个例子上,不同的环境中喙也有不同的解决方案。
      \begin{figure}
      \centering
        \framebox{\includegraphics[scale=0.45]{figures/hl_abbird}}
        \figcaption{达尔文雀}
        \label{pic:hl_abbird}
      \end{figure}
      上面的例子包含了A/B测试最核心的思想:多个方案并行测试;每个方案只有一个变量(比如鸟喙)不同;以某种规则优胜劣汰。评判用户画像模型的效率高低,主要是看该模型带来的点击率、转换率等指标数据,其他统计量见\autoref{tab:abtest}所示。理论上评测推荐系统的指标有用户满意度、预测准确度、覆盖率、多样性、新颖度、惊喜度、信任度、实时性、健壮性等。然而商业开发中,评测推荐结果只看重一个指标:点击转化率。能够提升商业价值,给业务带来更多利益的推荐系统,就是好的推荐系统。
      \begin{table}[htp]
      \centering
      \tabcaption{A/B测试主要评估指标}
      \label{tab:abtest}
      \begin{tabular}{ |c|p{10cm}| } \hline
       指标 & 描述 \\ \hline
       访客数 & 访客数就是指一天之内到底有多少不同的用户访问了你的网站。访客数要比IP数更能真实准确地反映用户数量。\\ \hline
       浏览量 & 即Page View,浏览量和访问次数是呼应的。用户访问网站时每打开一个页面,就记为1个PV。同一个页面被访问多次,浏览量也会累积。 \\ \hline
       点击转化率 & 点击转化率计算公式:点击转化率 = 成交笔数/浏览量 *100\%,成交笔数影响着成交金额,所以点击转化率成为了衡量推荐系统效果的重要数据之一。\\ \hline
       停留时长 & 停留时长是用户访问网站的平均停留时间,是衡量网站用户体验的一个重要指标。如果用户不喜欢主题包的内容,可能稍微看一眼就关闭页面了,那么停留时长就很短;如果用户对页面的内容很感兴趣,停留时长就很长。\\ \hline
       跳出率 & 跳出率是指访客来到网站后,只访问了一个页面就离开网站的访问次数占总访问次数的百分比,跳出率越低说明流量质量越好,用户对网站的内容越感兴趣。 \\ \hline
       其他指标 & 各种辅助性指标如点击量/用户,购买量/用户,下载量/用户等。\\ \hline
      \end{tabular}
      \end{table}

      A/B测试对用户画像建模的作用有三个:特征提取,一些标签对用户的兴趣有强相关作用,如性别标签,有些标签是弱相关作用,如用户职业标签,A/B测试需要筛选出强相关标签,过滤掉弱相关标签;权重量化,根据A/B测试实验显示,发现用户画像中的最近点击标签、最近关注标签所占权重比想象中的要大;标签组合,有些标签是冗余的,只需从中选一即可。A/B测试具体实现步骤如下:

      \begin{itemize}
      \item 方案设计。实验之前需得到一个基准版本,然后把又争议的标签按照优先度列举出来决定是否实验。真正的A/B测试只应一次改动一个地方,这意味着标签选择、权重量化、标签组合要分开来测试。
      \item 确定数据评估方案。根据实验内容不同评估它们好坏的标准也不同,如果是标签选择那么衡量的主要指标是点击量,如果是权重量化那么衡量的主要指标是点击转换率。
      \item 流量分配。为了试实验所得数据具备统计意义,能准确反映用户的真实行为,需要对流量设置一个下限。除此之外,为了使各个方案具有可比性, A、B俩个方案的流量必须是相等的。
      \item 测试周期。根据所需测试的项目的不同测试周期也有所不同,如添加一个地理标签需要的测试周期以天为单位,如果涉及到多个标签的权重变动则需要测试周期以周为单位。
      \item 评估结果。适者胜出,其代表的数据作为下一轮回A/B测试的基准版本。
      \item 建立通用的数据评估题型。在经过各种类型A/B测试实验后,已经积累很多的评估指标,有必要把这些指标抽象出来形成一个通用的数据评估模型,减少以后实验的重复设计评估指标的时间。
      \end{itemize}

      \section{总结}
      用户画像对于推荐系统来讲,主要如下几个方面的提升:提升推荐系统的精度。对于点击转化率指标,融合了用户画像的推荐模型比单纯的推荐模型提高了约2.8\%,考虑到300万用户的基数,2.8\%的提升是一个很大的进步;解决新用户的冷启动问题。对于一个新注册用户来讲,推荐系统可以利用用户画像的静态信息,然后结合商品信息进行推荐;提高推荐系统的时效性。对用户行为的离线预处理,可以节约推荐系统的大部分计算时间;增强推荐结果的长尾效应,主要通过用户画像的兴趣探索实现,用户兴趣探索模块将在下一章节件详细介绍。

  %自行添加
  
\chapter{用户兴趣探索}
\label{chap:interestExplore}
电子商务产品的设计往往是数据驱动的,即许多产品方面的决策都是把用户行为量化后得出的。但就商品而言,那些热门主题往往只代表了用户一小部分的个性化需求,只有通过对用户行为的充分分析,才能更好的挖掘出用户的兴趣,最终提升商品的销售量。现有的推荐算法注重用户或资源间的相似性的同时却忽略了用户兴趣的动态变化,从而导致系统在时间维度上有偏离用户需求的趋势。

为了更好的探索用户兴趣,手机主题推荐系统充分利用了用户画像和商品特征表。用户画像包括基本信息和兴趣特征向量,商品特征向量表包括分类、标签、适用人群等,给定某用户行为,用户兴趣探索过程分为如下几个步骤:首先,利用用户历史行为(评论,停留时长,评分,点赞,购买等)建模量化用户满意度,然后,利用用户兴趣特征向量与商品特征向量得出相关分数,如果商品与用户的相关分数很低,但有很高的用户满意度,说明是一次成功的用户兴趣探索,更新用户画像。如果是热门商品,大量的用户都会点击,但商品与用户不是很相关,则认为其探索效果是有限的,反之如果是小众商品,考虑到长尾效应,则可以认为其是更成功的兴趣探索。这里涉及到的关键概念包括用户满意度的量化、小众标签的定向挖掘、用户兴趣的动态化。

本章内容首先介绍海量用户行为数据的存储方式。用户行为数据拥有区别于传统数据库数据的特点有,用户行为数据量巨大,常面临TB甚至PB级的数据;含有较多的噪音;多维聚合式查询。针对这些特征,用户行为数据采用Hbase数据集群存储和hadoop集群计算。然后,介绍用户兴趣探索的算法模型;然后,介绍如何通过用户行为的分析量化用户满意度。最后,介绍小众兴趣标签的挖掘。

\section{用户行为数据的存储}
手机主题用户行为数据的特点包括:用户基数庞大。手机主题注册用户达千万级,活跃用户达百万级;用户规模增长快。月新注册用户达10万数量级。每个用户的行为数量较小。即使是活跃用户,每天最多也只能产生上百条行为记录,每年不超过十万条;用户行为的计算较为复杂。计算用户的两次登录间隔天数、反复购买的商品、累积在线时间,这些都是针对用户行为的计算,通常具有一定的复杂性;用户行为数据格式不规整,字段丢失率较高。根据用户行为数据的这些特点,我们采用基于Hadop分布式的架构。Hadop又如下几个优点:
\begin{itemize}
\item 高可靠性。Hadoop按位存储和处理数据的能力使其具有高可靠性。
\item 高扩展性。Hadoop是在可用的计算机集簇间分配数据并完成计算任务的,这些集簇可以根据用户增长规模方便地扩展到数以千计的节点中。
\item 高容错性。Hadoop能够自动保存数据的多个副本,并且能够自动将失败的任务重新分配。
\item 高效性。Hadoop能够在节点之间动态地移动数据,并保证各个节点的动态平衡,因此处理速度非常快。
\item 低成本。hadoop是开源的,项目的软件成本因此会大大降低。
\end{itemize}
Hadoop的框架最核心的设计是HDFS和MapReduce。HDFS为海量的数据提供了存储,则MapReduce为海量的数据提供了计算。接下来首先介绍HDFS的体系架构,然后介绍MapReduce。
  \subsection{HDFS的体系架构}
  HDFS采用主从(Master/Slave)结构模型,一个HDFS集群是由一个NameNode和若干个DataNode组成的(在最新的Hadoop2.2版本已经实现多个NameNode的配置)。NameNode作为主服务器,管理文件系统命名空间和客户端对文件的访问操作。DataNode管理存储的数据。HDFS支持文件形式的数据。从内部来看,文件被分成若干个数据块,这若干个数据块存放在一组DataNode上。NameNode执行文件系统的命名空间,如打开、关闭、重命名文件或目录等,也负责数据块到具体DataNode的映射。DataNode负责处理文件系统客户端的文件读写,并在NameNode的统一调度下进行数据库的创建、删除和复制工作。NameNode是所有HDFS元数据的管理者,用户数据永远不会经过NameNode,HDFS体系结构图如\autoref{pic:hl_hadoop}所示。
  \begin{figure}
  \centering
    \framebox{\includegraphics[scale=0.4]{figures/hl_hadoop}}
    \figcaption{HDFS体系结构}
    \label{pic:hl_hadoop}
  \end{figure}

  其中,NameNode、DataNode、Client。NameNode是管理者,DataNode是文件存储者、Client是需要获取分布式文件系统的应用程序。HDFS作为分布式文件系统在数据管理方面设计了多重容错冗余:一个Block会有三份备份,一份在NameNode指定的DateNode上,一份放在与指定的DataNode不在同一台机器的DataNode上,一根在于指定的DataNode在同一Rack上的DataNode上。备份的目的是为了数据安全,采用这种方式是为了考虑到同一Rank失败的情况,以及不同数据拷贝带来的性能的问题。
  \begin{itemize}
  \item 文件写入:首先,Client向NameNode发起文件写入的请求;然后,NameNode根据文件大小和文件块配置情况,返回给Client它管理的DataNode的信息;最后,Client将文件划分为多个block,根据DataNode的地址,按顺序将block写入DataNode块中。
  \item 文件读取:首先,Client向NameNode发起读取文件的请求;然后,NameNode返回文件存储的DataNode信息;最后,Client读取文件信息。
  \end{itemize}

  \subsection{MapReduce体系架构}
  MR框架是由一个单独运行在主节点上的JobTracker和运行在每个集群从节点上的TaskTracker共同组成。主节点负责调度构成一个作业的所有任务,这些任务分布在不同的不同的从节点上。主节点监视它们的执行情况,并重新执行之前失败的任务。从节点仅负责由主节点指派的任务。当一个Job被提交时,JobTracker接受到提交作业和配置信息之后,就会将配置信息等分发给从节点,同时调度任务并监控TaskTracker的执行。JobTracker可以运行于集群中的任意一台计算机上。TaskTracker负责执行任务,它必须运行在DataNode上,DataNode既是数据存储节点,也是计算节点。JobTracker将map任务和reduce任务分发给空闲的TaskTracker,这些任务并行运行,并监控任务运行的情况。如果JobTracker出了故障,JobTracker会把任务转交给另一个空闲的TaskTracker重新运行。

  HDFS和MR共同组成Hadoop分布式系统体系结构的核心。HDFS在集群上实现了分布式文件系统,MR在集群上实现了分布式计算和任务处理。HDFS在MR任务处理过程中提供了文件操作和存储等支持,MR在HDFS的基础上实现了任务的分发、跟踪、执行等工作,并收集结果,二者相互作用,完成分布式集群的主要任务。Hadoop上的并行应用程序开发是基于MR编程框架。MR编程模型原理:利用一个输入的key-value对集合来产生一个输出的key-value对集合。MR库通过Map和Reduce两个函数来实现这个框架。用户自定义的map函数接受一个输入的key-value对,然后产生一个中间的key-value对的集合。MR把所有具有相同的key值的value结合在一起,然后传递个reduce函数。Reduce函数接受key和相关的value结合,reduce函数合并这些value值,形成一个较小的value集合。通常我们通过一个迭代器把中间的value值提供给reduce函数,这样就可以处理无法全部放在内存中的大量的value值集合了。

  MapReduce体系中数据流动过程:首先,大数据集被分成众多小的数据集块,若干个数据集被分在集群中的一个节点进行处理并产生中间结果。然后,单节点上的任务,map函数一行行读取数据获得数据的(k1,v1),数据进入缓存,通过map函数执行map(基于key-value)排序执行后输入(k2,v2),有时候在map之后reduce之前有一个数据合并(Combine)操作,即将中间有相同的key的对合并,Combine能减少中间结果key-value对的数目,从而降低网络流量。最后,不同机器上的(k2,v2)通过merge排序的过程,reduce合并得到,(k3,v3),输出到HDFS文件中。数据流示意图如\autoref{pic:hl_MapReduce}所示。值得一提的是,Map任务的中间结果在做完Combine和Partition后,以文件的形式存于本地磁盘上。中间结果文件的位置会通知主控JobTracker,JobTracker再通知reduce任务到哪一个DataNode上去取中间结果。所有的map任务产生的中间结果均按其key值按hash函数划分成R份,R个reduce任务各自负责一段key区间。每个reduce需要向许多个map任务节点取的落在其负责的key区间内的中间结果,然后执行reduce函数,最后形成一个最终结果。有R个reduce任务,就会有R个最终结果,很多情况下这R个最终结果并不需要合并成一个最终结果,因为这R个最终结果可以作为另一个计算任务的输入,开始另一个并行计算任务。这就形成了多个输出数据片段(HDFS副本)。
  \begin{figure}
  \centering
    \framebox{\includegraphics[scale=0.5]{figures/hl_MapReduce}}
    \figcaption{MapReduce数据流}
    \label{pic:hl_MapReduce}
  \end{figure}

  \subsection{Hbase数据管理}
  Hbase作为Hadoop数据仓库。与传统的mysql、oracle还是有很大的差别。NoSql数据库与传统关系型数据的区别有如下几个方面:
  \begin{itemize}
  \item Hbase适合大量插入同时又有读的情况。输入一个Key获取一个value或输入一些key获得一些value。
  \item Hbase的瓶颈是硬盘传输速度。Hbase的操作包括往数据里面insert数据,update的实际上也是insert,只是插入一个新的时间戳的一行。Delete数据也是insert,只是insert一行带有delete标记的一行。Hbase的所有操作都是追加插入操作。Hbase是一种日志集数据库。存储方式像是日志文件一样,是批量大量的往硬盘中写,通常都是以文件形式的读写。所以读写速度就取决于硬盘与机器之间的传输有多快。而Oracle的瓶颈是硬盘寻道时间。其经常的操作时随机读写。要update一个数据,先要在硬盘中找到这个block,然后将其读入内存,在内存中的缓存中修改,过段时间再回写回去。硬盘的寻道时间主要由转速来决定的,所以形成了寻道时间瓶颈。
  \item Hbase中数据可以保存许多不同时间戳的版本。数据按时间排序,因此Hbase特别适合寻找按照时间排序寻找Top n的场景。找出某个人最近浏览的主题,最近购买的N款主题包,N种行为等等,因此Hbase在互联网应用非常多。
  \item Hbase的局限。只能做很简单的Key-value查询。它适合有高速插入,同时又有大量读的操作场景。而这种场景又很极端,并不是每一个公司都有这种需求。在一些公司,就是普通的OLTP(联机事务处理)随机读写。在这种情况下,Oracle的可靠性,系统的负责程度又比Hbase低一些。而且Hbase局限还在于它只有主键索引,因此在建模的时候就遇到了问题。
  \item Oracle是行式数据库,而Hbase是列式数据库。列式数据库的优势在于数据分析这种场景。数据分析与传统的OLTP的区别。数据分析,经常是以某个列作为查询条件,返回的结果也经常是某一些列,不是全部的列。在这种情况下,行式数据库反应的性能就很低效。
  \end{itemize}

  \subsection{Hive数据管理}
  Hive是建立在Hadoop上的数据仓库基础架构。它提供了一系列的工具,用来进行数据提取、转换、加载,是一种可以存储、查询和分析存储在Hadoop中的大规模数据机制。可以把Hadoop下结构化数据文件映射为一张成Hive中的表,并提供类sql查询功能,除了不支持更新、索引和事务,sql其它功能都支持。可以将sql语句转换为MapReduce任务进行运行,作为sql到MapReduce的映射器。提供shell、JDBC/ODBC、Thrift等接口。优点是成本低可以通过类sql语句快速实现简单的MapReduce统计。作为一个数据仓库,Hive的数据管理按照使用层次可以从元数据存储、数据存储和数据交换三个方面介绍。
  \begin{itemize}
  \item 元数据存储。Hive将元数据存储在RDBMS中,有三种方式可以连接到数据库:1)内嵌模式:元数据保持在内嵌数据库的Derby,一般用于单元测试,只允许一个会话连接;2)多用户模式:在本地安装Mysql,把元数据放到Mysql内;3)远程模式:元数据放置在远程的Mysql数据库。
  \item 数据存储。首先,Hive没有专门的数据存储格式,也没有为数据建立索引,用于可以非常自由的组织Hive中的表,只需要在创建表的时候告诉Hive数据中的列分隔符和行分隔符,这就可以解析数据了。其次,Hive中所有的数据都存储在HDFS中,Hive中包含4中数据模型:Tabel、ExternalTable、Partition、Bucket。Table类似与传统数据库中的Table,每一个Table在Hive中都有一个相应的目录来存储数据。例如:一个表zz,它在HDFS中的路径为:/wh/zz,其中wh是在hive-site.xml中由用户设定的数据仓库的目录,所有的Table数据(不含External Table)都保存在这个目录中。Partition类似于传统数据库中划分列的索引。在Hive中,表中的一个Partition对应于表下的一个目录,所有的Partition数据都存储在对应的目录中。例如:zz表中包含ds和city两个Partition,则对应于ds=20140214,city=beijing的HDFS子目录为:/wh/zz/ds=20140214/city=Beijing;Buckets对指定列计算的hash,根据hash值切分数据,目的是为了便于并行,每一个Buckets对应一个文件。将user列分数至32个Bucket上,首先对user列的值计算hash,比如,对应hash=0的HDFS目录为:/wh/zz/ds=20140214/city=Beijing/part-00000;对应hash=20的,目录为:/wh/zz/ds=20140214/city=Beijing/part-00020。ExternalTable指向已存在HDFS中的数据,可创建Partition。和Table在元数据组织结构相同,在实际存储上有较大差异。Table创建和数据加载过程,可以用统一语句实现,实际数据被转移到数据仓库目录中,之后对数据的访问将会直接在数据仓库的目录中完成。删除表时,表中的数据和元数据都会删除。ExternalTable只有一个过程,因为加载数据和创建表是同时完成。世界数据是存储在Location后面指定的HDFS路径中的,并不会移动到数据仓库中。
  \item 数据交换。用户接口包括客户端、Web界面和数据库接口,元数据通常存储在关系数据库如Mysql,Derby中。
  \end{itemize}
本小节主要介绍了Hadoop分布式计算平台最核心的分布式文件系统HDFS、MapReduce处理过程,以及数据仓库工具Hive和分布式数据库Hbase,基本涵盖了Hadoop分布式平台的所有技术核心。从体系架构到数据定义到数据存储再到数据处理,Hadoop分布式存储、计算平台为海量用户行为的分析和用户兴趣探索提供了可能。接下来的章节先介绍用户行为数据的分析,包括数据预处理和异常数据监测,然后介绍用户兴趣探索模块,包括算法模型、用户满意度量化、小众兴趣标签的挖掘。

\section{用户行为数据的的预处理}
数据预处理是数据挖掘过程中一个重要步骤,当原始数据存在不一致、重复、含噪声、维度高等问题时,更需要进行数据的预处理,以提高数据挖掘对象的质量,最终达到提高数据挖掘所获模式知识质量的目的。
  \subsection{背景}
  随着手机主题市场交易规模的逐步增大,积累下来的业务数据和用户行为数据越来越多,这些用户数据往往是电子商务平台最宝贵的财富。目前在手机主题推荐系统中大量地应用到了机器学习和数据挖掘技术,例如个性化推荐、搜索排序、用户画像建模等等,为企业创造了巨大的价值。本节主要介绍在用户兴趣探索实践中的数据预处理与特征挖掘方法。数据预处理主要工作是:
  \begin{itemize}
  \item 从原始数据,如文本、图像或者应用数据中清洗出特征数据和标注数据
  \item 对清洗出的特征和标注数据进行处理,例如样本采样,样本调权,异常点去除,特征归一化处理等过程。最终生成的数据主要是供模型直接使用。
  \end{itemize}

  \subsection{特征提取}
  用户兴趣探索的任务包括:探索用户的兴趣广度、兴趣深度、兴趣变动趋势。依据这些信息,推荐系统就能知道在面对某一个用户时要推荐哪几类型商品,每类商品所占的比例,未来几天推荐内容会有哪些变化。在确定了目标之后,接下来需要确定使用哪些数据来达到目标。提取哪些特征数据可能与用户是否点击购买相关,一方面可以借鉴一些业务经验,另一方面可以采用一些特征选择、特征分析等方法。从业务经验来判断,可能影响用户是否点击下单的因素有:
  \begin{itemize}
  \item 用户历史行为。对于老用户,之前可能有过点击、购买等行为。
  \item 用户实时兴趣。
  \item 用户满意度。上面的特征都是比较好衡量的,用户满意度可能是更复杂的一个特征,具体体现在用户评分、评价、购买后使用频率、时长等。
  \item 是否热门,商品评价人数,购买数等。
  \end{itemize}
  在确定好要使用哪些数据之后,还需要对使用数据的可用性进行评估,包括数据的获取难度,数据的规模,数据的准确率,数据的覆盖率等。
  \begin{itemize}
  \item 用户历史行为。只有老用户才会有行为,新用户是没有的。
  \item 数据获取难度。获取用户id不难,但是获取用户年龄和性别较困难,因为用户注册或者购买时,这些并不是必填项,即使填了也不完全准确。如果一些特征需要通过其他预测模型交叉验证的话,就存在着模型精度的问题。
  \item 数据覆盖率。数据覆盖率也是一个重要的考量因素,例如地理位置特征,并不是所有用户的距离我们都能获取到,PC端的就没有地理位置,还有很多用户禁止使用它们的定位功能。
  \item 用户实时行为。如果用户刚打开app,还没有任何行为,同样面临着一个冷启动的问题。
  \item 数据的准确率。有时候用户购买一款主题,不一定是其真心喜欢,可能是因为遇到限时半价、购买返现等活动。
  \end{itemize}

  \subsection{特征获取方式}
  特征提取方式分为在线提取和离线提取。

  离线特征获取方案。离线可以使用海量的数据,借助于分布式文件存储平台,例如HDFS等,使用例如MapReduce,Spark等处理工具来处理海量的数据等。

  在线特征获取方案。在线特征比较注重获取数据的延时,由于是在线服务,需要在非常短的时间内获取到相应的数据,对查找性能要求非常高,可以将数据存储在索引、key-value存储等,也可以使用Kafka等处理工具。Kafka是一种分布式的,基于发布/订阅的消息系统。主要设计目标如下:1)以时间复杂度为O(1)的方式提供消息持久化能力,即使对TB级以上数据也能保证常数时间的访问性能;2)高吞吐率。即使在非常廉价的商用机器上也能做到单机支持每秒100K条消息的传输;3)同时支持离线数据处理和实时数据处理;4)分布式系统,易于向外扩展。所有的producer、broker和consumer都会有多个,均为分布式的。无需停机即可扩展机器。

  \subsection{用户行为数据预处理}
  根据不同业务数据的预处理方式也不同,一般来讲原始服务器日志数据脏数据的形成原因包括:缩写词不统一,数据输入错误,不同的惯用语,重复记录,丢失值,不同的计量单位,过时的编码等。相应的,数据预处理内容包括数据清理、数据集成、数据变换、数据归约、数据离散化。

  1)数据清理包括格式标准化、异常数据清除、错误纠正、重复数据的清除。对于手机主题用户数据来讲,引起空缺值的原因主要是用户设备异常造成的,有些时候是因为与其他已有数据不一致而被删除或数据的改变没有进行日志记载。根据数据空缺情况的不同有不同的处理方式:
  \begin{itemize}
  \item 忽略元组。当一个记录中有多个属性值空缺、特别是关键信息丢失时,已不能反映真实情况,它的效果非常差。
  \item 去掉属性。缺失严重时,已无挖掘意义。
  \item 人工填写空缺值。但是工作量大且可行性低。
  \item 默认值。比如使用unknown或-∞。
  \item 使用属性的平均值填充空缺值。
  \item 预测最可能的值填充空缺值。使用贝叶斯公式或判定树这样的基于推断的方法。
  \end{itemize}

  2)数据集成就是将多个数据源中的数据整合到一个一致的存储中,需要注意以下几个情况:
  \begin{itemize}
  \item 模式集成。整合不同数据源中的元数据时的实体识别问题,比如匹配俩个表中的用户ID,A.custId=B.customerNo。
  \item 检测/解决数值冲突。对现实世界中的同一实体,来自不同数据源的属性值可能有所不同,如同表示停留时长,A表单位是秒,B表单位为毫秒。
  \item 多表之间的数据冗余。同一属性在不同的数据库中会有不同的字段名,有些时候冗余可以被相关分析检测出来,计算公式如所示,其中$\overline{A}$和$\overline{B}$表示为字段A和B的平均值,$\sigma_A和\sigma_B$表示其的标准差。仔细将多个数据源中的数据集成起来,能够减少或避免结果数据中的冗余与不一致性,从而可以提高挖掘的速度和质量。
  \begin{equation}
    r_{A,B} = \frac{\sum (A-\overline{A})(B-\overline{B})}{(n-1)\sigma_A\sigma_B}
    \label{F-Measure}
  \end{equation}
  \end{itemize}

  3)数据变换包括数据的平滑变换、数据聚集和数据规范化。所谓规范化是指将数据按比例缩放,使之落入一个小的特定区间,有如下几种方式:
  \begin{itemize}
  \item 最小-最大规范化。如\autoref{equ:minMax}所示,原始数值范围为[min,max],通过公式映射到新区间[newMin,newMax],$v'$表示属性v的公式映射。
  \begin{equation}
    v' = \frac{v-min}{max-min}(newMax-newMin)+newMin
    \label{equ:minMax}
  \end{equation}
  \item z-score规范化。Z-score表示原始数据偏离均值的距离长短,而该距离度量的标准是标准方差,如果统计数据量足够多,Z-score数据分布可以满足,68\%的数据分布在“-1”与“1”之间,95\%的数据分布在“-2”与“2”之间,99\%的数据分布在“-3”与“3之间”。如\autoref{equ:zscore}所示,其中v是原始数据,v'为v的映射,mean是全部数据的均值,$\sigma$为标准方差。
  \begin{equation}
    v' = \frac{v-mean}{\sigma}
    \label{equ:zscore}
  \end{equation}
  \item 小数定标规范化,小数定标规范化通过移动数据A的小数点位置进行规范化,小数点的移动位置依赖数据A的最大值。如\autoref{equ:small}所示,其中j是使$Max(| v'|)<1$的最小整数。
  \begin{equation}
    v' = \frac{v}{10^j}
    \label{equ:small}
  \end{equation}
  \end{itemize} 

  4)数据归约是数据字段从源数据集中得到数据集的归约表示。数据仓库中往往存有海量数据,在其上进行复杂的数据分析与挖掘需要很长的时间,通过数据归约使得数据小得多,且可以产生相同的分析结果,需要注意的是用于数据归约的时间不应当超过或“抵消”在归约后的数据上挖掘节省的时间。数据归约策略包括:
  \begin{itemize}
  \item 数据立方体聚集。最底层的方体对应于基本方体,基本方体对应于感兴趣的实体。在数据立方体中存在着不同级别的汇总,每次较高层次的抽象将进一步减少结果数据。数据立方体提供了对预计算的汇总数据的快速访问,所有尽可能对于汇总数据的查询使用数据立方体。
  \item 维归约,通过删除不相干的属性或维减少数据量,维归约又属性子集选择和启发式俩种实现方式。属性子集选择是指找出最小属性集,使得数据类的概率分布尽可能的接近使用所有属性的原分布,同时减少出现在发现模式上的属性的数目,使得模式更易于理解。启发式的方法有:逐步向前选择;逐步向后删除;向前选择和向后删除相结合;判定归纳树;基于统计分析的归约如主成分分析、回归分析等。
  \end{itemize}

  6)数据离散化,即连续属性的范围划分为区间,减少给定连续属性值的个数,区间的标号可以代替实际的数据值。
  \begin{itemize}
  \item 概念分层。通过使用高层的概念替代底层的属性值,如用青年、中年、 老年代替年龄数据值。
  \item 分箱(binning)。分箱技术递归的用于结果划分,可以产生概念分层。
  \item 直方图分析(histogram)。直方图分析方法递归的应用于每一部分,可以 自动产生多级概念分层。
  \item 聚类分析。将数据划分成簇,每个簇形成同一个概念层上的一个节点,每个簇可再分成多个子簇,形成子节点。
  \item 基于熵的离散化。
  \item 自然划分。将数值区域划分为相对一致的、易于阅读的、看上去更直观或自然的区间,划分步骤:如果一个区间最高有效位上包含3,6,7或9个不同的值,就将该区间划分为3个等宽子区间;如果一个区间最高有效位上包含2,4,或8个不 同的值,就将该区间划分为4个等宽子区间;如果一个区间最高有效位上包含1,5,或10个 不同的值,就将该区间划分为5个等宽子区间;将该规则递归的应用于每个子区间,产生给定数值属性的概念分层;对于数据集中出现的最大值和最小值的极端分布,为了避免上述方法出现的结果扭曲,可以在顶层分段时,选用一个大部分的概率空间,如5\%-95\%。
  \end{itemize}

\section{用户兴趣探索的算法模型}
用户兴趣探索就是不断学习用户所感兴趣的内容反馈给个性化推荐模型去加强推送相关内容,本节首先介绍用户兴趣模型的基本概念,然后介绍算法模型的组成结构:用户异常兴趣探测,用户小众兴趣标签的挖掘和用户满意度量化,用户兴趣衰减算法。

  \subsection{基本概念概述}
  实体域。当我们想基于用户行为分析来建立用户兴趣模型时,我们必须把用户行为和兴趣主题限定在一个实体域上。个性化推荐落实在具体的推荐中都是在某个实体域的推荐。对于手机主题应用市场来说,实体域包括所有的主题,背景图片,铃声,闹铃等。

  用户行为。浏览,点击,下载,试用,购买,评论等都可是用户行为。本文所指的用户行为都是指用户在某实体域上的行为。比如用户在手机铃声产生的行为。
  用户兴趣。用户的兴趣维度,同样是限定在某实体域的兴趣,通常以标签+权重的形式来表示。比如,对于手机主题,用户兴趣向量可以是「动漫,0.6」,「体育,0.1」,「情感,0.7」等分类标签。值得一提的是,用户兴趣只是从用户行为中抽象出来的兴趣维度,并无统一标准。而兴趣维度的粒度也不固定,如「体育」,「电影」等一级分类,而体育下有「篮球」,「足球」等二级分类,篮球下有「NBA」,「CBA」,「火箭队」等三级分类。我们选取什么粒度的兴趣空间取决于具体业务模型。

  兴趣空间。在同一层次上兴趣维度的集合,比如手机主题中,可以用「热门」,「游戏」,「限时特价」,「科技」来构成一个程序员兴趣标签空间,也可以用「二次元」,「萝莉」,「魔幻」,「纯真」,「召唤兽」·····「法术」等构成一个动漫兴趣标签空间。


  \subsection{用户异常兴趣探测算法}
  统计学中的数据异常值是一个测量变量中的随机错误或偏差,信息安全学中的数据异常是指引起不正确属性值的原因包括恶意hack行为、数据输入等。传统的的异常兴趣检测是基于统计。首先根据现有用户画像建立数据统计模型,异常是那些模型不能完美拟合或是相对远离预测值的对象,对于常用的回归模型,如\autoref{pic:hl_regression}所示。但缺点是用户兴趣概率分布模型计算比较耗费计算资源。
  \begin{figure}
  \centering
    \framebox{\includegraphics[scale=0.5]{figures/hl_regression}}
    \figcaption{回归异常值检测}
    \label{pic:hl_regression}
  \end{figure}

  本文中涉及到的异常是指用户兴趣行为的差异化趋势,比如,用户小磊每次都会浏览动漫、美少女主题,但是有一天却购买了一款汽车手机主题,那么程序可以检测到这个异常情况,然后将汽车标签更新到用户画像中,并作为个性化推荐的依据。事实上用户兴趣迭代过程可以在很短的时间内完成,基于 hive + MapReduce 平台的时长维度为天,而基于 kafka + spark 平台可以将时长维度降到小时级别。用户异常兴趣检测算法要从用户的行为和偏好中发现新的兴趣标签,并基于此给予推荐,工作内容包括收集用户的最新的行为数据是并分析得出异常标签,算法如\autoref{algo:newTagsExpore}所示
  \begin{algorithm}
  %% \SetLine
  \KwIn{用户画像数据 userProfile ,用户显示、隐式行为数据 logUsers}
  \KwOut{用户异常兴趣标签 newUsersTags}
  init newUsersTags;\\
  \For{$(user_i in logUsers)$}{
    \For{$(tag_j in user_i)$}{
      \If{$(tag_j.weight==0)$}{
        //若标签权重已经为 0,该用户兴趣标签将被删除\\
        remove $tag_j$;
      }
      \Else {
        //成功探测到新用户兴趣标签\\
        temp.get($user_i$).set($tag_j$);
      }
    } 
  }
  \Return temp;
  \caption{用户异常兴趣探测}
  \label{algo:newTagsExpore}
  \end{algorithm}

  \subsection{长尾标签抽取算法}
  标签集中度(tagFocus)是指,如果某个标签在一类主题中出现的频率高,其他主题类型很少出现,则认为此兴趣标签具有很好的类别区分能力。这是因为包含兴趣标签t的主题越少,也就是n越小,则说明标签t具有很好的兴趣区分,则其探索权重越大。如果某一类主题包C中包含兴趣标签t的个数为tagInThemeNum,而其它类包含t的总数为tagInOtherNum,则所有包含t的主题数n=tagInThemeNum+tagInOtherNum,当m大的时候,n也大,标签权重值会小,就说明该标签t类别区分能力不强。实际上,如果一个标签在一个类的主题中频繁出现,则说明该标签能够很好代表这类主题的特征,这样的标签应该给它们赋予较高的权重,并选来作为该类主题的特征向量以区别于其它类主题。热度(tagPopular)指的是某一个给定标签在用户画像中出现的频率。例如在300万用户总数中,十分之一的用户标签中有"火影"标签,那么其热度为0.1,除此之外有些标签如"精品","气质"等标签占了总词频的80\%以上,而它对区分主题类型几乎没有用。我们称这种词叫“应删标签”。即应删除词的权重应该是零,也就是说在度量相关性是不应考虑它们的频率。标签集中度公式如\autoref{equ:focus}所示,我们很容易发现,如果一个标签只在很少的主题包中出现,我们通过它就容易锁定搜索目标,它的权重也就应该大。反之如果一个词在大量主题包中出现,我们看到它仍然不很清楚要找什么内容,因此它应该小。热度公式如\autoref{equ:hot}所示。长尾标签抽取算法如\autoref{algo:longTailTags}所示。
    \begin{equation}
      tagFocus=log\frac{|tagInThemeNum|}{|tagInThemeNum+tagInOtherNum|}
      \label{equ:focus}
    \end{equation}

    \begin{equation}
      tagPopular=log\frac{|peopleLikeTagNum|}{|allPeople|}
      \label{equ:hot}
    \end{equation}

  \begin{algorithm}
  %% \SetLine
  \KwIn{用户画像数据 userProfile ,用户显示、隐式行为数据 logUsers}
  \KwOut{长尾兴趣标签 longTailTags}
  init longTailTags;\\
  \For{$(user_i in logUsers)$}{
    \For{$(tag_j in user_i)$}{
      $weight_{ij}$=$tag_j$.tagFocus/$tag_j$.tagPopular;\\
      \If{$(weight_{ij}<=threshold)$}{
        //若标签权重小于阈值,该用户兴趣标签将被删除\\
        remove $tag_j$;
      }
      \Else {
        //成功探测到新用户兴趣标签\\
        longTailTags.get($user_i$).set($tag_j$);
      }
    } 
  }
  \Return longTailTags;
  \caption{长尾兴趣探测}
  \label{algo:longTailTags}
  \end{algorithm}

  \subsection{用户满意度量化算法}
  要从用户的行为和偏好中量化用户满意度,并基于此实现兴趣标签探索,如何收集用户的偏好行为成为用户兴趣探索效果最基础的决定因素。用户有很多方式向系统提供自己的偏好信息,而且不同的应用也可能大不相同。
    \begin{table}[htp]
    \centering
    \tabcaption{用户行为和其权重}
    \label{tab:userAction}
    \begin{tabular}{ |c|c|p{4cm}|p{5cm}|c|} \hline
     用户行为 & 类型 & 特征 & 作用 & 权重\\ \hline
     评分 & 显式 & 整数量化的偏好,可能的取值是 [0, 5] & 通过用户对物品的评分,可以精确的得到用户的满意度,但是噪声比较大,比如遇到好评返现活动 & 1\\ \hline
     分享 & 显式 & 布尔量化的偏好,取值是 0 或 1 & 通过用户对物品的投票,可以精确的得到用户的喜好度,同时可以推理得到被转发人的兴趣取向(不太精确)& 2\\ \hline
     评论 & 显式 & 一段文字,需要进行文本分析,得到偏好 & 通过分析用户的评论,可以得到用户的情感:喜欢还是讨厌 & 1\\ \hline
     赞/踩 & 显示 & 布尔量化的偏好,取值是 0 或 1 & 带有很强的个人喜好度 & 3 \\ \hline
     购买、试用 & 显式 & 布尔量化的偏好,取值是 0 或 1 & 用户的购买是很明确的说明这个项目它感兴趣。& 3 \\ \hline
     点击流 & 隐式 & 包括滑屏频率,滑屏次数,屏停留时长,用户对物品感兴趣,需要进行分析,得到偏好 & 用户的点击一定程度上反映了用户的注意力,所以它也可以从一定程度上反映用户的喜好。& 1 \\ \hline
     停留时长 & 隐式 & 一组时间信息,噪音大,需 要进行去噪,分析,得到偏 好 & 用户的页面停留时间一定程度上反映了用户的注意力和喜好,但噪音偏大,不好利用。比如说用户在浏览一个主题的时候,丢下手机和同学出去踢球去了,页面提留时长可能会很长 & 1 \\ \hline
    \end{tabular}
    \end{table}
  \autoref{tab:userAction}列举的用户行为都是比较通用的,设计人员也可以根据实际情况添加特殊的用户行为,并用他们表示用户对物品的喜好。一般来讲我们提取的用户行为一般都多于一种,根据不同行为反映用户喜好的程度将它们进行加权,得到用户对于物品的总体喜好。显式的用户反馈比隐式的权值大,但比较稀疏,毕竟进行显示反馈的用户是少数;而隐式用户行为数据是用户在使用应用过程中产生的,它可能存在大量的噪音和用户的误操作,我们可以通过经典的数据挖掘算法过滤掉行为数据中的噪音,这样可以是我们的分析更加精确。然后是归一化操作,因为不同行为的数据取值可能相差很大,比如,用户的浏览数据必然比购买数据大的多,如何将各个行为的数据统一在一个相同的取值范围中,从而使得加权求和得到的总体喜好更加精确,就需要我们进行归一化处理使得数据取值在 [0,1] 范围中。算法如\autoref{algo:longTailTags}所示。

  \begin{algorithm}
  %% \SetLine
  \KwIn{用户显示、隐式行为数据 logUsers}
  \KwOut{用户行为权重 userActionWeight}
  init userActionWeight;\\
  \For{$(user_i in logUsers)$}{
    \For{$(action_j in user_i)$}{
      //获取用户此次行为的偏好权重并做归一化\\
      $weight_{ij}$=getWeigth($action_j$);
      \If{$(user_i exsit in userActionWeight)$}{
        //对用户行为做加权处理。\\
        double remaind=userActionWeight.get($user_i$);
        userActionWeight.get($user_i$).set(remaind+$action_j*weight_j$);
      }\Else{
        userActionWeight.get($user_i$).set($action_j*weight_j$);
      }
    } 
  }
  \Return userActionWeight;
  \caption{用户满意度量化算法}
  \label{algo:userActionWeight}
  \end{algorithm}

  \subsection{标签权重的线性衰减}
  实际中使用的是基于时间衰减的用户兴趣检测模型。该算法的特点是基于向量空间模型的用户画像建模,结合手机主题用户兴趣偏好变化频繁的特点,根据时间因素权重自动进行衰减,以此反映出用户兴趣的变化。该模型是指用户对资源项目的评分仅代表评价当时的兴趣度,随着时间的推移,用户对该资源项目的评分将规律性地自动衰减,当项目评分衰减到 0 时,该资源项目将被兴趣模型所淘汰。评分衰减可以按照线性规律进行,如\autoref{pic:hl_lineReg}所示。算法描述如\autoref{algo:init_userProfile}所示。
  \begin{algorithm}
  %% \SetLine
  \KwIn{用户画像模型中所有用户兴趣哈希表 users{}}
  \KwOut{更新后的所有用户兴趣哈希表 users{}}
  \For{$(user_i in users)$}
  {
    //考虑到断电或意外关机等原因导致系统中断运行的特殊情况,算法加入了临时变量temp\\
    temp.add($user_i$.profile{});
  }
  \For{$(user_i in logUsers)$}
  {
    \For{$(tag_j in user_i)$}{
      \If{$(tag_j==0)$}{
        //若标签权重已经为 0,该用户兴趣标签将被删除\\
        remove $tag_j$;
      }\ElseIf {$(tag_j$ exsit in temp$(user_i)$}{
        //若存在新的评分值,则更新为新标签权重
        temp.get($user_i$).set($tag_j$);
      }\Else{
        //否则的话,将偏好值减少 0.5,进行衰减\\
        temp.get($user_i$).set($tag_j$-0.5);
      }
    }
  }
  \Return temp;
  \caption{用户画像线性衰减}
  \label{algo:init_userProfile}
  \end{algorithm}

  \begin{figure}
  \centering
    \framebox{\includegraphics[scale=0.5]{figures/hl_lineReg}}
    \figcaption{线性衰减模型}
    \label{pic:hl_lineReg}
  \end{figure}

\section{用户兴趣探索评估方法}
评价方法分为以下俩种:线下测试和线上测试。首先介绍线下测试基本概念,然后具体介绍工业界常用的线上A/B测试。
  \subsection{线下测试}
  笔者曾参加过2014年阿里举办的大数据竞赛,当时主要用线下测试评估算法模型优劣,总结出的基本准则是:不要过早优化模型调参和模型融合,这两部分应当留到中后期来做;不要一次性添加大量特征,最好一部分一部分添加,这样会对添加的特征效果有个大体的认识。线下测试具体步骤如下:
  \begin{itemize}
  \item 选定数据集选择和应用相关的数据集。数据需要是无偏的(unbiased),通过随机抽样能满足要求。将现有数据集分成训练集(train set) 验证集(validation set) 测试集(test set)。其中训练集用来估计模型,验证集用来确定网络结构或者控制模型复杂程度的参数,而测试集则检验最终选择最优的模型的性能如何。一个典型的划分是训练集占总样本的50%,而其它各占25%。样本少的时候可以留少部分做测试集。然后对其余N个样本采用K折交叉验证法。就是将样本打乱,然后均匀分成K份,轮流选择其中K-1份训练,剩余的一份做验证,计算预测误差平方和,最后把K次的预测误差平方和再做平均作为选择最优模型结构的依据。
  \item 建立算法模型.
  \item 准确度的评估、反馈。准确度的评估方法有Mean Absolute Error和Root Mean Squared Error,对于一个元素是 user-item 对(u, i)的集合 T,实际评分为 $r_{ui}$,预测评分为 $\hat{r}_{ui}$,无论是通过 MAE 还是 RMSE 计算,最终的结果值越小证明结果越准确。但从公式可以看出,RMSE 通过平方扩大了偏离量,同样的两组结果用RMSE得出的差异值将比MAE更大。对应公式如下:
  \begin{equation}
    MAE = \frac{1}{|T |} \sum_{(u,i)\in T}|\hat{r}_{ui}-r_{ui}|
    \label{cosine-similiarity}
  \end{equation}
  \begin{equation}
    RMSE = \sqrt{\frac{1}{|T |} \sum_{(u,i)\in T}(\hat{r}_{ui}-r_{ui})^2}
    \label{cosine-similiarity}
  \end{equation}
  \end{itemize}

  \subsection{线上A/B测试}
  在太平洋东部加拉帕戈斯(Galapagos)的一个小岛上有一种名叫达尔文雀的鸟,一部分生活在岛的西部,另一部分生活在岛的东部,由于生活环境的细微不同它们进化出了不同的喙,如\autoref{pic:hl_abbird}所示,这被认为是自然选择学说上的一个重要例证。同样一种鸟,究竟哪一种喙更适合生存呢?自然界给出了她的解决方案,让鸟儿自己变异(设计多个方案),然后优胜劣汰。具体到达尔文雀这个例子上,不同的环境中喙也有不同的解决方案。
  \begin{figure}
  \centering
    \framebox{\includegraphics[scale=0.45]{figures/hl_abbird}}
    \figcaption{达尔文雀}
    \label{pic:hl_abbird}
  \end{figure}
  上面的例子包含了A/B测试最核心的思想:多个方案并行测试;每个方案只有一个变量(比如鸟喙)不同;以某种规则优胜劣汰。评判用户画像模型的效率高低,主要是看该模型带来的点击率、转换率等指标数据,其他统计量见\autoref{tab:abtest}所示。理论上评测推荐系统的指标有用户满意度、预测准确度、覆盖率、多样性、新颖度、惊喜度、信任度、实时性、健壮性等。然而商业开发中,评测推荐结果只看重一个指标:点击转化率。能够提升商业价值,给业务带来更多利益的推荐系统,就是好的推荐系统。
  \begin{table}[htp]
  \centering
  \tabcaption{A/B测试主要评估指标}
  \label{tab:abtest}
  \begin{tabular}{ |c|p{10cm}| } \hline
   指标 & 描述 \\ \hline
   访客数 & 访客数就是指一天之内到底有多少不同的用户访问了你的网站。访客数要比IP数更能真实准确地反映用户数量。\\ \hline
   浏览量 & 即Page View,浏览量和访问次数是呼应的。用户访问网站时每打开一个页面,就记为1个PV。同一个页面被访问多次,浏览量也会累积。 \\ \hline
   点击转化率 & 点击转化率计算公式:点击转化率 = 成交笔数/浏览量 *100\%,成交笔数影响着成交金额,所以点击转化率成为了衡量推荐系统效果的重要数据之一。\\ \hline
   停留时长 & 停留时长是用户访问网站的平均停留时间,是衡量网站用户体验的一个重要指标。如果用户不喜欢主题包的内容,可能稍微看一眼就关闭页面了,那么停留时长就很短;如果用户对页面的内容很感兴趣,停留时长就很长。\\ \hline
   跳出率 & 跳出率是指访客来到网站后,只访问了一个页面就离开网站的访问次数占总访问次数的百分比,跳出率越低说明流量质量越好,用户对网站的内容越感兴趣。 \\ \hline
   其他指标 & 各种辅助性指标如点击量/用户,购买量/用户,下载量/用户等。\\ \hline
  \end{tabular}
  \end{table}

  A/B测试对用户画像建模的作用有三个:特征提取,一些标签对用户的兴趣有强相关作用,如性别标签,有些标签是弱相关作用,如用户职业标签,A/B测试需要筛选出强相关标签,过滤掉弱相关标签;权重量化,根据A/B测试实验显示,发现用户画像中的最近点击标签、最近关注标签所占权重比想象中的要大;标签组合,有些标签是冗余的,只需从中选一即可。A/B测试具体实现步骤如下:

  \begin{itemize}
  \item 方案设计。实验之前需得到一个基准版本,然后把又争议的标签按照优先度列举出来决定是否实验。真正的A/B测试只应一次改动一个地方,这意味着标签选择、权重量化、标签组合要分开来测试。
  \item 确定数据评估方案。根据实验内容不同评估它们好坏的标准也不同,如果是标签选择那么衡量的主要指标是点击量,如果是权重量化那么衡量的主要指标是点击转换率。
  \item 流量分配。为了试实验所得数据具备统计意义,能准确反映用户的真实行为,需要对流量设置一个下限。除此之外,为了使各个方案具有可比性, A、B俩个方案的流量必须是相等的。
  \item 测试周期。根据所需测试的项目的不同测试周期也有所不同,如添加一个地理标签需要的测试周期以天为单位,如果涉及到多个标签的权重变动则需要测试周期以周为单位。
  \item 评估结果。适者胜出,其代表的数据作为下一轮回A/B测试的基准版本。
  \item 建立通用的数据评估题型。在经过各种类型A/B测试实验后,已经积累很多的评估指标,有必要把这些指标抽象出来形成一个通用的数据评估模型,减少以后实验的重复设计评估指标的时间。
  \end{itemize}

\section{总结}
这一章主要研究了标签动态变化的对推荐系统的影响,实际中用户同时会受到社会因素和个人因素的影响,但这两种因素在会产生不同强度的影响。在快速变化的系统中,用户行为更加会受到社会因素 的影响,而在变化相对较慢的系统中,用户行为则更加受到个人因素的影响。本章首先介绍了用户行为数据的存储方式以及基于此的用户行为数据的的预处理。然后介绍了用户兴趣探索的组成内容,包括用户异常兴趣探测、长尾标签抽取、用户满意度量化、标签权重的线性衰减。最后给出了用户兴趣探索评估方法,包括离线和在线俩种。下一章节主要介绍如何把用户画像和兴趣探索融入到推荐系统中,从而搭建出一个具有长尾性、实时性的推荐系统。
  
\chapter{动态推荐系统设计}
  \section{前言}
  推荐系统的形式化定义如下:设C是所有用户的集合,S是所有可以推荐给用户的主题的集合。实际上,C和S集合的规模通常很大,如上百万的顾客以及上亿种歌曲等。设效用函数u()可以计算主题s对用户c的推荐度(如提供商的可靠性(vendorreliability)和产品的可得性(productavailability)等),即$u=C\times S \rightarrow R$,R是一定范围内的全序的非负实数,推荐要研究的问题就是找到推荐度R最大的那些主题S*,如\autoref{equ:fromal}
    \begin{equation}
    \forall c \in C,S^{*}=arg  max_{s \in S} u(c,s)
    \label{equ:fromal}
    \end{equation}
  除了推荐系统自身如冷启动、数据的稀疏性等问题,还有一个关注点就是推荐系统的时间效应问题。比较常见的时间效应问题主要反映在用户兴趣的变化、物品流行度的变化以及商品的季节效应,这些问题都可以利用用户画像解决。本章节主要介绍如何搭建一个具有长尾性、实时性的动态推荐形态。动态推荐形态由3个重要的模块组成:用户画像、兴趣探索模块、推荐主题建模模块、推荐算法模块和评测指标模块。通用的推荐系统模型流程如所示。推荐系统把用户模型中兴趣需求信息和推荐主题模型中的特征信息匹配,同时使用相应的推荐算法进行计算筛选,找到用户可能感兴趣的推荐主题,然后推荐给用户。

  用户画像模块对应着用户长期兴趣,用户兴趣探索对应着用户短期动态兴趣。短期兴趣的特点是临时、易变;长期兴趣的特点是长久、稳定;用户的短期兴趣可能会转化为长期兴趣,所以需要在推荐时综合考虑长期兴趣和短期兴趣。考虑到推荐系统的时间效应问题,将输入数据集归结为一个四元组,即{用户,物品,行为,时间},通过研究用户的历史行为来预测用户将来的行为。需要解决以下俩个问题:动态评分预测、时效性的影响。首先,动态评分预测问题。数据集可以选用比较直观的显性反馈数据集,即(用户,物品,评分,时间),研究是这样一个问题,给定用户u,物品i,时间t,预测用户u在时间t对物品i的评分r。对于该类问题,与时间无关的评分预测问题算法主要有以下几种:用户兴趣的变化,如年龄增长,从儿童长成青少年壮年;生活状态的变化,由以前的小学生到大学生;社会事件的影响如俩会等。此外还有季节效应问题,一些在春季很流行的,在夏季节未必就很流行。该问题的解决有待进一步思考。对于时效性的影响,每个在线系统都是一个动态系统,但它们有不同的演化速率。比如说,新闻,手机主题更新很快,但音乐,电影的系统演化的却比较慢。

  本章首先介绍用户画像和兴趣探索模块,其中兴趣探索模块需要根据业务的演化速率来调整迭代深度。然后介绍推荐主题模块,之后介绍推荐算法模块和指标体系,最后做总结。
  
  \section{用户画像和兴趣探索模块}
  目前基于用户画像的推荐,主要用在基于内容的推荐,从最近的RecSys大会(ACM Recommender Systems)上来看,不少公司和研究者也在尝试基于用户画像做Context-Aware的推荐(情境感知,又称上下文感知)。利用用户的画像,结合时间、天气等上下文信息,给用户做一些更加精准化的推荐是一个不错的方向。一个好的推荐系统要给用户提供个性化的、高效的、动态准确的推荐,那么推荐系统应能够获取反映用户多方面的、动态变化的兴趣偏好,推荐系统有必要为用户建立一个用户兴趣探索模型,该模型能获取、表示、存储和修改用户兴趣偏好,能进行推理,对用户进行分类和识别,帮助系统更好地理解用户特征和类别。推荐系统根据用户画像进行推荐,所以用户画像对推荐系统的质量有至关重要的影响。建立用户画像模型之前,需要考虑:模型的输入数据有哪些,如何获取模型的输入数据;如何考虑用户的兴趣及需求的变化;建模的对象是谁以及如何建模;模型的输出是什么。用户画像模型的输入数据主要有以下几种:
  \begin{itemize}
  \item 用户属性,分为社会属性和自然属性,包括用户最基本的如用户的姓名、年龄、职业、收入、学历等信息。用户注册时的对自然属性和社会属性进行初始建模。 
  \item 用户手工输入的信息:是用户主动输出给系统的信息,包括用户在搜索引擎中打出的关键词,用户评论中发布的感兴趣的主题、频道。还有一类重要的信息就是用户反馈的信息,包括用户自己对推荐结果的满意程度;用户标注的浏览页面的感兴趣、不感兴趣或感兴趣的程度等。
  \item 用户的浏览行为和浏览内容:用户浏览的行为和内容体现了用户的兴趣和需求,它们包括浏览次数、频率、停留时间等,浏览页面时的操作(收藏、保存、复制等)、浏览时用户表情的变化等。服务器端保存的日志也能较好地记录用户的浏览行为和内容。
  \item 推荐对象的属性特征:不同的推荐对象,用户建模的输入数据也不同。网页等推荐对象通常考虑对象的内容和用户之间的相似性,而产品等推荐对象通常考虑用户对产品的评价。为提高推荐质量,推荐对象的相关的属性也要考虑进去,比如除网页内容以外,还要考虑网页的发布人、时间等。产品类的对象还要考虑产品的品牌、价格、出售时间等。
  \end{itemize}

  用户行为的权重排序。用户显式行为数据记录了用户在平台上不同的环节的各种行为,这些行为一方面用于候选集触发算法中的离线计算(主要是点击、浏览),另外一方面,这些行为代表的用户兴趣强弱不同,因此在训练重排序模型时可以针对不同的行为设定不同的权重值,以更细地刻画用户的行为强弱程度。此外,用户的购买、试用等行为还可以作为重排序模型的交叉特征,用于模型的离线训练和在线预测。负反馈数据反映了当前的结果可能在某些方面不能满足用户的需求,因此在后续的候选集触发过程中需要考虑对特定的因素进行过滤或者降权,提高用户体验;同时在重排序的模型训练中,A/B测试结果可以作为不可多得的负例参与模型训练。用户画像是刻画用户属性的元数据,其中有些是直接获取的基础数据,有些是经过挖掘的二次数据,这些属性一方面可以用于候选集触发过程中对标签进行加权或降权,另外一方面可以作为重排序模型中的用户维度特征。通过对数据的挖掘可以提取出一些关键词,然后使用这些关键词给主题打标签,用于主题的个性化展示。

  用户行为的获取方式。模型输入数据的方式有显式获取、隐式获取和启发式获取三种方式。显式获取用户兴趣偏好的方法是简单而直接的做法,能准确地反映用户的需求,同时所得的信息比较具体、全面、客观,结果比较可靠。缺点就是数量稀少,原因用户不太愿意花时间来向商家表达自己的喜好,并且这种方法灵活性差,答案存在异质性,当用户兴趣主题改变时需要用户手动更改系统中用户兴趣。同时该方法对用户不是很人性化。解决人性化问题是推荐系统未来的一个研究方向,来研究用户能够接受的评价方式是什么,比如能够有耐心进行几次评分。利用固定负担模型来计量用户评价的负担,将人性化设计问题转化为最优化问题来研究。隐式获取法是指系统通过记录用户行为数据,通过权重排序获取用户的兴趣偏好,用户的很多动作都能暗示用户的喜好,包括查询、浏览页面和文章、标记书签、反馈信息、滑屏等。隐式的跟踪可以在建立用户画像基本数据的同时不打扰用户的正常消费活动。这种方法的缺点就是跟踪的结果未必能正确反映用户的兴趣偏好。同时系统若过度跟踪用户的历史记录,有时会引发用户隐私问题,而放弃对当前推荐系统的使用。 上述获取兴趣偏好的方法有时受用户教育背景、职业和习惯等因素的限制,用户有时意识不到自己的兴趣主题,因此能为用户提供启发式信息,如领域术语抽取和相似度物品聚类,可以实现领域知识的复用,为用户间的协同提供支持,提高用户兴趣获取质量。用户的兴趣和需求会随着时间和情景发生变化,用户画像模块要考虑到用户长期兴趣偏好和短期兴趣偏好,还要考虑兴趣的变化,目前很多研究关注了用户的长期兴趣,建立了静态用户画像模型,但用户兴趣探索模型也越来越受到关注。结合长期和短期兴趣的动态建模将是未来的一个研究方向,如\autoref{pic:hl_iterate}所示。
  \begin{figure}
    \centering
      \framebox{\includegraphics[scale=0.4]{figures/hl_iterate}}
      \figcaption{用户画像的使用}
      \label{pic:hl_iterate}
  \end{figure}

  用户画像更新采用了时间窗方法和遗忘机制来反映用户兴趣的变化。目前的更新机制无法及时跟踪用户兴趣的变化,just-in-time型有更强学习效率和动态变化适应能力的建模也是未来的重要研究方向。 

  \section{推荐主题模块}
  推荐主题分为单用户建模和群组建模,单用户建模针对个体用户进行建模,比如基于主题内容的推荐,群组建模是针对一类用户进行建模,比如基于商品的协同推荐。

  应用于不同的领域的推荐系统其推荐的主题也各不相同,如何对推荐主题进行描述对推荐系统也有很重要的影响。和用户画像一样,要对推荐主题进行描述之前要考虑:提取推荐主题的什么特征,如何提取,提取的特征用于什么目的,主题的特征描述和用户画像之间有关联。提取到的每个主题特征标签对推荐结果会有什么影响。主题的特征描述文件能否自动更新。
  推荐主题的描述文件中的主题特征和用户画像中的兴趣标签进行推荐计算,获得推荐主题的推荐权重,所以推荐主题的描述文件与用户画像密切相关,通常的做法是用同样的方法来表达用户的兴趣偏好和推荐主题。推荐系统推荐主题包括众多的领域,比如体育、动漫、科技、国家,还有诸如音乐、电影等多媒体资源等等。不同的主题,特征也不相同,目前并没有一个统一的标准来进行统一描述,主要有基于内容的方法和基于分类的方法两大类方法。 基于内容的方法是从主题本身抽取相关信息来表示主题,使用最广泛的方法是用加权关键词矢量,该方法通过对标注主题的标签进行统计分析得出的特征向量。方法很多,比较简单的做法就是计算每个特征的熵,选取具有最大熵值的若干个特征;也可以计算每个标签的信息增量(Information gain),即计算每个特征在主题中出现前后的信息熵之差;还可以计算每个特征的互信息(mutual information),即计算每个特征和主题的相关性。在完成主题特征提取后,还需要计算每个特征的权值,权值大的对推荐结果的影响就大。基于分类的方法是把推荐主题放入不同类别中,这样可以把同类主题推荐给对该类主题感兴趣的用户了。文本分类的方法有多种,比如朴素贝叶斯(Naive-Bayes),k最近邻方法(KNN)和支持向量机(SVM)等。 主题的类型可以预先定义,也可以利用聚类算法自动产生。研究表明聚类的精度非常依赖于主题的数量,而且由自动聚类产生的类型可能对用户来说是毫无意义的,因此可以有选择的进行手工选定的类型来分类主题,在没有对应的候选类型或需要进一步划分某类型时,才使用聚类产生的类型。推荐系统推荐给用户的主题首先不能与用户购买过的主题重复,其次也不能与用户刚刚看过的主题不是太形似或者太不相关,这就是所谓的模型过拟合问题(可扩展性问题)。出现这一问题的本质上来自数据的不完备性,解决的主要的方法是引入随机性,使算法收敛到全局最优或者逼近全局最优。针对这一问题考察了被推荐的主题的相关性和冗余性,要同时保证推荐的多样性,又不能与用户看过的主题重复或毫不相关。关于这一问题的研究是推荐系统研究的一个难点和重点。 推荐系统中出现新的主题时,推荐系统尤其是协同过滤系统中,新主题出现后必须等待一段时间才会有用户浏览和评价,而在此之前推荐系统是无法对此主题进行推荐,这就是推荐系统研究的另一个难点和重点——商品冷启动问题。解决这一问题的方法就是考虑利用组合推荐方法。
  
  \section{推荐算法模块}
  推荐算法类型很多,但是各有各的局限,比较常用的有基于内容推荐,协同过滤推荐,基于关联规则推荐,基于效用推荐,基于知识推荐,组合推荐。他们的主要优缺点对比如所示。
  \begin{table}[htp]
  \centering
  \tabcaption{推荐系统主要算法比较}
  \label{tab:algarithm}
  \begin{tabular}{ |c|p{6cm}|p{6cm}| } \hline
   推荐方法 & 优点 & 缺点 \\ \hline
   基于内容推荐 & 推荐结果直观,容易解释;不需要领域知识 & 稀疏问题;新用户问题;复杂属性不好处理;要有足够数据构造分类器 \\ \hline
   协同过滤推荐 & 新异兴趣发现、不需要领域知识;随着时间推移性能提高;推荐个性化、自动化程度高;能处理复杂的非结构化对象 & 稀疏问题;可扩展性问题;新用户问题;质量取决于历史数据集;系统开始时推荐质量差; \\ \hline
   基于规则推荐 & 能发现新兴趣点;不要领域知识 & 规则抽取难、耗时;产品名同义性问题;个性化程度低; \\ \hline
   基于效用推荐 & 无冷开始和稀疏问题;对用户偏好变化敏感;能考虑非产品特性 & 用户必须输入效用函数;推荐是静态的,灵活性差;属性重叠问题; \\ \hline
   基于知识推荐 & 能把用户需求映射到产品上;能考虑非产品属性 & 知识难获得;推荐是静态的\\ \hline
  \end{tabular}
  \end{table}
  推荐算法本身是一个综合性的问题,可以简单地用最基本的Content-based,再复杂点可以Collaborative Filtering,更深入一些诸如基于SVD/LDA等的降维算法和基于SVD++等的评分预测算法,或者把推荐问题再转换成分类问题,或者采用以上算法前先用各种聚类算法做数据的预处理。
    
    \subsection{推荐算法}
    基于内容推荐。基于内容的推荐(Content-based Recommendation)是信息过滤技术的延续与发展,它是建立在项目的内容信息上作出推荐的,而不需要依据用户对项目的评价意见,更多地需要用机 器学习的方法从关于内容的特征描述的事例中得到用户的兴趣资料。在基于内容的推荐系统中,项目或对象是通过相关的特征的属性来定义,系统基于用户评价对象 的特征,学习用户的兴趣,考察用户资料与待预测项目的相匹配程度。用户的资料模型取决于所用学习方法,常用的有决策树、神经网络和基于向量的表示方法等。 基于内容的用户资料是需要有用户的历史数据,用户资料模型可能随着用户的偏好改变而发生变化。基于内容推荐方法的优点是:不需要其它用户的数据,没有冷开始问题和稀疏问题。能为具有特殊兴趣爱好的用户进行推荐。能推荐新的或不是很流行的项目,没有新项目问题。通过列出推荐项目的内容特征,可以解释为什么推荐那些项目。已有比较好的技术,如关于分类学习方面的技术已相当成熟。缺点是要求内容能容易抽取成有意义的特征,要求特征内容有良好的结构性,并且用户的口味必须能够用内容特征形式来表达,不能显式地得到其它用户的判断情况。

    协同过滤推荐。协同过滤推荐(Collaborative Filtering Recommendation)技术是推荐系统中应用最早和最为成功的技术之一。它一般采用最近邻技术,利用用户的历史喜好信息计算用户之间的距离,然后 利用目标用户的最近邻居用户对商品评价的加权评价值来预测目标用户对特定商品的喜好程度,系统从而根据这一喜好程度来对目标用户进行推荐。协同过滤最大优 点是对推荐对象没有特殊的要求,能处理非结构化的复杂对象,如音乐、电影。协同过滤是基于这样的假设:为一用户找到他真正感兴趣的内容的好方法是首先找到与此用户有相似兴趣的其他用户,然后将他们感兴趣的内容推荐给此用户。其基本 思想非常易于理解,在日常生活中,我们往往会利用好朋友的推荐来进行一些选择。协同过滤正是把这一思想运用到电子商务推荐系统中来,基于其他用户对某一内 容的评价来向目标用户进行推荐。基于协同过滤的推荐系统可以说是从用户的角度来进行相应推荐的,而且是自动的,即用户获得的推荐是系统从购买模式或浏览行为等隐式获得的,不需要用户努力地找到适合自己兴趣的推荐信息,如填写一些调查表格等。和基于内容的过滤方法相比,协同过滤具有如下的优点:能够过滤难以进行机器自动内容分析的信息,如艺术品,音乐等。共享其他人的经验,避免了内容分析的不完全和不精确,并且能够基于一些复杂的,难以表述的概念(如信息质量、个人品味)进行过滤。有推荐新信息的能力。可以发现内容上完全不相似的信息,用户对推荐信息的内容事先是预料不到的。这也是协同过滤和基于内容的过滤一个较大的差别,基于内容的过滤推荐很多都是用户本来就熟悉的内容,而协同过滤可以发现用户潜在的但自己尚未发现的兴趣偏好。能够有效的使用其他相似用户的反馈信息,较少用户的反馈量,加快个性化学习的速度。 虽然协同过滤作为一种典型的推荐技术有其相当的应用,但协同过滤仍有许多的问题需要解决。最典型的问题有稀疏问题(Sparsity)和可扩展问题(Scalability)。协同过滤(CF)可以看做是一个分类问题,也可以看做是矩阵分解问题。协同滤波主要是基于每个人自己的喜好都类似这一特征,它不依赖于个人的基本信息。比如刚刚那个电影评分的例子中,预测那些没有被评分的电影的分数只依赖于已经打分的那些分数,并不需要去学习那些电影的特征。

    基于关联规则推荐。基于关联规则的推荐(Association Rule-based Recommendation)是以关联规则为基础,把已购商品作为规则头,规则体为推荐对象。关联规则挖掘可以发现不同商品在销售过程中的相关性,在零 售业中已经得到了成功的应用。管理规则就是在一个交易数据库中统计购买了商品集X的交易中有多大比例的交易同时购买了商品集Y,其直观的意义就是用户在购 买某些商品的时候有多大倾向去购买另外一些商品。比如购买牛奶的同时很多人会同时购买面包。算法的第一步关联规则的发现最为关键且最耗时,是算法的瓶颈,但可以离线进行。其次,商品名称的同义性问题也是关联规则的一个难点。

    基于效用推荐。基于效用的推荐(Utility-based Recommendation)是建立在对用户使用项目的效用情况上计算的,其核心问题是怎么样为每一个用户去创建一个效用函数,因此,用户资料模型很大 程度上是由系统所采用的效用函数决定的。基于效用推荐的好处是它能把非产品的属性,如提供商的可靠性(Vendor Reliability)和产品的可得性(Product Availability)等考虑到效用计算中。

    基于知识推荐。基于知识的推荐(Knowledge-based Recommendation)在某种程度是可以看成是一种推理(Inference)技术,它不是建立在用户需要和偏好基础上推荐的。基于知识的方法因它们所用的功能知识不同而有明显区别。效用知识(Functional Knowledge)是一种关于一个项目如何满足某一特定用户的知识,因此能解释需要和推荐的关系,所以用户资料可以是任何能支持推理的知识结构,它可以是用户已经规范化的查询,也可以是一个更详细的用户需要的表示。

    组合推荐。由于各种推荐方法都有优缺点,所以在实际中,组合推荐(Hybrid Recommendation)经常被采用。研究和应用最多的是内容推荐和协同过滤推荐的组合。最简单的做法就是分别用基于内容的方法和协同过滤推荐方法 去产生一个推荐预测结果,然后用某方法组合其结果。尽管从理论上有很多种推荐组合方法,但在某一具体问题中并不见得都有效,组合推荐一个最重要原则就是通 过组合后要能避免或弥补各自推荐技术的弱点。在组合方式上,有研究人员提出了七种组合思路:加权(Weight):加权多种推荐技术结果。变换(Switch):根据问题背景和实际情况或要求决定变换采用不同的推荐技术。混合(Mixed):同时采用多种推荐技术给出多种推荐结果为用户提供参考。特征组合(Feature combination):组合来自不同推荐数据源的特征被另一种推荐算法所采用。层叠(Cascade):先用一种推荐技术产生一种粗糙的推荐结果,第二种推荐技术在此推荐结果的基础上进一步作出更精确的推荐。特征扩充(Feature augmentation):一种技术产生附加的特征信息嵌入到另一种推荐技术的特征输入中。元级别(Meta-level):用一种推荐方法产生的模型作为另一种推荐方法的输入。

    \subsection{AB测试}
    产品的改变并不总是意味着进步,有时候无法评判多种设计方案中哪一种更优秀的, 这时A/B测试就派上用场了,A/B测试可以回答两个问题:哪个方案好结果的可信程度A/B测试结果是基于用户得到的结果,用数据说话, 而不是凭空想象去为用户代言,并且通过一定的数学分析给出结果的可信度。A/B测试需要如下几个前提:多个方案并行测试;每个方案只有一个变量不同;能够以某种规则优胜劣汰其中第2点暗示了A/B测试的应用范围:A/B测试必须是单变量,但有的时候,我们并不追求知道某个细节对方案的影响,而只想知道方案的整体效果如何,那么可以适当增加变量,当然测试方案有非常大的差异时一般不太适合做A/B测试,因为它们的变量太多了,变量之间会有很多的干扰,所以很难通过A/B测试的方法找出各个变量对结果的影响程度。在满足上述前提时,便可以做A/B测试了。

    目标转换率变化区间估计:在做A/B测试的时候,抽样得到的数据并不能准确反映整体的真实水平,即样本得到的估计是有偏差的,因此需要去评估这个值可能的变化区间。例如通过区间估计得到:A方案转换率为:6.5\% ± 1.5\%B方案转换率为:7.5\% ± 1.5\%方案胜出概率估计:由于最终有意义的是确立胜出的版本,然而并不是所有的实验都能做到样本足够大,区分度足够高的,因此确定版本胜出的概率,很多英文资料里面记为Chance to beat baseline,即在给定转换率下,变体版本的实际转换率高于参展版本(默认是原始版本)的实际转换率的可能性。在实验之前需要设定一个阈值(称为置信度),某版本胜出的可能性高于这个值并且稳定时,便可以宣布该版本胜出。置信度越高,结果的可靠信越高;随着置信度的增加实验时间将会变长。

  \section{动态推荐系统底层架构}
    \subsection{基于Spark}
    基于Spark的方式在架构上,第一种是使用Spark把模型计算放在内存中,加快模型计算速度,Hadoop中作业的中间输出结果是放到硬盘的HDFS中,而Spark是直接保存在内存中,因此Spark能更好地适用于数据挖掘与机器学习等需要迭代的模型计算,如表9-2所示。
    \begin{table}[htp]
    \centering
    \tabcaption{MR和spark对比}
    \label{tab:spark}
    \begin{tabular}{ |p{3cm}|p{5cm}|p{5cm}| } \hline
     过程 & MapReduce & Spark \\ \hline
     collect & 在内存中构造了一块数据结构用于map输出的缓冲 & 没有在内存中构造一块数据结构用于map输出的缓冲,而是直接把输出写到磁盘文件 \\ \hline
     sort & map输出的数据有排序 & map输出的数据没有排序 \\ \hline
     merge & 对磁盘上的多个spill文件最后进行合并成一个输出文件 & 在map端没有merge过程,在输出时直接是对应一个reduce的数据写到一个文件中,这些文件同时存在并发写,最后不需要合并成一个 \\ \hline
     copy框架 & jetty & netty或者直接socket流 \\ \hline
     对于本节点上的文件 & 仍然是通过网络框架拖取数据 & 不通过网络框架,对于在本节点上的map输出文件,采用本地读取的方式\\ \hline
     copy过来的数据存放位置 & 先放在内存,内存放不下时写到磁盘 & 一种方式全部放在内存;另一种方式先放在内存\\ \hline
     merge sort & 最后会对磁盘文件和内存中的数据进行合并排序 & 对于采用另一种方式时也会有合并排序的过程\\ \hline
    \end{tabular}
    \end{table}

    \subsection{基于Kiji框架}
    Kiji是一个用来构建大数据应用和实时推荐系统的开源框架,本质上是对HBase上层的一个封装,用Avro来承载对象化的数据,使得用户能更容易地用HBase管理结构化的数据,使得用户姓名、地址等基础信息和点击、购买等动态信息都能存储到一行,在传统数据库中,往往需要建立多张表,在计算的时候要关联多张表,影响实时性。Kiji提供了一个KijiScoring模块,它可以定义数据的过期策略,如综合产品点击次数和上次的点击时间,设置数据的过期策略把数据刷新到KijiScoring服务器中,并且根据自己定义的规则,决定是否需要重新计算得分。如用户有上千万浏览记录,一次的行为不会影响多少总体得分,不需要重新计算,但如果用户仅有几次浏览记录,一次的行为,可能就要重新训练模型。Kiji也提供了一个Kiji模型库,使得改进的模型部署到生产环境时不用停掉应用程序,让开发者可以轻松更新其底层的模型。

    \subsection{基于Storm}
    最后一种基于 Storm 的实时推荐系统。在动态推荐上,算法本身不能设计的太复杂,手机主题推荐系统的数据库是TB级别,实时读写大表比较耗时。可以把算法分成离线部分和实时部分,利用Hadoop离线任务尽量把查询数据库比较多的、可以预先计算的模型先训练好,或者把计算的中间数据先计算好,比如,线性分类器的参数、聚类算法的群集位置或者协同过滤中条目的相似性矩阵,然后把少量更新的计算留给Storm实时计算,一般是具体的评分阶段。用HBase或HDFS存储历史的浏览、购买行为信息,用Hadoop基于User CF的协同过滤,先把用户的相似度离线生成好,用户到商品的矩阵往往比较大,运算比较耗时,把耗时的运行先离线计算好,实时调用离线的结果进行轻量级的计算有助于提高主题推荐的实时性。协同过滤算法在storm上计算过程为:首先程序获取用户和主题的历史数据,得到用户到主题的偏好矩阵,利用Jaccard相似系数(Jaccard coefficient)、向量空间余弦相似度(Cosine similarity)、皮尔逊相关系数(Pearson correlation coefficient)等相似度计算方法,得到相邻的用户(User CF)或相似商品(Item CF)。在User CF中,基于用户历史偏好的相似度得到邻居用户,将邻居用户偏好的主题推荐给该用户;在Item CF中,基于用户对物品的偏好向量得到相似主题,然后把这款主题推荐给喜欢相似主题的其他用户。然后通过Kafka或者Redis队列,保存前端的最新浏览等事件流,在Storm的Topology中实时读取里面的信息,同时获取缓存中用户topN个邻居用户,把邻居用户喜欢的商品存到缓存中,前端从缓存中取出商品,根据一定的策略,组装成推荐列表。除了相似性矩阵,其他模型大体实现也相似,比如实际的全品类电商中不同的品类和栏位,往往要求不同的推荐算法,如母婴主题,如果结合商品之间的序列模式和母婴年龄段的序列模式,效果会比较好,可以把模型通过Hadoop预先生成好,然后通过Storm实时计算来预测用户会买哪些主题。

  \section{量化评估推荐系统}
  推荐系统还是看目的是如何的,从用户角度讲是为了更好的理解用户,减少用户查找内容的时间和次数,从产品本身角度讲,是增加单位面积单位时间内的点击数或者说内容有效。 从业务角度的衡量:衡量点击和打开率,这说明用户是否对内容感兴趣。衡量通过推荐系统替代用户主动搜索或者主动浏览的次数,可以通过横向与使用其他产品对比较,比如使用推荐系统提供内容的用户搜索次数和点击浏览目录次数明显下降。衡量推荐系统的满意度口碑,刨除因为页面位置效果等因素,衡量推荐系统一个重要的就是满意度的口碑问题,这个可以通过单个用户是否有重复使用的行为,曲线是否是一直上升的来衡量,如果一直有新用户访问,但一直没有老用户重复使用,说明用户满意度有问题。
  \section{总结}
  推荐系统经过了相当时间的发展,同时一些重点和难点问题得到了研究者的关注,相信是未来研究的热点问题。用户兴趣偏好获取方法和推荐对象的特征提取方法的研究目前的推荐系统中实际上较少使用了用户和推荐对象的特征,即使使用很广泛的协同推荐使用的是用户的评分。主要是用户兴趣偏好的获取方法和推荐对象特征提取方法不是很适用,需要引入更精确适用的用户和对象特征。(2)推荐系统的安全性研究进行协同推荐时需要掌握用户的兴趣偏好等用户信息,但用户担心个人数据得不到有效保护而不愿暴露个人信息,这是协同推荐长期存在的一个问题。既能得到用户信息而提高推荐系统性能,又能有效保护用户信息将是未来推荐系统的一个研究方向。同时一些不法的用户为了提高或降低某些对象的推荐概率,恶意捏造用户评分数据而达到目的,这也是推荐系统存在的一个安全问题,被称为推荐攻击[93-96]。检测并能预防推荐攻击也将是未来一个研究方向。(3)基于复杂网络理论及图方法的推荐系统研究复杂网络理论和图方法同协同推荐存在契合点,在文献中网络视频推荐问题转化为热量散播平衡态网络上的谱图分割问题,通过设计长尾发现的推荐策略引导用户发现潜在的感兴趣的网络视频。利用复杂网络理论和图方法进行推荐也是推荐系统研究的一个方向。(4)推荐的多维度研究目前的推荐研究都是基于用户-对象二维空间进行研究的,但是用户选择某个对象以及对对象的评分在不同的情况下会有所不同,也就是推荐使用的特征维度会有所不同,研究推荐的多维度也是未来的一个研究方向。(5)稀疏性和冷启动研究稀疏性和冷启动问题是困扰推荐系统很长时间了,包括经典协同过滤算法和新出现的基于网络结构的推荐算法都存在该问题。有很多研究者对这一问题进行研究并提出解决办法,但该问题依然存在,还需要对其进行研究。(6)推荐系统性能评价指标的研究用户对算法准确度的敏感度、算法对不同领域的普适性、广义的质量评价方法等都是未来推荐系统性能评价要进行研究的目标。
  
\chapter{结束语}
  如果说过去的十年是搜索技术大行其道的十年,那么个性化推荐技术将成为未来十年中最重要的革新之一。目前几乎所有大型的电子商务系统,如Amazon、阿里、小米、滴滴等,都不同程度地使用了各种形式的推荐系统。一个好的推荐系统需要满足的目标有:个性化推荐系统必须能够基于用户之前的口味和喜好提供相关的精确的推荐,而且这种口味和喜欢的收集必须尽量少的需要用户的劳动。推荐的结果必须能够实时计算,这样才能够在用户离开网站前之前获得推荐的内容,并且及时的对推荐结果作出反馈。实时性也是推荐系统与通常的数据挖掘技术显著不同的一个特点。一个完整的推荐系统由三部分构成:用户画像模块,用户行为模块、推荐算法模块。用户画像模块记录了用户长期的信息,刻画用户的基础类型。用户行为模块负责记录能够体现用户喜好的行为,比如购买、下载、评分等。这部分看起来简单,其实需要非常仔细的设计。比如说购买和评分这两种行为表达潜在的喜好程度就不尽相同完善的行为记录需要能够综合多种不同的用户行为,处理不同行为的累加。推荐算法模块的功能则实现了对用户行为记录的分析,采用不同算法建立起模型描述用户的喜好信息,通过推荐模块实时的从内容集筛选出目标用户可能会感兴趣的内容推荐给用户。因此,除了推荐系统本身,为了实现推荐,还需要一个可供推荐的内容集。比如,对于手机主题推荐系统来说,所有上线主题就是这样的内容集。我们对内容集本身需要提供的信息要求非常低,在经典的协同过滤算法下,内容集甚至只需要提供ID就足够。而对于基于内容的推荐系统来说,由于往往需要对内容进行特征抽取和索引,我们就会需要提供更多的领域知识和标签属性。

  推荐系统是一种联系用户和内容的信息服务系统,一方面它能够帮助用户发现他们潜在感兴趣的内容,另一方面它能够帮助内容供者将内容投放给对它感兴趣的用户。推荐系统的主要方法是通过分析用户的历史行为来预测他们未来的行为。因此,时间是影响用户行为的重要因素。关于推荐系统动态特性的研究相对比较少,特别是缺乏系统性的研究。对动态推荐系统的研究,无论是从促进用户兴趣模型的理论角度出发,还是从实际需求来看,都具有重要的意义,本文的研究工作正是在这一背景下展开。

  \section{研究工作总结}
    本文对推荐系统特别是与用户画像相关的动态推荐系统的相关工作做了总结和回顾之外,主要的工作包括以下几个方面:
    \begin{itemize}
      \item 设计出了基于用户画像的推荐模型:按照用户属性和行为特征对全部用户进行聚类和精细化的客户群细分,将用户行为相同或相似的用户归类到一个消费群体,这样就可以将推荐平台所有的用户划分为N个不同组,每个组用户拥有相同或相似的行为特征,这样电商平台就可以按照不同组的用户行为对其进行个性化智能推荐。目前国内主流电商平台,在进行个性化智能推荐系统升级过程,都在逐步向DNN渗透和扩展,也是未来个性化智能推荐必经之路。在现有用户画像、用户属性打标签、客户和营销规则配置推送、同类型用户特性归集分库模型基础上,未来将逐步扩展机器深度学习功能,通过系统自动搜集分析前端用户实时变化数据,依据建设的机器深度学习函数模型,自动计算匹配用户需求的函数参数和对应规则,推荐系统根据计算出的规则模型,实时自动推送高度匹配的营销活动和内容信息。
      \item 设计出了考虑用户长期兴趣和短期兴趣的用户兴趣探索模型:通过量化用户满意度和量化量化主题标签的流行度,最大化推荐系统的推荐多样性和长尾性,提高用户的惊喜度。
      \item 动态推荐系统的原型设计:综合前几章的推荐系统各个模块的研究,设计了一个实际的推荐系统的原型。该系统包含了用户画像和用户兴趣探索模型。能够实时根据用户行为变化的趋势,实时的调整推荐结果排名,从而不断改善用户在推荐系统中的体验。
    \end{itemize}

  \section{对未来工作的展望}
  本文对推荐系统的用户画像和用户兴趣探索模型进行了较深入的研究,但是针对用户兴趣变化的推荐模型的实现还有很多工作要做。本人认为用户兴趣动态推荐系统的实现有待解决的问题如下:
    \begin{itemize}
      \item 用户行为的离线和在线计算的分配:用户行为每天产生的数据量很大,哪些行为需要在线实时计算反馈,哪些行为只需要离线计算即可,需要根据具体业务的特点和用户习惯赋予每种行为一个权重,然后根据权重排名决定计算方式。因此,用户行为的特征提取、分析将是我们将来工作的一个重要方面。
      \item 用户兴趣探索模型对推荐系统的影响:本文的所有工作基本集中在高推荐系统的点击购买转换率上。但点击购买转换率并不是推荐系统追求的唯一指标。比如,预测用户可能会去看,从而给用户推荐速度与激情,这并不是一个好的推荐。因为速度与激情的热度很高,因此并不需要别人给他们推荐。上面这个例子涉及到了推荐系统的长尾度,即用户希望推荐系统能够给他们新颖的推荐结果,而不是那些他们已经知道的物品。此外,推荐系统还有多样性等指标。如何利用时间信息,在不牺牲转换率的同时, 提高推荐的其他指标,是我们将来工作研究的一个重要方面。
      \item 推荐系统随时间的进化:用户的行为和兴趣是随时间变化的,意味着推荐系统本身也是一个不断演化的系统。其各项指标,包括长尾度,多样性,点击率都是随着数据的变化而演化。如何让推荐系统能够通过利用实时变化的用户反馈,向更好的方面发展是推荐系统研究的一个重要方面。
    \end{itemize}

  最后,希望本文的研究工作能够对动态推荐系统的发展作出一定的贡献,并真诚的希望老师们出宝贵的批评意见和建议。

%%%%%%%%%%%%%%%%%%%%%%%%%%%%%%
%% 附件部分
%%%%%%%%%%%%%%%%%%%%%%%%%%%%%%
\backmatter

  % 参考文献
  % 使用 BibTeX
  % 选择参考文献的排版格式。注意ustcbib这个格式不保证完全符合要求,请自行决定是否使用
  %\bibliographystyle{ustcbib}%{GBT7714-2005NLang-UTF8}
 % \bibliography{bib/tex}
  %\nocite{*} % for every item
  % 不使用 BibTeX
  %\renewcommand{\baselinestretch}{0.5}
\begin{thebibliography}{10}
\bibitem{long-tail}
O. Celma.
\newblock{\em Music Recommendation and Discovery in the Long Tail}.
\newblock Springer. 2010.

\bibitem{content-based}
Marko Balabanovi ́c and Yoav Shoham.
\newblock {\em Fab: content-based, collaborative recommendation}.
\newblock Commun. ACM, 40:66–72, March 1997.

\bibitem{cold-start}
Andrew I. Schein, Alexandrin Popescul, Lyle H. Ungar, David M. Pennock.
\newblock {\em Methods and Metrics for Cold-Start Recommendations}.
\newblock New York City, New York: ACM. pp. 253–260. 2002.

\bibitem{user-interest}
C\TeX{} Sia, K.C., Zhu, S., Chi, Y., Hino, K., Tseng, B.L.
\newblock {\em Capturing User Interests by Both Exploitation and Exploration}.
\newblock Technical report, NEC Labs America. 2006.

\bibitem{info-retrieval}
Jansen, B. J. and Rieh, S.
\newblock {\em The Seventeen Theoretical Constructs of Information Searching and Information Retrieval}.
\newblock Journal of the American Society for Information Sciences and Technology. 61(8), 2010.

\bibitem{date-mining}
Han, Jiawei; Kamber, Micheline.
\newblock {\em Data mining: concepts and techniques}.
\newblock Morgan Kaufmann. p. 5. 2001.recmd-system

\bibitem{recmd-system}
Francesco Ricci and Lior Rokach and Bracha Shapira.
\newblock {\em Introduction to Recommender Systems Handbook}.
\newblock Springer, pp. 1-35. 2011.

\bibitem{matthew-effect}
Robert K. Merton.
\newblock {\em The Matthew Effect in Science}.
\newblock Science, 159(3810):56– 63, January 1968.

\bibitem{matthew-effect:2}
Junghoo Cho and Sourashis Roy.
\newblock {\em Impact of search engines on page popularity}.
\newblock In Proceedings of the 13th international conference on World Wide Web, WWW ’04, pages 20–29, New York, NY, USA, ACM. 2004.

\bibitem{matthew-effect:3}
Daniel M. Fleder and Kartik Hosanagar.
\newblock {\em Recommender systems and their impact on sales diversity}.
\newblock In Proceedings of the 8th ACM conference on Electronic commerce, EC ’07, pages 192–199, New York, NY, USA, ACM. 2007.

\bibitem{social-filter}
Henry Kautz, Bart Selman, and Mehul Shah.
\newblock {\em Referral web: combining social networks and collaborative filtering}.
\newblock Commun. ACM, 40:63–65, March 1997.

\bibitem{collab-filter}
Jonathan L. Herlocker, Joseph A. Konstan, Loren G. Terveen, and John T. Riedl.
\newblock {\em Evaluating collaborative filtering recommender systems}.
\newblock ACM Trans. Inf. Syst., 22:5–53, January 2004.

\bibitem{ab-test}
Kohavi, Ron, Longbotham, Roger.
\newblock {\em Online Controlled Experiments and A/B Tests}.
\newblock In Sammut, Claude; Webb, Geoff. 2015.

\bibitem{cognitive-science}
Elaine Rich.
\newblock {\em Readings in intelligent user interfaces}.
\newblock chapter User modeling via stereotypes, pages 329–342. 1998.

\bibitem{info-overload}
Anne-F. Rutkowski and Carol S. Saunders.
\newblock {\em Growing pains with information overload}.
\newblock Computer, 43:96–95, June 2010.

\bibitem{Forecast-principle}
J. Scott Armstrong, editor.
\newblock {\em Principles of Forecasting - A Handbook for Researchers and Practitioners}.
\newblock Kluwer Academic, 2001.

\bibitem{cf-sn}
Henry Kautz, Bart Selman, and Mehul Shah.
\newblock {\em Referral web: combining social networks and collaborative filtering}.
\newblock Commun. ACM, 40:63–65, March 1997.

\bibitem{info-overload}
Greg Linden, Brent Smith, and Jeremy York. 
\newblock {\em Amazon.com recommendation- s: Item-to-item collaborative filtering}.
\newblock IEEE Internet Computing, 7:76–80, January 2003.

\bibitem{Amazon-cf}
Anne-F. Rutkowski and Carol S. Saunders.
\newblock {\em Growing pains with information overload}.
\newblock Computer, 43:96–95, June 2010.

\bibitem{temporal-cf}
Yehuda Koren.
\newblock {\em Collaborative filtering with temporal dynamics}.
\newblock In Proceedings of the 15th ACM SIGKDD international conference on Knowledge discovery and data mining, KDD ’09, pages 447–456, New York, NY, USA, 2009. ACM.

\bibitem{latent-cf}
Thomas Hofmann and Jan Puzicha.
\newblock {\em Latent class models for collaborative filtering}.
\newblock In Proceedings of the Sixteenth International Joint Conference on Artificial Intelligence, IJCAI ’99, pages 688–693, San Francisco, CA, USA, 1999. Morgan Kaufmann Publishers Inc.

\bibitem{demo-data}
Bruce Krulwich.
\newblock {\em Lifestyle finder: Intelligent user profiling using large-scale demographic data}.
\newblock AI Magazine, 18(2):37–45, 1997.

\bibitem{Trust-walker}
Mohsen Jamali and Martin Ester.
\newblock {\em Trustwalker: a random walk model for combining trust-based and item-based recommendation}.
\newblock In Proceedings of the 15th ACM SIGKDD international conference on Knowledge discovery and data mining, KDD ’09, pages 397–406, New York, NY, USA, ACM. 2009.

\end{thebibliography}


  % 附录,没有请注释掉
  %\begin{appendix}
  %  \include{chapter/chap-req}
  %\end{appendix}

  \makeatletter
  \ifustc@bachelor\relax\else
    % 致谢
	
\begin{thanks}

人生就是一个关于成长的漫长故事。而在中科大求学作为本人人生体验的一部分,亦是这样的一段故事。在此的俩年半,俯仰之间,科大的“问道”、“学术”于此,让我经历了这样的三段成长:学于师友,安于爱好,观于内心。

“古之学者必有师,师者,所以传道、授业、解惑也”。师友的教诲不可能一直跟着自己,可是他们治学态度却融入了我的人生观。授课的华保健老师的严谨、郭燕老师的认真、丁菁老师的直率、席菁老师的踏实都曾触动我,并给予我前进方向上的指引。

本论文内容为数据挖掘在电商行业的工程实现,因此有一段真实的、贴近数据挖掘领域的实习经历尤为重要。感谢我在苏州国云数据公司实习的 CEO 马晓东学长,让我有机会一窥大数据行业的内幕;感谢我在小米实习的导师方流博士,感谢我在滴滴出行工作的机器学习研究院李佩博士和袁森博士,让我成为大数据挖掘工程师的梦想又更近了一步;感谢我的导师周武旸教授和张四海教授,指导我完成论文。
向师友和书籍学习,是从外界汲取;只有回归到自己的内心和思绪才能沉淀.在每个夜幕深沉或是晨曦初露的时刻里,感受自己情绪的流动,反思自己的取舍得失,然后才有了融于师友和书籍时的奋进。这样的三段成长,如今已是一体,不断地相互印证与反馈!

“逝者如斯夫,不舍昼夜”。成长亦复如是,不断的和昨日的自己告别。但是,一路有你,真好!相会是缘,同行是乐,共事是福!

\vskip 18pt

\begin{flushright}

~~~~\ustc@author~~~~

\today
%\ustc@submitdate

\end{flushright}

\end{thanks}
%硕博致谢部分
    % 发表文章目录
    %\include{chapter/pub}
  \fi
  \makeatother

\end{document}

\begin{cnabstract}
本文是中国科学技术大学本硕博毕业论文模板示例文件。本模板由ywg@USTC创建,适用于撰写学士、硕士和博士学位论文,本模板由原来的本科模板和硕博模板整合优化而来。本示例文件除了介绍本模板的基础用法外,本文还是一个简要的学位论文写作指南。

\keywords{中国科学技术大学\enskip 学位论文\enskip \LaTeX{}~通用模板\enskip 学士\enskip 硕士\enskip 博士}
\end{cnabstract}

\begin{enabstract}
Because of the dynamic evolution and propagation of the Internet and the popularization of social networks, huge amounts of information about the users's behaviour are accessible to commercial internet companies. Most companies can be easily approaching OOS(open source software) such as hadoop to analyse and interpreting users's behaviour data-set,which just several years ago was not available or inaccessible. Also because of the evolution and popularization of the Internet and the vast amount of data stored, it has become desirable to moderate and select the content that is being displayed to the user,Mechanisms presenting goods on application try to present content that can interest the user based on his previous queries or browsing history. commercial internet companies such as e.g. Xiaomi filter the phone theme application visiting information and show only phone themes that the user potentially be interested in, trying to sell additional goods by recommender products based on user previous purchase behaviours.

Based on my working experience on Xiaomi as internship, The biggest challenge of recommender systems in commercial internet companies is: How to optimize a recommender system in accordance with the true business objective. for commercial internet company like Xiaomi the fitness recommender system is something that mix the social part, the long-tail, the cold-start and many other factors to finally aproximate what the user really wants. It's really a fascinating and complex working. 

The aim of this paper is analyse the long tail feature and cold start feature of a android phone theme recommender system with the help of user profiles and user interests exploration. In order to build the user profile, Information Retrieval (IR) and Data Mining (DM) techniques such as content-based Collaborative Recommendation and text pre-processing methods have been used; In order to approaching discovery and recommendation in long tail, user interests exploration has been refered. Because of the subjective nature of such a solution, verification process such as A/B test is also introduced.

\enkeywords{recommender systems, long tail, cold start, user profile, user interests exploration}
\end{enabstract}

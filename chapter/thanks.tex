
\begin{thanks}

人生就是一个关于成长的漫长故事。而在中科大求学作为本人人生体验的一部分,亦是这样的一段故事。在此的俩年半,俯仰之间,科大的“问道”、“学术”于此,让我经历了这样的三段成长:学于师友,安于爱好,观于内心。

“古之学者必有师,师者,所以传道、授业、解惑也”。师友的教诲不可能一直跟着自己,可是他们治学态度却融入了我的人生观。授课的华保健老师的严谨、郭燕老师的认真、丁菁老师的直率、席菁老师的踏实都曾触动我,并给予我前进方向上的指引。

本论文内容为数据挖掘在电商行业的工程实现,因此有一段真实的、贴近数据挖掘领域的实习经历尤为重要。感谢我在苏州国云数据公司实习的 CEO 马晓东学长,让我有机会一窥大数据行业的内幕;感谢我在小米实习的导师方流博士,感谢我在滴滴出行工作的机器学习研究院李佩博士和袁森博士,让我成为大数据挖掘工程师的梦想又更近了一步。
向师友和书籍学习,是从外界汲取;只有回归到自己的内心和思绪才能沉淀。在每个夜幕深沉或是晨曦初露的时刻里,感受自己情绪的流动,反思自己的取舍得失,然后才有了融于师友和书籍时的奋进。这样的三段成长,如今已是一体,不断地相互印证与反馈!

“逝者如斯夫,不舍昼夜”。成长亦复如是,不断的和昨日的自己告别。但是,一路有你,真好!相会是缘,同行是乐,共事是福!

\vskip 18pt

\begin{flushright}

~~~~\ustc@author~~~~

\today
%\ustc@submitdate

\end{flushright}

\end{thanks}

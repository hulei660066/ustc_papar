\chapter{基于用户画像的推荐系统综述}
	\section{引言}
	自从1992年著名的施乐公司的科学家们为了解决困扰已久的信息负载问题,第一次从概念上提出协同过滤的算法模型。1998年,林登及其同事们成功申请了item协同过滤技术的专利,经过多年的工程实践,美国电商亚马逊公司的工程师们骄傲的宣称:在公司所有的销售量,推荐系统占比已经占到整个Gross Merchandise Volume的百分之三十以上。不久之后的美国公司Netflix,因为其创始人与前任公司签署有若干年内不得从事同行工作的限制,于是通过举办推荐算法优化竞赛绕开限制,用以开发出更好的推荐算法。此次竞赛吸引了数以千计的团队参与角逐,期间进行了上百种的算法模型组合、优化的尝试,虽然Netflix公司为冠军团队支付了百万美金,但回报是Netflix推荐系统的快速发展以及营收的俩位数增长。其中冠军团队凭借Sigular Value Decomposition和Gavin Potter跨界引入的心理学方法进行的组合算法模型,在诸多优秀团队中脱颖而出。其中,矩阵分解的核心是将一个非常稀疏的用户评分矩阵R分解为两个更小的矩阵:只包含User特性的矩阵P和只包含Item特性的矩阵Q,利用P和Q相乘的结果R'来拟合原来的评分矩阵R,使得矩阵R'在R相同位置之间的损失函数值尽量的小,通过定义一个R和R'之间的距离计算公式(一般为曼哈顿距离),如果矩阵R'是正定矩阵,那么把矩阵分解转化成梯度下降求解的局部最优解,就是全局最优解。与此同时,Pandora、LinkedIn、Hulu等网站在个性化推荐领域都展开你争我抢的竞争势头,使得推荐系统在各个细分行业、垂直领域开始全面开花,都有了不少爆发性进展。但是,对于拥有全品类的综合性购物电商、广告营销,推荐系统的进展还是缓慢,主要原因是因为不同类型的商品,消费者的心态也是不同的,例如大型家电,消费者肯定是先看了又看、选了又选,从价格、定位、功能到噪声比、性价比,大多数都会先做足了调查,才会购买;与此相反,对于日常用品消费者可能眼睛都不眨就购买了,对于这俩种极端的消费情况,推荐系统需要做出截然不同的推荐策略,具体的,单个模型在母婴品类的推荐效果还比较好,但在其他品类就可能很差,很多时候需要根据场景、推荐栏位、品类等不同,设计不同的推荐模型。同时由于用户兴趣随时间会不停的变动\citep{user-interests-explore,user-interests-explore1,user-interests-explore3,user-interests-explore4},需要一种机制,使得推荐系统能定期对数据进行评估、分析,除此之外不同类型的商品有不同的更新频率,这就对推荐系统提出了更加智能化的挑战。还有,如果定期更新模型,则可能会因为计算资源的限制损害推荐的实时性\citep{temporal-cf},因为模型训练需要一定的cpu计算时间,而传统的Hadoop的方法实在是无法进行大的更新频率,spark框架又因为昂贵的内存限制了其应用场景。

	传统推荐算法包括基于人口统计学的推荐\citep{social-filter}、基于商品内容的推荐\citep{content-based}和user-based/item-based的协同过滤\citep{collab-filter}的推荐等都有冷启动问题。基于内容的推荐对物品冷启动问题免疫,但是无法解决用户冷启动问题\citep{cold-start}。

	由此,笔者在实际工程中,针对传统推荐算法的种种弊端,选择了用户画像。伟大的数学家、计算机学家Knuth先生说:如果遇到一个不好搞定的问题,那么就该添加一层中间层,用以屏蔽掉问题。实际上,用户画像作为底层数据仓库和上层推荐系统的缓冲层,起的就是这种作用。

	\section{用户画像的研究现状}
		\subsection{用户画像的组成部分}
		基于内容和用户画像的个性化推荐,有两个实体:内容和用户。需要有一种文本机制联系这两者的东西,我们定义其为标签。内容特征文本化为标签即为内容特征化,用户兴趣文本化标签则称为用户特征化\citep{user-profile,user-profile1,user-profile2,user-profile3,user-profile4}。因此,对于基于用户画像的推荐,主要分为以下几个关键部分:
		\begin{enumerate}[(1)]
		\item 标签库

		标签是联系用户与用户、用户与商品、商品与商品之间的纽带,也是反应用户兴趣的重要数据源,标签的最终用途在于标记用户行为。标签库则是对标签进行聚合的系统,包括对标签的管理、更新等。在用户画像的过程中有一个很重要的概念叫做颗粒度,就是我们的用户画像应该细化到哪种程度。举一个极端的例子,如果“用户画像”最细的颗粒度应该是细到每一个用户每一具体的生活场景中,但是这基本上是一个不可能完成的任务,同时如果用户画像的颗粒度太大,又会影响推荐精度,一般来说,标签是以层级的形式组织的,如体育为一级维度、篮球为二级维度、NBA篮球为三级维度等。

		\item 内容特征化

		内容特征化即给商品打标签。目前有两种方式:人工打标签和机器自动打标签。在实际工程中,主题推荐系统采用人工打标签方式,具体就是提供一个关键字库,供设计师从中选择适当关键字作为作品的标签。

		\item 用户特征化

		用户特征化即为用户打文本标签。通过用户的行为日志和一定的模型算法得到用户的每个标签的权重。用户对内容的行为:点赞、不感兴趣、点击、浏览。对用户的反馈行为如点赞赋予权值1,默认为0,不感兴趣为-1;对于用户的浏览行为,则可使用点击、浏览作为权值。对商品发生的行为可以认为对此商品所有标签的行为。用户的兴趣是时间衰减的,即离当前时间越远的兴趣比重越低。时间衰减函数使用1/[log(t)+1], t为事件发生的时间距离当前时间的大小。要考虑到热门商品会干预用户的标签,需要对其标签进行降权。
		\end{enumerate}

		\subsection{用户画像的构建周期}
		用户画像,即用户信息标签化,就是企业通过收集与分析消费者社会属性、生活习惯、消费行为等主要信息的数据之后,获得用户的数据标签库。构建周期如\autoref{pic:userprofile_process}。
		\begin{figure}
	    \centering
	      \framebox{\includegraphics[scale=0.45]{figures/userprofile_process}}
	      \figcaption{用户画像的构建周期示意图}
	      \label{pic:userprofile_process}
	    \end{figure}
	    \begin{enumerate}[(1)]
	    \item 数据收集

	    数据收集大致分为四类:1、网络行为数据包括页面浏览量、活跃人数、访问时长、浏览注册转化率、注册活跃转换率等。服务内行为数据:点击浏览路径、网页停留时长、滑屏次数、滑屏频率、滑屏时长。用户内容偏好数据:点击、浏览、收藏内容、评价、评分、评论内容、社交内容、品牌偏好等。用户交易数据(交易类服务):购买率、折扣率、导流率、流失率等。收集到的数据没必要是百分之百的准确,大体差不多即可。应用中,具体就是在数据清洗阶段过滤一部分不靠谱的异常值,验证、更新数据这块需要在后面的阶段再做判断,比如某用户在性别一栏填的女,但其语言数据显示其为男的概率更大,根据业务再选择丢弃数据还是更新数据。
	    
	    \item 用户画像基本成型

	    该阶段需要利用用户的基本属性,如性别、地域、年龄,得出用户更高层的抽象概念,如消费能力、忠诚度、活跃度、社交爱好等。因为用户画像永远也无法百分百地拟合现实中的一个人,因此,用户画像需要根据变化的基础数据不断修正已有的更高层的抽象概念,尽可能模拟用户的变化趋势。

	    \item 数据可视化

	    最后是数据可视化分析,这部分是最能体现推荐系统的产出,因为人类对数据不如对图画来的敏感,在此步骤中一般是针对群体做进一步的抽象,按照消费习惯、消费能力、消费偏好把用户归类为一类人,比如可以根据用户对价格的敏感度细分出高价值用户、核心用户、高忠诚用户。而决策层所做出的评估也应该是基于某一群体的分布规律。典型的用户画像如\autoref{pic:user_profile}。
	    \end{enumerate}
		\begin{figure}
	    \centering
	      \framebox{\includegraphics[scale=0.4]{figures/user_profile}}
	      \figcaption{用户画像示意图}
	      \label{pic:user_profile}
	    \end{figure}

		\subsection{用户画像的建模}
		用户画像的建模包括内容标签化和标签权重量化。建模过程:1、内容分析,从原先的物品描述信息中提取有用的信息用一种规范化的标签表示,有时候这种信息源自于作者提供的描述,有时候源自于用户的评价,不管如何,都需要人工做进一步的审核;2、上传、记录用户注册信息,生成用户基本信息,这些信息基本是不会变化的;上传、记录用户行为数据,这些数据是不断变化着的,通常是采用数据挖掘算法从潜在物品集合中取出若干个结果表示用户喜好的模型。例如,一个网页推荐系统,可以通过分析用户过往浏览过的文章,得出用户喜欢浏览类似于范冰冰的花边新闻,如果用户点击了所推荐的文章,则说明分析正确,否则需要根据反馈重新训练模型,从而实现一个反馈-推荐-反馈的闭环;3、推荐系统得出推荐集合后往往需要取topN,因为推荐系统的本质在精不在多。通过定义一个距离算法,匹配用户标签和商品标签的相关度,相关度一般正则为0-1之间,结果是一个二元的离散量:(feature, score)。根据相关度将生成一个用户潜在感兴趣的物品评分列表,然后去掉用户之前看过的商品,取topN即可。例如在电影用户画像的建模中,首先分析用户打分比较高的电影的共同特性,包括导演、演员、风格等,这些电影的标签就会成为此用户画像的一部分,根据打分的多少,给定一个合适的权重值。用户-标签用矩阵A表示,电影-标签用矩阵B表示,A乘B得出矩阵C,C代表了用户与电影之间的相关度,固定一个用户,对所有相关度不为零的电影做排序,取topN即是推荐结果。用户画像建模的根本在于用户标签的获取和权重的定量分析。

		\subsection{用户画像和推荐系统的评测}
		首先,用户画像作为一个工具,只用在运用到某一场景才有意义,并能评估出其产出,因此本节主要介绍推荐系统的评测,根据推荐系统的表现好坏才能评估出用户画像的推荐质量。实际工程中,笔者利用A/B实验对若干组模型进行定量对比。标准的A/B实验是指通过一定的规则把类似的用户群随机分成俩组,采用旧模型的分组叫对照组,采用新模型的分组叫实验组\citep{ab-test}。通过对用户展示不同的模型,得出用户的使用指标,关键是各种转化率,这样仅仅通过对比倆者的转化率即可得出各个模型的优劣。策略实验的难点在于如何找到合适的实验设计方案。通过时间交错能够在一定程度上减少由时间片带来的误差,这样就有一个难题:  如何选择合适长度的时间片。策略实验往往伴随着携带效应(carry-over effects),也就是上一个时间片的策略会对下一个时间片带来影响。笔者和同事们提出一个方案,当选择适当大的时间片的时候,通过A/A实验的数据调整A/B实验的结果,具体来说,如果A/A实验的结果是 0.4\%, A/B实验的结果是 1.2\%。那么我们认为A/A实验是真实的时间片之间的差异, 我们需要用倆者之差的绝对值去调整时间片带来的影响。

	\section{用户画像在推荐系统的应用现状}
	Amazon的仓库里堆着数百万图书,Netflix的服务器中存储有数万部电影,淘宝平台上的小卖家总共拥有8亿件物品,除此之外,这三家公司都保留有数以亿计的用户行为数据。互联网电子商务开始积累了海量的用户数据,然后因为数据量过于庞大,有用信息如金矿中的金子一样很难挖掘利用,与此同时,用户发现常常需要面对过多的选择。心理学研究证实过多的选择会使人犹豫不决,导致消极等待,最终可能放弃消费的决定,这个问题严峻到可以造成肉眼可见的用户流失。近代统计学理论的发展加上最近几年的数据科学和数据挖掘工程的进步,为电子商务平台提供更有效的应对方案:推荐算法。推荐系统在帮助用户解决信息过载问题的同时,提升了企业价值。如今的企业不再局限于传统的推荐功能,通过建立完备的用户画像,推荐系统可以帮助企业更了解用户,在推广、反作弊、精细化运营等领域中发挥重要的作用。
		\subsection{基于用户画像的推荐系统的商业应用}
		\begin{figure}
	    \centering
	      \framebox{\includegraphics[scale=0.45]{figures/recmd_facebook}}
	      \figcaption{Facebook个性化推荐用户界面}
	      \label{pic:recmd_facebook}
	    \end{figure}
		作为全球社交网站中的翘楚,Facebook在很早的时候就预言到了大数据+推荐系统+用户画像的无限前景。Facebook自己的推荐系统就是需要利用分布式计算框架快速的帮助用户找到他们可能感兴趣的人、文章、分析、用户组等。Facebook是个伟大的公司,一直为开源软件贡献着一份力量,最近在其官网就公布了Facebook自己的推荐系统原理、性能及使用情况\citep{recmd-facebook}。Facebook的推荐系统需要面对的数据量应该是所有互联网公司中的数一数二,约包含了1000亿级别的评分数、10亿级别的用户数以及百万级别的虚拟商品,如何在如此庞大的数据规模下,仍然保持良好性能已经成为世界级的难题,而Facebook解决了,通过分析Facebook给出了一些实验的结果,表明,Facebook的系统比传统系统要快10倍左右。目前,该方法已经用到Facebook的多个应用中,包括用户、用户组的推荐。Facebook推荐主页如\autoref{pic:recmd_facebook}。
		
		Facebook的用户画像进展也十分可观,几乎是与推荐系统同步发展。2011年12月,Facebook发布了里程碑式的大数据产品——Timeline,通过开发API接口,允许用户自行编辑个人的时间轴:在什么时间、什么地点做了什么,遇到了谁,可以说在这条时间线记录这个人的全部生活故事。Timeline通过帮用户回忆自己的点点滴滴的同时,完成了用户数据捕获、存储,而一旦拥有了这些历史数据,Facebook就可以做进一步的数据分析、挖掘,这时的Facebook就如同和你从小长大的小伙伴,一个懂你的陌生人。可以说用户留下的数据越多,Facebook就越了解这个人,投放的广告就会更加精准,最终Facebook利用庞大的用户数据生态赚足了钱。

		\subsection{推荐系统的主要方法}
		推荐系统主要有俩种思路:评分预测和TopN预测,核心的目标都是找到最适合用户的候选集合s,从候选集合里挑选目标集合是一个非常复杂的非线性优化问题,通常采用的方案是用局部最优近似非线性最优,通过定义一个的损失函数,选取TopN\citep{recmd-Next}。

		推荐系统的算法基于统计学、概率论、线性代数、微积分技术,找出用户最有可能喜欢的商品,应该是现代互联网电商的明星应用。目前用的比较广泛的推荐算法还属协同过滤推荐算法,其基本思想是根据与他兴趣相近的用户的选择,得出推荐商品候选集,取topN推荐给目标用户,用维度为m×n的矩阵表示所有用户对所有物品的兴趣值,这个值应该是根据用户历史行为数据得出,值越高表示这个用户越喜欢,利用特殊值0表示没有接触过。图中行向量表示某个用户对所有商品的喜爱程度,列向量表示某个商品对所有用户的吸引程度,因此单个元素$U_{ij}$表示用户i对物品j的喜欢程度。协同过滤分为两个阶段:预测阶段和推荐阶段。预测阶段是基于所有原始集商品,预测这个用户有没有可能对其感兴趣,量化为一个数值,只要值不为零即可归为候选集中;推荐是根据预测结果,先去重后去除消费过的商品,然后取去TopN推荐给用户。

		尽管有这么多的优点,协同过滤算法也存在两大问题:1、数据稀疏性。一个大型的电子商务平台一般有百万级别的物品,用户可能接触到的商品占所有商品的百分之一不到,因此用户之间购买过的物品重叠性非常小,以至于没办法做推荐,一个办法是利用算法添补部分值\citep{recmd-slopone}。2、扩展性较差,因为一般来讲,电子商务平台中的商品变动很小,用户流入流出、日益增加、变动很大,基于用户的协同过滤算法需要不停的跟新迭代保证跟上用户变动的步伐。遇到这种情况,可以考虑基于商品的协同过滤算法,其基本思想类似于基于用户的协同过滤算法,只是相似性计算对象是商品,而商品一般变动很小可以忽略不计。如果我们知道物品a和b相似,而一般喜欢a的用户也喜欢b,如果用户A喜欢a,那么我们有很大把握得知A也应该喜欢b,推荐了准没错。而物品之间的相似性比较固定,因此可以一次性计算出物品的相似度,将结果存储到Redis中,推荐时查询Redis即可。

	\section{本章小结}
	本章首先介绍了传统推荐算法的存在的弊端,包括冷启动问题;然后介绍了用户画像的组成部分,包括用户的标签库和商品的标签库;之后介绍了用户画像的构建周期,包括数据收集、用户画像成型和数据可视化;然后介绍了用户画像的建模,包括内容标签化和标签权重量化。
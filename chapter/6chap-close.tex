
\chapter{结束语}
  如果说过去的十年是搜索技术大行其道的十年,那么个性化推荐技术将成为未来十年中最重要的革新之一。目前几乎所有大型的电子商务系统,如Amazon、阿里、小米、滴滴等,都不同程度地使用了各种形式的推荐系统。一个好的推荐系统需要满足的目标有:个性化推荐系统必须能够基于用户之前的口味和喜好提供相关的精确的推荐,而且这种口味和喜欢的收集必须尽量少的需要用户的劳动。推荐的结果必须能够实时计算,这样才能够在用户离开网站前之前获得推荐的内容,并且及时的对推荐结果作出反馈。实时性也是推荐系统与通常的数据挖掘技术显著不同的一个特点。一个完整的推荐系统由三部分构成:用户画像模块,用户行为模块、推荐算法模块。用户画像模块记录了用户长期的信息,刻画用户的基础类型。用户行为模块负责记录能够体现用户喜好的行为,比如购买、下载、评分等。这部分看起来简单,其实需要非常仔细的设计。比如说购买和评分这两种行为表达潜在的喜好程度就不尽相同完善的行为记录需要能够综合多种不同的用户行为,处理不同行为的累加。推荐算法模块的功能则实现了对用户行为记录的分析,采用不同算法建立起模型描述用户的喜好信息,通过推荐模块实时的从内容集筛选出目标用户可能会感兴趣的内容推荐给用户。因此,除了推荐系统本身,为了实现推荐,还需要一个可供推荐的内容集。比如,对于手机主题推荐系统来说,所有上线主题就是这样的内容集。我们对内容集本身需要提供的信息要求非常低,在经典的协同过滤算法下,内容集甚至只需要提供ID就足够。而对于基于内容的推荐系统来说,由于往往需要对内容进行特征抽取和索引,我们就会需要提供更多的领域知识和标签属性。

  推荐系统是一种联系用户和内容的信息服务系统,一方面它能够帮助用户发现他们潜在感兴趣的内容,另一方面它能够帮助内容供者将内容投放给对它感兴趣的用户。推荐系统的主要方法是通过分析用户的历史行为来预测他们未来的行为。因此,时间是影响用户行为的重要因素。关于推荐系统动态特性的研究相对比较少,特别是缺乏系统性的研究。对动态推荐系统的研究,无论是从促进用户兴趣模型的理论角度出发,还是从实际需求来看,都具有重要的意义,本文的研究工作正是在这一背景下展开。

  \section{研究工作总结}
    本文对推荐系统特别是与用户画像相关的动态推荐系统的相关工作做了总结和回顾之外,主要的工作包括以下几个方面:
    \begin{itemize}
      \item 设计出了基于用户画像的推荐模型:按照用户属性和行为特征对全部用户进行聚类和精细化的客户群细分,将用户行为相同或相似的用户归类到一个消费群体,这样就可以将推荐平台所有的用户划分为N个不同组,每个组用户拥有相同或相似的行为特征,这样电商平台就可以按照不同组的用户行为对其进行个性化智能推荐。目前国内主流电商平台,在进行个性化智能推荐系统升级过程,都在逐步向DNN渗透和扩展,也是未来个性化智能推荐必经之路。在现有用户画像、用户属性打标签、客户和营销规则配置推送、同类型用户特性归集分库模型基础上,未来将逐步扩展机器深度学习功能,通过系统自动搜集分析前端用户实时变化数据,依据建设的机器深度学习函数模型,自动计算匹配用户需求的函数参数和对应规则,推荐系统根据计算出的规则模型,实时自动推送高度匹配的营销活动和内容信息。
      \item 设计出了考虑用户长期兴趣和短期兴趣的用户兴趣探索模型:通过量化用户满意度和量化量化主题标签的流行度,最大化推荐系统的推荐多样性和长尾性,提高用户的惊喜度。
      \item 动态推荐系统的原型设计:综合前几章的推荐系统各个模块的研究,设计了一个实际的推荐系统的原型。该系统包含了用户画像和用户兴趣探索模型。能够实时根据用户行为变化的趋势,实时的调整推荐结果排名,从而不断改善用户在推荐系统中的体验。
    \end{itemize}

  \section{对未来工作的展望}
  本文对推荐系统的用户画像和用户兴趣探索模型进行了较深入的研究,但是针对用户兴趣变化的推荐模型的实现还有很多工作要做。本人认为用户兴趣动态推荐系统的实现有待解决的问题如下:
    \begin{itemize}
      \item 用户行为的离线和在线计算的分配:用户行为每天产生的数据量很大,哪些行为需要在线实时计算反馈,哪些行为只需要离线计算即可,需要根据具体业务的特点和用户习惯赋予每种行为一个权重,然后根据权重排名决定计算方式。因此,用户行为的特征提取、分析将是我们将来工作的一个重要方面。
      \item 用户兴趣探索模型对推荐系统的影响:本文的所有工作基本集中在高推荐系统的点击购买转换率上。但点击购买转换率并不是推荐系统追求的唯一指标。比如,预测用户可能会去看,从而给用户推荐速度与激情,这并不是一个好的推荐。因为速度与激情的热度很高,因此并不需要别人给他们推荐。上面这个例子涉及到了推荐系统的长尾度,即用户希望推荐系统能够给他们新颖的推荐结果,而不是那些他们已经知道的物品。此外,推荐系统还有多样性等指标。如何利用时间信息,在不牺牲转换率的同时, 提高推荐的其他指标,是我们将来工作研究的一个重要方面。
      \item 推荐系统随时间的进化:用户的行为和兴趣是随时间变化的,意味着推荐系统本身也是一个不断演化的系统。其各项指标,包括长尾度,多样性,点击率都是随着数据的变化而演化。如何让推荐系统能够通过利用实时变化的用户反馈,向更好的方面发展是推荐系统研究的一个重要方面。
    \end{itemize}

  最后,希望本文的研究工作能够对动态推荐系统的发展作出一定的贡献,并真诚的希望老师们出宝贵的批评意见和建议。
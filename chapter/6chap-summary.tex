
\chapter{结束语}
  如果说过去的五年是推荐系统大行其道的时间,那么个性化将成为推荐技术未来五年中最重要的革新之一。一个好的推荐系统需要满足的目标有:在尽可能少的用户参与下,实时提供个性推荐服务,推荐结果满足用户-商品的新颖性、惊喜性和长尾性。推荐结果必须足够及时,这样才能在用户浏览之后、购买之前就获得推荐服务。完整的推荐系统由三部分模块组成:用户画像模块,用户行为挖掘模块、推荐引擎模块。用户画像模块记录了用户长期的信息,刻画用户的基础类型。用户行为挖掘模块负责记录能够体现用户喜好的行为,比如购买、下载、评分等。这部分看起来简单,其实需要非常仔细的设计。比如说购买和评分这两种行为表达潜在的喜好程度就不尽相同。完善的行为记录需要能够综合多种不同的用户行为,处理不同行为的累加。推荐引擎模块的功能则实现了对用户行为记录的分析、计算、排序。

  推荐系统的主要方法是通过分析用户过去的行预测未来,因此基于用户画像模型的研究是一个很好的突破口。关于推荐系统动态特性的研究相对比较少,特别是缺乏系统性的研究。对动态推荐系统的研究,无论是从促进用户兴趣模型的理论角度出发,还是从实际需求来看,都具有重要的意义,本文的研究工作正是在这一背景下展开。

  \section{研究工作总结}
    本文对推荐系统特别是与用户画像相关的动态推荐系统的相关工作做了总结和回顾之外,主要的工作包括以下几个方面:
    \begin{itemize}
      \item 设计了用户画像模型:按照用户属性和行为特征对全部用户进行聚类和精细化的客户群细分,将用户行为相同或相似的用户归类到一个消费群体,这样就可以将推荐平台所有的用户划分为N个不同组,每个组用户拥有相同或相似的行为特征,这样电商平台就可以按照不同组的用户行为对其进行个性化智能推荐。在现有用户画像、用户属性打标签、客户和营销规则配置推送、同类型用户特性归集分库模型基础上,未来将逐步扩展机器深度学习功能,通过系统自动搜集分析前端用户实时变化数据,依据建设的机器深度学习函数模型,自动计算匹配用户需求的函数参数和对应规则,推荐系统根据计算出的规则模型,实时自动推送高度匹配的营销活动和内容信息。
      \item 设计了用户兴趣探索模型:模型能够根据用户行为变化的趋势,实时的调整推荐结果排名,从而不断改善用户在推荐系统中的体验。
      \item 利用线性衰减算法成功融合用户长期兴趣和短期兴趣:本文在研究用户画像建模和用户兴趣探索的基础上,结合电子商务参与者兴趣偏好变化频繁的特点,提出了基于线性衰减的用户兴趣融合模型。该模型采用一个0到10的数值表示用户偏好,表示用户对每个标签的喜好程度,权重值根据时间进行线性衰减,以反映用户兴趣的变化。
    \end{itemize}

  \section{对未来工作的展望}
  本文对推荐系统的用户画像和用户兴趣探索模型进行了较深入的研究,但是针对用户兴趣变化的推荐模型的实现还有很多工作要做。本人认为推荐系统有待解决的问题有:
    \begin{itemize}
      \item 用户行为的离线和在线计算的分配:用户行为每天产生的数据量很大,哪些行为需要在线实时计算反馈,哪些行为只需要离线计算即可,需要根据具体业务的特点和用户习惯赋予每种行为一个权重,然后根据权重排名决定计算方式。因此,用户行为的特征提取、分析将是我们将来工作的一个重要方面。
      \item 用户兴趣探索模型对推荐系统的影响:本文的所有工作的评估集中在点击购买转换率上。但点击购买转换率并不是推荐系统追求的唯一指标,比如,预测用户可能会去看,从而给用户推荐热门商品,这并不是一个好的推荐。因为热门商品本身的转化率就很高,这里涉及到了推荐系统的长尾度,即用户希望推荐系统能够给他们新颖的推荐结果,而不是那些他们已经知道的物品。此外,推荐系统还有多样性等指标。如何利用时间信息,在不牺牲转换率的同时, 提高推荐的其他指标,是笔者将来工作研究的一个重要方面。
      \item 推荐系统随时间的进化:用户的行为和兴趣是随时间变化的,意味着推荐系统本身也是一个不断演化的系统。其各项指标,包括长尾度,多样性,点击率都是随着数据的变化而演化。如何让推荐系统能够通过利用实时变化的用户反馈,向更好的方面发展是推荐系统研究的一个重要方面。
    \end{itemize}

  最后,希望本文的研究工作能够对动态推荐系统的发展作出一定的贡献,并真诚的希望老师们出宝贵的批评意见和建议。
\begin{cnabstract}
信息爆炸使得用户很难有效的从海量的数据中快速获取自己需要的信息,推荐系统凭借精准定位和"千人千面"的个性化服务受到互联网企业的青睐和研究者的重视。本论文讨论了如何构建一个基于用户画像模块和用户兴趣探索模块的手机主题推荐系统,并详细介绍了用户画像建模和用户兴趣探索。

在小米科技的工作经历让笔者意识到传统的个性化推荐系统面临着诸多挑战,其中最根本的问题是如何根据企业的商业目标和业务特点来优化推荐系统,具体到手机主题行业,推荐系统需要解决社交化、长尾性、冷启动、动态推荐等一系列综合问题。由此,笔者提出并实现了一种基于用户画像和用户兴趣探索的手机主题个性化推荐系统。本文的主要工作和贡献如下:
\begin{itemize}
	\item 实现了推荐系统的用户画像模块:利用信息检索(IR)技术从用户注册信息获取到用户的人口属性、职业、地理位置、性别等信息并标签化,不同标签的来源,标签的传递路径,转发关系,标签的本身,以及标签与用户之间的共现关系决定了这个标签对应的权重,权重越高则认为该标签的可信度越高。AB测试显示结合了用户画像的推荐系统提升了约8\%的点击转换率。
	\item 实现了推荐系统的用户兴趣探索模块:为了迎合用户不断变化着的兴趣,用户兴趣探索使用了特征提取技术和用户满意度量化技术,对每个用户维护一个能动态变化的短期兴趣标签向量空间。首先,利用用户兴趣特征向量和商品特征向量计算出用户-商品的相关分数。然后,利用用户行为(购买,评分,点赞,划屏频率等)量化用户满意度。一次成功的用户兴趣标签探索,首先应该有很低的相关分数和很高的满意度,其次兴趣标签应该是一个小众兴趣标签。实验表明示结合了用户兴趣探索的推荐系统能显著提升推荐商品的多样性。
	\item 系统地研究了时效性对推荐系统的影响:用户画像针对的是用户的静态信息,代表了用户的长期兴趣,用户兴趣探索针对的是用户的动态信息,代表了用户的短期兴趣,对于不同的行业俩者对用户行为的影响程度也不同。本文提出了基于时间的线性衰减模型能有效融合用户的长、短期兴趣。
	\item 设计了融合用户画像和用户兴趣探索的原型推荐系统:论文在总结本人在用户画像、用户兴趣探索工作的基础上设计了动态推荐系统。该系统能够实时反馈用户的最新行为,并根据用户行为的变化自动探索出用户新的兴趣,从而不断改善用户在推荐系统中的体验。
\end{itemize}

\keywords{推荐系统\enskip 长尾效应\enskip 动态兴趣\enskip 用户画像建模\enskip 用户兴趣探索\enskip}
\end{cnabstract}

\begin{enabstract}


\enkeywords{recommender systems, long tail, dynamic, user profile, user interests exploration}
\end{enabstract}
